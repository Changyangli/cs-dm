%% Generated by Sphinx.
\def\sphinxdocclass{report}
\documentclass[letterpaper,10pt,english]{sphinxmanual}
\ifdefined\pdfpxdimen
   \let\sphinxpxdimen\pdfpxdimen\else\newdimen\sphinxpxdimen
\fi \sphinxpxdimen=.75bp\relax

\usepackage[utf8]{inputenc}
\ifdefined\DeclareUnicodeCharacter
 \ifdefined\DeclareUnicodeCharacterAsOptional
  \DeclareUnicodeCharacter{"00A0}{\nobreakspace}
  \DeclareUnicodeCharacter{"2500}{\sphinxunichar{2500}}
  \DeclareUnicodeCharacter{"2502}{\sphinxunichar{2502}}
  \DeclareUnicodeCharacter{"2514}{\sphinxunichar{2514}}
  \DeclareUnicodeCharacter{"251C}{\sphinxunichar{251C}}
  \DeclareUnicodeCharacter{"2572}{\textbackslash}
 \else
  \DeclareUnicodeCharacter{00A0}{\nobreakspace}
  \DeclareUnicodeCharacter{2500}{\sphinxunichar{2500}}
  \DeclareUnicodeCharacter{2502}{\sphinxunichar{2502}}
  \DeclareUnicodeCharacter{2514}{\sphinxunichar{2514}}
  \DeclareUnicodeCharacter{251C}{\sphinxunichar{251C}}
  \DeclareUnicodeCharacter{2572}{\textbackslash}
 \fi
\fi
\usepackage{cmap}
\usepackage[T1]{fontenc}
\usepackage{amsmath,amssymb,amstext}
\usepackage{babel}
\usepackage{times}
\usepackage[Bjarne]{fncychap}
\usepackage[dontkeepoldnames]{sphinx}

\usepackage{geometry}

% Include hyperref last.
\usepackage{hyperref}
% Fix anchor placement for figures with captions.
\usepackage{hypcap}% it must be loaded after hyperref.
% Set up styles of URL: it should be placed after hyperref.
\urlstyle{same}
\addto\captionsenglish{\renewcommand{\contentsname}{Contents:}}

\addto\captionsenglish{\renewcommand{\figurename}{Fig.}}
\addto\captionsenglish{\renewcommand{\tablename}{Table}}
\addto\captionsenglish{\renewcommand{\literalblockname}{Listing}}

\addto\captionsenglish{\renewcommand{\literalblockcontinuedname}{continued from previous page}}
\addto\captionsenglish{\renewcommand{\literalblockcontinuesname}{continues on next page}}

\addto\extrasenglish{\def\pageautorefname{page}}

\setcounter{tocdepth}{1}


\usepackage{amsmath}
    \usepackage{cases}


\title{Mathematical Logic in Software Development Documentation}
\date{Jan 29, 2018}
\release{1}
\author{Kevin Sullivan}
\newcommand{\sphinxlogo}{\vbox{}}
\renewcommand{\releasename}{Release}
\makeindex

\begin{document}

\maketitle
\sphinxtableofcontents
\phantomsection\label{\detokenize{index::doc}}



\chapter{Requirement, Specifications, and Implementations}
\label{\detokenize{01-reqs-specs-impls:requirement-specifications-and-implementations}}\label{\detokenize{01-reqs-specs-impls::doc}}\label{\detokenize{01-reqs-specs-impls:welcome-to-mathematical-logic-in-software-development}}
Software is an increasingly critical component of major societal
systems, from rockets to power grids to healthcare, etc. Failures are
not always bugs in implementation code. The most critical problems
today are not in implementations but in requirements and
specifications.
\begin{itemize}
\item {} 
\sphinxstylestrong{Requirements:} Statements of the effects that a system is meant to have in a given domain

\item {} 
\sphinxstylestrong{Specification:} Statements of the behavior required of a machine to produce such effects

\item {} 
\sphinxstylestrong{Implementation:} The definition (usually in code) of how a machine produces the specified behavior

\end{itemize}

Avoiding software-caused system failures requires not only a solid
understanding of requirements, specifications, and implementations,
but also great care in both the \sphinxstyleemphasis{validation} of requirements and of
specifications, and \sphinxstyleemphasis{verification} of code against specifications.
\begin{itemize}
\item {} 
\sphinxstylestrong{Validation:} \sphinxstyleemphasis{Are we building the right system?} is the specification right; are the requirements right?

\item {} 
\sphinxstylestrong{Verification:} \sphinxstyleemphasis{Are we building the system right?} Does the implementation behave as its specification requires?

\end{itemize}

You know that the language of implementation is code. What is the
language of specification and of requirements?

One possible answer is \sphinxstyleemphasis{natural language}. Requirements and
specifications can be written in natural languages such as English or
Mandarin. The problem is that natural language is subject to
ambiguity, incompleteness, and inconsistency. This makes it a risky
medium for communicating the precise behaviors required of complex
software artifacts.

The alternative to natural language that we will explore in this class
is the use of mathematical logic, in particular what we call propositional
logic, predicate logic, set theory, and the related field of type theory.

Propositional logic is a language of simple propositions. Propositions
are assertions that might or might not be judged to be true. For
example, \sphinxstyleemphasis{Tennys (the person) plays tennis} is actually a true
proposition (if we interpret \sphinxstyleemphasis{Tennys} to be the person who just played
in the French Open).  So is \sphinxstyleemphasis{Tennys is from Tennessee}. And because
these two propositions are true, so is the \sphinxstyleemphasis{compound} proposition (a
proposition built up from smaller propositions) that Tennys is from
Tennessee \sphinxstylestrong{and} Tennys plans tennis.

Sometimes we want to talk about whether different entities satisfy
give propositions. For this, we introduce propositions with parameters,
which we will call \sphinxstyleemphasis{properties}. If we take \sphinxstyleemphasis{Tennys} out of \sphinxstyleemphasis{Tennys
plays tennis} and replace his name by a variable, \sphinxstyleemphasis{P}, that can take
on the identify of any person, then we end up with a parameterized
proposition, \sphinxstyleemphasis{P plays tennis}. Substituting the name of any particular
person for \sphinxstyleemphasis{P} then gives us a proposition \sphinxstyleemphasis{about that person} that we
can judge to be true or false. A parameterized proposition thus gives
rise to a whole family of propositions, on for each possible value of
\sphinxstyleemphasis{P}.

Sometimes we write parameterized propositions so that they look like
functions, like this: \sphinxstyleemphasis{PlaysTennis(P)}. \sphinxstyleemphasis{PlaysTennis(Tennys)} is thus
the proposition, \sphinxstyleemphasis{Tennys plays Tennis} while \sphinxstyleemphasis{PlaysTennis(Kevin)} is
the proposition \sphinxstyleemphasis{Kevin plays Tennis}. For each possible person name,
\sphinxstyleemphasis{P}, there is a corresponding proposition, \sphinxstyleemphasis{PlaysTennis(P)}.

Some such propositions might be true. For instance,
\sphinxstyleemphasis{PlaysTennis(Tennys)} is true in our example. Others might be false. A
parameterized proposition thus encodes a \sphinxstyleemphasis{property} that some things
(here people) have and that others don’t have (here, the property of
\sphinxstyleemphasis{being a tennis player}).

A property, also sometimes called a \sphinxstyleemphasis{predicate}, thus also serves to
identify a \sphinxstyleemphasis{subset} of elements in a given \sphinxstyleemphasis{domain of discourse}. Here
the domain of discourse is the of all people. The subset of people who
actually do \sphinxstyleemphasis{play tennis} is exactly the set of people, P, for whom
\sphinxstyleemphasis{PlaysTennis(P)} is true.

We note briefly, here, that, like functions, propositions can have
multiple parameters. For example, we can generalize from \sphinxstyleemphasis{Tennys plays
Tennis **and*} Tennys is from Tennessee* to \sphinxstyleemphasis{P plays tennis and P is
from L,} where P ranges over people and L ranges over locations. We
call a proposition with two or more parameters a \sphinxstyleemphasis{relation}. A
relation picks out \sphinxstyleemphasis{combinations} of elements for which corresponding
properties are true. So, for example, the \sphinxstyleemphasis{pair} (Tennys, Tennessee)
is in the relation (set of \sphinxstyleemphasis{P-L} pairs) picked out by this
parameterized proposition. On the other hand, the pair, (Kevin,
Tennessee), is not, because Kevin is actually from New Hampshire, so
the proposition \sphinxstyleemphasis{Kevin plays tennis **and*} Kevin is from Tennessee*
is not true. More on relations later!


\chapter{Logic and Code}
\label{\detokenize{02-logic-and-code::doc}}\label{\detokenize{02-logic-and-code:logic-and-code}}
We’ve discussed requirements, specifications, and implementations as
software artifacts serving distinct purposes. For good reasons, these
artifacts are generally written in different languages. In this unit,
we discuss these different kinds of languages—mathematical logic for
specifications and imperative languages for code—why they are used
for different purposes, the fundamental advantages and disadvantages
of each, and why modern software development requires fluence in and
tools for handling artifacts written in multiple such languages.


\section{Imperative Implementations and Declarative Specifications}
\label{\detokenize{02-logic-and-code:imperative-implementations-and-declarative-specifications}}
The language of implementations is code in what we call an \sphinxstyleemphasis{imperative
programming language}. Examples of such languages include Python,
Java, C++, and Javascript. The most salient property of such a
language is that it is \sphinxstyleemphasis{procedural}. Programs in these languages
describe step-by-step \sphinxstyleemphasis{procedures}, in the form of sequences of
\sphinxstyleemphasis{commands}, for solving given problem instances. Commands in such
languages operate by (1) reading, computing with, and updating values
stored in a \sphinxstyleemphasis{mutable memory}, and (2) interacting with the world
outside of the computer by executing input and output (IO) commands.

The language of formal requirements and specifications, on the other
hand, is some kind of \sphinxstyleemphasis{mathematical logic}. Examples of logics that we
will study and use include \sphinxstyleemphasis{propositional} and \sphinxstyleemphasis{predicate} logic.  An
example of a kind of logic important in software development but that
we will not study in this class is \sphinxstyleemphasis{temporal logic.}

For purposes of software specification, the most salient property of
such a logical language is that it is \sphinxstyleemphasis{declarative}.  Expressions in
logic will state \sphinxstyleemphasis{what} properties or relationships must hold in a
given situation, particularly how results must relate to inputs,
without providing executable, step-by-step procedures describing \sphinxstyleemphasis{how}
to actually compute such relationships.

To make the difference between procedural and declarative styles of
description clear, consider the problem of computing the positive
square root of a given non-negative number, \sphinxstyleemphasis{x}. We can \sphinxstyleemphasis{specify} the
answer in a clear and precise logical style by simply stating that,
for any given non-negative number \sphinxstyleemphasis{x}, we require a value, \sphinxstyleemphasis{y}, such
that \(y^2 = x\). We would write this mathematically as
\(\forall x \mid x >= 0, sqrt(x) = y | y^2 = x\). In English, we’d
pronounce this formula as, “for any \sphinxstyleemphasis{x} where \sphinxstyleemphasis{x} is greater than or
equal to zero, the square root of \sphinxstyleemphasis{x} is a value \sphinxstyleemphasis{y} such that \sphinxstyleemphasis{y}
squared is equal to \sphinxstyleemphasis{x}.”

We now have a \sphinxstyleemphasis{declarative specification} of the desired relationship
between \sphinxstyleemphasis{x} and \sphinxstyleemphasis{y}. What we don’t have, however, is a step-by-step
\sphinxstyleemphasis{procedure} for computing this relation by finding a value of \sphinxstyleemphasis{y} for
any given value of \sphinxstyleemphasis{x}. You can’t just run a specification written in
the language of mathematical logic.

The solution is to shift from mathematics as a specification language
to imperative code as an implementation language.  In such a language,
we then craft a step-by-step procedure that, when run, computes the
results we seek. Here’s an example of a program in the imperative
language, Python, for computing positive square roots of non-negative
numbers using Newton’s method.

\begin{sphinxVerbatim}[commandchars=\\\{\}]
\PYG{k}{def} \PYG{n+nf}{sqrt}\PYG{p}{(}\PYG{n}{x}\PYG{p}{)}\PYG{p}{:}
    \PYG{l+s+sd}{\PYGZdq{}\PYGZdq{}\PYGZdq{}for x\PYGZgt{}=0, return non\PYGZhy{}negative y such that y\PYGZca{}2 = x\PYGZdq{}\PYGZdq{}\PYGZdq{}}
    \PYG{n}{estimate} \PYG{o}{=} \PYG{n}{x}\PYG{o}{/}\PYG{l+m+mi}{2}
    \PYG{k}{while} \PYG{n+nb+bp}{True}\PYG{p}{:}
        \PYG{n}{newestimate} \PYG{o}{=} \PYG{p}{(}\PYG{p}{(}\PYG{n}{estimate}\PYG{o}{+}\PYG{p}{(}\PYG{n}{x}\PYG{o}{/}\PYG{n}{estimate}\PYG{p}{)}\PYG{p}{)}\PYG{o}{/}\PYG{l+m+mi}{2}\PYG{p}{)}
        \PYG{k}{if} \PYG{n}{newestimate} \PYG{o}{==} \PYG{n}{estimate}\PYG{p}{:}
            \PYG{k}{break}
        \PYG{n}{estimate} \PYG{o}{=} \PYG{n}{newestimate}
    \PYG{k}{return} \PYG{n}{estimate}
\end{sphinxVerbatim}

This procedure initializes and then repeatedly updates the values
stored at two locations in memory, referred to by the two variables,
\sphinxstyleemphasis{estimate} and \sphinxstyleemphasis{newestimate}. It repeats the update process until the
process \sphinxstyleemphasis{converges} on the answer, which occurs when the values of the
two variables become equal. The answer is then returned to the caller
of this procedure.

Note that, following good programming style, we included an English
rendering of the specification as a document string in the second line
of the program.  There are however several deep problems with this
approach. First, as we’ve discussed, natural language is subject to
ambiguity, inconsistency, and incompleteness. Second, because the
document string is just a comment, there’s no way for the compiler to
check consistency between the code and this specification. Third, in
practice, code evolves (changes over time), and in their rush to ship
code, developers often forget, or neglect, to update comments. So, in
practice, even if a given procedure is initially consistent with a
specification given as comment, inconsistencies can and often do
develop over time.


\section{Why Not a Single Language for both Programming and Specification?}
\label{\detokenize{02-logic-and-code:why-not-a-single-language-for-both-programming-and-specification}}
The dichotomy between specification logic and implementation code
raises an important question? Why not just design a single language
that’s good for both?

The answer is that there are fundamental tradeoffs in language design.
One of the most important is a tradeoff between \sphinxstyleemphasis{expressiveness}, on
one hand, and \sphinxstyleemphasis{efficient execution}, on the other.

What we see in our square root example is that mathematical logic is
highly \sphinxstyleemphasis{expressive}. Logic language can be used so say very clearly
\sphinxstyleemphasis{what} we want. On the other hand, it’s hard using logic to say \sphinxstyleemphasis{how}
to get it. In practice, mathematical logic is clear but can’t be \sphinxstyleemphasis{run}
(at least not efficiently).

On the other hand, imperative code states \sphinxstyleemphasis{how} a computation is to be
carried out, but enerally doesn’t make clear \sphinxstyleemphasis{what} it’s computing. You
would be hard-pressed, based on a quick look at the Python code above,
to explain \sphinxstyleemphasis{what} it does (but for the fact that we embedded the spec
into the code as a doc string).

We are driven to a situation in which we have to express what we want
and how to get it, respectively, in very different languages. This
situation creates a difficult new problem: to verify that a program
written in an imperative language satisfies a specification written in
a declarative language.  This is the problem of \sphinxstyleemphasis{verification}. Have
we built a program right (where right is defined by a specification)?


\chapter{Problems with Imperative Code}
\label{\detokenize{03-problems-with-imperative-code::doc}}\label{\detokenize{03-problems-with-imperative-code:problems-with-imperative-code}}
There’s no free lunch: One can have the expressiveness of mathematical
logic, useful for specification, or one can have the ability to run
code efficiently, along with indispensable ability to interact with an
external environment provided by imperative code, but one can not have
all of this at once at once.

A few additional comments about expressiveness are in order here. When
we say that imperative programming languages are not as expressive as
mathematical logic, what we mean is not ony that the code itself is not
very explicit about what it computes. It’s also that it is profoundly
hard to fully comprehend what imperative code will do when run, in large
part due precisely to the things that make imperative code efficient: in
particular to the notion of a mutable memory.

One major problem is that when code in one part of a complex program
updates a variable (the \sphinxstyleemphasis{state} of the program), another part of the
code, far removed from the first, that might not run until much later,
can read the value of that very same variable and thus be affected by
actions taken much earlier by code far away in the program text. When
programs grow to thousands or millions of lines of code (e.g., as in
the cases of the Toyota unintended acceleration accident that we read
about), it can be incredibly hard to understand just how different and
seemingly unrelated parts of a system will interact.

As a special case, one execution of a procedure can even affect later
executions of the same procedure. In pure mathematics, evaluating the
sum of two and two \sphinxstyleemphasis{always} gives four; but if a procedure written in
Python updates a \sphinxstyleemphasis{global} variable and then incoporates its value into
the result the next time the procedure is called, then the procedure
could easily return a different result each time it is called even if
the argument values are the same. The human mind is simpl not powerful
enough to see what can happen when computations distant in time and in
space (in the sense of being separated in the code) interact with each
other.

A related problem occurs in imperative programs when two different
variables, say \sphinxstyleemphasis{x} and \sphinxstyleemphasis{y}, refer to the same memory location. When
such \sphinxstyleemphasis{aliasing} occurs, updating the value of \sphinxstyleemphasis{x} will also change the
value of \sphinxstyleemphasis{y}, even though no explicit assignment to \sphinxstyleemphasis{y} was made. A
peice of code that assumes that \sphinxstyleemphasis{y} doesn’t change unless a change is
made expliticly might fail catastrophically under such circumstances.
Aliasing poses severe problems for both human understanding and also
machine analysis of code written in imperative languages.

Imperative code is thus potentially \sphinxstyleemphasis{unsafe} in the sense that it can
not only be very hard to fully understand what it’s going to do, but
it can also have effects on the world, e.g., by producing output
directing some machine to launch a missile, fire up a nuclear reactor,
steer a commercial aircraft, etc.


\chapter{Pure Functional Programming as Runnable Mathematics}
\label{\detokenize{04-runnable-math:pure-functional-programming-as-runnable-mathematics}}\label{\detokenize{04-runnable-math::doc}}
What we’d really like would be a language that gives us everything:
the expressiveness and the \sphinxstyleemphasis{safety} of mathematical logic (there’s no
concept of a memory in logic, and thus no possibility for unexpected
interactions through or aliasing of memory), with the efficiency and
interactivity of imperative code. Sadly, there is no such language.

Fortunately, there is an important point in the space between these
extremes: in what we call \sphinxstyleemphasis{pure functional,} as opposed to imperative,
\sphinxstyleemphasis{programming} languages. Pure functional languages are based not on
commands that update memories and perform I/O, but on the definition
of functions and their application to data values. The expressiveness
of such languages is high, in that code often directly refects the
mathematical definitions of functions. And because there is no notion
of an updateable (mutable) memory, aliasing and interactions between
far-flung parts of programs through \sphinxstyleemphasis{global variables} simply cannot
happen. Furthermore, one cannot perform I/O in such languages. These
languages thus provide far greater safety guarantees than imperative
languages.  Finally, unlike mathematical logic, code in functional
languages can be run with reasonable efficiency, though often not with
the same efficiency as in, say, C++.

In this chapter, you will see how functional languages allow one to
implement runnable programs that closely mirror the mathematical
definitions of the functions that they implement.


\section{The identify function (for integers)}
\label{\detokenize{04-runnable-math:the-identify-function-for-integers}}
An \sphinxstyleemphasis{identity function} is a function whose values is simply the value
of the argument to which it is applied. For example, the identify
function applied to an integer value, \sphinxstyleemphasis{x}, just evaluates to the value
of \sphinxstyleemphasis{x}, itself. In the language of mathematical logic, the definition
of the function would be written like this.
\begin{equation*}
\begin{split}\forall x \in \mathbb{Z}, x.\end{split}
\end{equation*}
In English, this would be pronounced, “for all (\(\forall\))
values, \sphinxstyleemphasis{x}, in (\(\in\)) the set of integers
(\(\mathbb{Z}\)), the function simply reduces to value of \sphinxstyleemphasis{x},
itself. The infinite set of integers is usually denoted in
mathematical writing by a script or bold Z. We will use that
convention in these notes.

While such a mathematical definition is not “runnable”, we can
\sphinxstyleemphasis{implement} it as a runnable program in pure functional language. The
code will then closely reflects the abstract mathematical definition.
And it will run!  Here’s an implementation of \sphinxstyleemphasis{id} written in the
functional sub-language of Dafny.

\begin{sphinxVerbatim}[commandchars=\\\{\}]
function method id (x: int): int \PYGZob{} x \PYGZcb{}
\end{sphinxVerbatim}

The code declares \sphinxstyleemphasis{id} to be what Dafny calls a “function method”,
which indicates two things.  First, the \sphinxstyleemphasis{function} keyword states that
the code will be written in a pure functional, not in an imperative,
style. Second, the \sphinxstyleemphasis{method} keyword instructs the compiler to produce
runnable code for this function.

Let’s look at the code in detail. First, the name of the function is
defined to be \sphinxstyleemphasis{id}. Second, the function is defined to take just one
argument, \sphinxstyleemphasis{x}, declared of type \sphinxstyleemphasis{int}.  The is the Dafny type whose
values represent integers (negative, zero, and positive whole number)
of any size. The Dafny type \sphinxstyleemphasis{int} thus represents (or \sphinxstyleemphasis{implements})
the mathematical set, \({\mathbb Z}\), of all integers. The \sphinxstyleemphasis{int}
after the argument list and colon then indicates that, when applied to
an int, the function returns (or \sphinxstyleemphasis{reduces to}) a value of type \sphinxstyleemphasis{int}.
Finally, within the curly braces, the expression \sphinxstyleemphasis{x}, which we call
the \sphinxstyleemphasis{body} of this function definition, specifies the value that this
function reduces to when applied to any \sphinxstyleemphasis{int}. In particular, when
applied to avalue, \sphinxstyleemphasis{x}, the function application simply reduces to the
value of \sphinxstyleemphasis{x} itself.

Compare the code with the abstract mathematical definition and you
will see that but for details, they are basicaly \sphinxstyleemphasis{isomorphic} (a word
that means identical in structure). It’s not too much of a stretch to
say that pure functional programs are basically runnable mathematics.

Finally, we need to know how expressions involving applications of
this function to arguments are evaluated. They fundamental notion at
the heart of functional programming is this: to evaluate a function
application expression, such as \sphinxstyleemphasis{id(4)}, you substiute the value of
the argument (here \sphinxstyleemphasis{4}) for every occurence of the argument variable
(here \sphinxstyleemphasis{x}) in the body of the function definition, the you evaluate
that expression and return the result. In this case, we substite \sphinxstyleemphasis{4}
for the \sphinxstyleemphasis{x} in the body, yielding the literal expression, \sphinxstyleemphasis{4}, which,
when evaluated, yeilds the value \sphinxstyleemphasis{4}, and that’s the result.


\section{Data and function types}
\label{\detokenize{04-runnable-math:data-and-function-types}}
Before moving on to more interesting functions, we must mention the
concepts of \sphinxstyleemphasis{types} and \sphinxstyleemphasis{values} as they pertain to both \sphinxstyleemphasis{data} and
\sphinxstyleemphasis{functions}. Two types appear in the example of the \sphinxstyleemphasis{id} function. The
first, obvious, one is the type \sphinxstyleemphasis{int}. The \sphinxstyleemphasis{values} of this type are
\sphinxstyleemphasis{data} values, namely values representing integers. The second type,
which is less visible in the example, is the type of the the function,
\sphinxstyleemphasis{id}, itself. As the function takes an argument of type \sphinxstyleemphasis{int} and also
returns a value of type \sphinxstyleemphasis{int}, we say that the type of \sphinxstyleemphasis{id} is
\(int \rightarrow int\). You can pronounce this type as \sphinxstyleemphasis{int to
int}.


\section{Other function values of the same type}
\label{\detokenize{04-runnable-math:other-function-values-of-the-same-type}}
There are many (indeed an uncountable infinity of) functions that
convert integer values to other integer values. All such functions
have the same type, namely \(int \rightarrow int\), but they
constitute different function \sphinxstyleemphasis{values}. While the type of a function
is specified in the declaration of the function argument and return
types, a function \sphinxstyleemphasis{value} is defined by the expression comprising the
\sphinxstyleemphasis{body} of the function.

An example of a different function of the same type is what we will
call \sphinxstyleemphasis{inc}, short for \sphinxstyleemphasis{increment}. When applied to an integer value,
it reduces to (or \sphinxstyleemphasis{returns}) that value plus one. Mathematically, it
is defined as \(\forall x \in {\mathbb Z}, x + 1\). For example,
\sphinxstyleemphasis{inc(2)} reduces to \sphinxstyleemphasis{3}, and \sphinxstyleemphasis{inc(-2)}, to \sphinxstyleemphasis{-1}.

Here’s a Dafny functional program that implements this function. You
should be able to understand this program with ease. Once again, take
a moment to see the relationship between the abstract mathematical
definition and the concrete code. They are basically isomorphic. The
pure functional programmer is writing \sphinxstyleemphasis{runnable mathematics}.

\begin{sphinxVerbatim}[commandchars=\\\{\}]
function method inc (x: int): int \PYGZob{} x + 1 \PYGZcb{}
\end{sphinxVerbatim}

Another example of a function of the same type is, \sphinxstyleemphasis{square}, defined
as returing the square of its integer argument. Mathematically it is
the function, \(\forall x \in {\mathbb Z}, x + 1\). And here is
a Dafny implementation.

\begin{sphinxVerbatim}[commandchars=\\\{\}]
function method h (x: int): int \PYGZob{} x * x \PYGZcb{}
\end{sphinxVerbatim}

Evaluating expressions in which this function is applied to an
argument happens as previously described. To evaluate \sphinxstyleemphasis{square(4)}, for
example, you rewrite the body, \sphinxstyleemphasis{x * x}, replacing every \sphinxstyleemphasis{x} with a
\sphinxstyleemphasis{4}, yielding the expression \sphinxstyleemphasis{4 * 4}, then you evaluate that
expression and return the result, here \sphinxstyleemphasis{16}. Function evaluation is
done by substituting actual parameter values for all occurrences of
corresponding formal parameters in the body of a function, evaluating
the resulting expression, and returning that result.


\section{Recursive function definitions and implementations}
\label{\detokenize{04-runnable-math:recursive-function-definitions-and-implementations}}
Many mathematical functions are defined \sphinxstyleemphasis{recursively}. Consider the
familiar \sphinxstyleemphasis{factorial} function. An informal explanation of what the
function produces when applied to a natural number (a non-negative
integer), \sphinxstyleemphasis{n}, is the product of natural numbers from \sphinxstyleemphasis{1} to \sphinxstyleemphasis{n}.

That’s a perfectly understandable definition, but it’s not quite
precise (or even correct) enough for a mathematician. There are at
least two problems with this definition. First, it does not define the
value of the function \sphinxstyleemphasis{for all} natural numbers. In particular, it
does not say what the value of the function is for zero. Second, you
can’t just extend the definition by saying that it yields the product
of all the natural numbers from zero to \sphinxstyleemphasis{n}, because that is always
zero!

Rather, if the function is to be defined for an argument of zero, as
we require, then we had better define it to have the value one when
the argument is zero, to preserve the product of all the other numbers
larger than zero that we might have multiplied together to produce the
result. The trick is to write a mathematical definition of factorial
in two cases: one for the value zero, and one for any other number.
\begin{equation*}
\begin{split}factorial(n) := \forall n \in {\mathbb Z} \mid n >= 0, \begin{cases}
\text{if n=0}, & 1,\\ \text{otherwise}, & n *
factorial(n-1).\end{cases}\end{split}
\end{equation*}
To pronounce this mathematical definition in English, one would say
that for any integer, \sphinxstyleemphasis{n}, such that \sphinxstyleemphasis{n} is greater than or equal to
zero, \sphinxstyleemphasis{factorial(n)} is one if \sphinxstyleemphasis{n} is zero and is otherwise \sphinxstyleemphasis{n} times
\sphinxstyleemphasis{factorial(n-1)}.

Let’s analyze this definition. First, whereas in earlier examples we
left mathematical definitions anonymous, here we have given a name,
\sphinxstyleemphasis{factorial}, to the function, as part of its mathematical definition.
We have to do this because we need to refer to the function within its
own definition.  When a definition refers to the thing that is being
defined, we call the definition \sphinxstyleemphasis{recursive.}

Second, we have restricted the \sphinxstyleemphasis{domain} of the function, which is to
say the set of values for which it is defined, to the non-negative
integers only, the set known as the \sphinxstyleemphasis{natural numbers}. The function
simply isn’t defined for negative numbers.  Mathematicians usually use
the symbol, \({\mathbb N}\) for this set. We could have written
the definition a little more concisely using this notation, like this:
\begin{equation*}
\begin{split}factorial(n) := \forall n \in {\mathbb N}, \begin{cases}
\text{if n=0}, & 1,\\ \text{otherwise}, & n *
factorial(n-1).\end{cases}\end{split}
\end{equation*}
Here, then, is a Dafny implementation of the factorial function.

\begin{sphinxVerbatim}[commandchars=\\\{\}]
function method fact(n: int): int
   requires n \PYGZgt{}= 0 // for recursion to be well founded
\PYGZob{}
    if (n==0) then 1
    else n * fact(n\PYGZhy{}1)
\PYGZcb{}
\end{sphinxVerbatim}

This code exactly mirrors our first mathematical definition. The
restriction on the domain is expressed in the \sphinxstyleemphasis{requires} clause of the
program. This clause is not runnable code. It’s a specification: a
\sphinxstyleemphasis{predicate} (a proposition with a parameter) that must hold for the
program to be used. Dafny will insist that this function only ever be
applied to values of \sphinxstyleemphasis{n} that have the \sphinxstyleemphasis{property} of being \(>=
0\). A predicate that must be true for a program to be run is called a
\sphinxstyleemphasis{pre-condition}.

To see how the recursion works, consider the application of
\sphinxstyleemphasis{factorial} to the natural number, \sphinxstyleemphasis{3}. We know that the answer should
be \sphinxstyleemphasis{6. The evaluation of the expression, *factorial(3)}, works as for
any function application expression: first you subsitute the value of
the argument(s) for each occurrence of the formal parameters in the
body of the function; then you evaluate the resulting expression
(recursively!) and return the result. For \sphinxstyleemphasis{factorial(3)}, this process
leads through a sequence of intermediate expressions as follows (leaving
out a few details that should be easy to infer):
\begin{align*}\!\begin{aligned}
factorial\ (3) & \text{ ; a function application expression}\\
if\ (3 == 0)\ then\ 1\ else\ (3 * factorial\ (3-1)) & \text{ ; expand body with  parameter/argument substitution}\\
if\ (3 == 0)\ then\ 1\ else\ (3 * factorial\ (2))  & \text{ ; evaluate $(3-1)$}\\
if\ false\ then\ 1\ else\ (3 * factorial\ (2)) & \text{ ; evaluate $(3==0)$ }\\
(3 * factorial\ (2)) & \text{ ; evaluate $ifThenElse$ }\\
(3 * (if\ (2==0)\ then\ 1\ else\ (2 * factorial\ (1))) & \text{ ; etc }\\
(3 * (2 * factorial\ (1))\\
(3 * (2 * (if\ (1==0)\ then\ 1\ else\ (1 * factorial\ (0)))))\\
(3 * (2 * (1 * factorial\ (0))))\\
(3 * (2 * (1 * (if\ (0==0)\ then\ 1\ else\ (0 * factorial\ (-1))))))\\
(3 * (2 * (1 * (if\ true\ then\ 1\ else\ (0 * factorial\ (-1))))))\\
(3 * (2 * (1 * 1)))\\
(3 * (2 * 1))\\
(3 * 2)\\
6\\
\end{aligned}\end{align*}
The evaluation process continues until the function application expression
is reduced to a data value. That’s the answer!

It’s important to understand how recursive function application
expressions are evaluated. Study this example with care. Once you’re
sure you see what’s going on, go back and look at the mathematical
definition, and convince yourself that you can understand it \sphinxstyleemphasis{without}
having to think about \sphinxstyleemphasis{unrolling} of the recursion as we just did.

Finally we note that the the precondition is essential. If it were not
there in the mathematical definition, the definition would not be what
mathematicians call \sphinxstyleemphasis{well founded}: the recursive definition might
never stop looping back on itself. Just think about what would happen
if you could apply the function to \sphinxstyleemphasis{-1}. The definition would involve
the function applied to \sphinxstyleemphasis{-2}. And the definition of that would involve
the function applied to \sphinxstyleemphasis{-3}. You can see that there will be an
infinite regress.

Similarly, if Dafny would allow the function to be applied to \sphinxstyleemphasis{any}
value of type \sphinxstyleemphasis{int}, it would be possible, in particular, to apply the
function to negative values, and that would be bad!  Evaluating the
expression, \sphinxstyleemphasis{factorial(-1)} would involve the recursive evaluation of
the expression, \sphinxstyleemphasis{factorial(-2)}, and you can see that the evaluation
process would never end. The program would go into an “infinite loop”
(technically an unbounded recursion). By doing so, the program would
also violate the fundamental promise made by its type: that for \sphinxstyleemphasis{any}
integer-valued argument, an integer result will be produced. That can
not happen if the evaluation process never returns a result. We see
the precondition in the code, implementing the domain restriction in
the mathematical definition, is indispensible. It makes the definition
sound and it makes the code correct!


\section{Dafny is a Program Verifier}
\label{\detokenize{04-runnable-math:dafny-is-a-program-verifier}}
Restricting the domain of factorial to non-negative integers is
critical. Combining the non-negative property of ever value to which
the function is applied with the fact that every recursive application
is to a smaller value of \sphinxstyleemphasis{n}, allows us to conclude that no \sphinxstyleemphasis{infinite
decreasing chains} are possible. Any application of the function to a
non-negative integer \sphinxstyleemphasis{n} will terminate after exactly \sphinxstyleemphasis{n} recursive
calls to the function. Every non-negative integer, \sphinxstyleemphasis{n} is finite. So
every call to the function will terminate.

Termination is a critical \sphinxstyleemphasis{property} of programs. The proposition that
our factorial program with the precondition in place always terminates
is true as we’ve argued. Without the precondition, the proposition is
false.

Underneath Dafny’s “hood,” it has a system for proving propositions
about (i.e., properties of) programs. Here we see that It generates a
propostion that each recursive function terminates; and it requires a
proof that each such proposition is true.

With the precondition in place, there not only is a proof, but Dafny
can find it on its own. If you remove the precondition, Dafny won’t be
able to find a proof, because, as we just saw, there isn’t one: the
proposition that evaluation of the function always terminates is not
true. In this case, because it can’t prove termination, Dafny will
issue an error stating, in effect, that there is the possibility that
the program will infinitely loop. Try is in Dafny.  You will see.

In some cases there will be proofs of important propositions that
Dafny nevertheless can’t find it on its own. In such cases, you may
have to help it by giving it some additional propositions that it
can verify and that help point it in the right direction. We’ll see
more of this later.

The Dafny language and verification system is powerful mechansim for
finding subtle bugs in code, but it require a knowledge of more than
just programming. It requires an understanding of specification, and
of the languages of logic and proofs in which specifications of code
are expressed and verified.


\chapter{Integrating Formal Specification with Imperative Programming}
\label{\detokenize{05-putting-it-together::doc}}\label{\detokenize{05-putting-it-together:integrating-formal-specification-with-imperative-programming}}
An important approach to solving such problems is to enable the
integration of \sphinxstyleemphasis{formal specifications} with imperative programming
code along with mechansims (based on \sphinxstyleemphasis{logical proof} technology) for
checking the consistency of code with specifications. Specifications
are given not as comments but as expressions in the language of logic
right along with the code, and checkers attempt to verify that code
satisfies its corresponding \sphinxstyleemphasis{specs}.

Dafny is a cutting-edge software language and tooset developed at
Microsoft Research—one of the top computer science research labs in
the world—that provides such a capability. We will explore Dafny and
the ideas underlying it in the first part of this course, both to give
a sense of the current state of the art in program verification and,
most importantly, to explain why it’s vital for a computer scientist
today to have a substantial understanding of logic and proofs along
with the ability to \sphinxstyleemphasis{code}.

Tools such as TLA+, Dafny, and others of this variety give us a way
both to express formal specifications and imperative code in a unified
way (albeit in different sub-languages), and to have some automated
checking done in an \sphinxstyleemphasis{attempt} to verify that code satisfies its spec.

We say \sphinxstyleemphasis{attempt} here, because in general verifying the consistency of
code and a specification is a literally unsolvable problem. In cases
that arise in practice, much can often be done. It’s not always easy,
but if one requires ultra-high assurance of the consistency of code
and specification, then there is no choice but to employ the kinds of
\sphinxstyleemphasis{formal methods} introduced here.

To understand how to use such state-of-the-art software development
tools and methods, one must understand not only the language of code,
but also the languages of mathematical logic, including set and type
theory. One must also understand precisely what it means to \sphinxstyleemphasis{prove}
that a program satisfies its specification; for generating proofs is
exactly what tools like Dafny do \sphinxstyleemphasis{under the hood}.

A well educated computer scientist and a professionally trained
software developer must understand logic and proofs as well as coding,
and how they work together to help build \sphinxstyleemphasis{trustworthy} systems. Herein
lies the deep relevance of logic and proofs, which might otherwise
seem like little more than abstract nonsense and a distraction from
the task of learning how to program.


\section{To integrate}
\label{\detokenize{05-putting-it-together:to-integrate}}

\section{Fitting it All Together}
\label{\detokenize{05-putting-it-together:fitting-it-all-together}}
So as we go forward, here’s what we’ll see. Ultimately, for purposes
of efficiency and interactivity (I/O), we will write imperative code
to implement software systems. That said, we can often use functional
code to implement subroutines that perform computations that do not
require mutable storage or I/O. We will \sphinxstyleemphasis{also} use pure functional
programs as parts of \sphinxstyleemphasis{specifications}.

For example, we might specify that an \sphinxstyleemphasis{imperative} implementation of
the factorial function must take any natural number \sphinxstyleemphasis{n} as an argument
and return the value of \sphinxstyleemphasis{fact(n),} our \sphinxstyleemphasis{functional} program for the
factorial function. The logical specification of the imperative
program will be an \sphinxstyleemphasis{implication} stating that if a proper argument is
presented, a correct result \sphinxstyleemphasis{as defined by a functional program} will
be produced.

We can thus use pure functional programs both for computation \sphinxstyleemphasis{when
appropriate}, yielding certain benefits in terms of understandability
and safety, and as elements in logical specifications of imperative
code. In Dafny, a pure functional program that is intended only for
use in specifications is declared as a \sphinxstyleemphasis{function}. A pure functional
program intended to be called from imperative code is declared as a
\sphinxstyleemphasis{function method.} Imperative programs are simply declared as methods.

Here’s a complete example: an imperative program for computing the
factorial function with a specification that first requires \sphinxstyleemphasis{n\textgreater{}0}
and that then requires that the result be \sphinxstyleemphasis{fact(n)} as defined by
our functional program.

\begin{sphinxVerbatim}[commandchars=\\\{\}]
method factorial(n: nat) returns (f: nat)
\PYGZob{}
    if (n == 0)
    \PYGZob{}
        return 1;
    \PYGZcb{}
    var t: nat := n;
    var a: nat := 1;
    while (t !=  0)
    \PYGZob{}
        a := a * t;
        t := t \PYGZhy{} 1;
    \PYGZcb{}
    f := a;
\PYGZcb{}
\end{sphinxVerbatim}

\begin{sphinxVerbatim}[commandchars=\\\{\}]
method factorial(n: int) returns (f: int)
  requires n\PYGZgt{}= 0
  ensures f == fact(n)
\PYGZob{}
  if (n == 0)
  \PYGZob{}
      return 1;
  \PYGZcb{}
  var t := n;
  var a := 1;
  while (t !=  0)
  \PYGZob{}
      a := a * t;
      t := t \PYGZhy{} 1;
  \PYGZcb{}
  return a;
\PYGZcb{}
\end{sphinxVerbatim}

Unfortunately Dafny reports that it cannot guarantee—formally prove
to itself—that the \sphinxstyleemphasis{postcondition} (that the result be right) will
necessarily hold. Generating proofs is hard, not only for people but
also for machines. In general it’s impossibly hard, so the best that a
machine can do in practice is to try its best. If Dafny fails, as it
does in this case, what comes next is that the developer has to give
it some help. This is done by adding some additional logic to the code
to help Dafny see its way to proving that the code satisfies the spec.

We’ll see some of what’s involved as we go forward in this class. We
will also eventually dive in to understand what proofs even are, and
why in general they are hard to construct. Lucky for mathematicians!
If this weren’t true, they’d all be out of jobs. Before we go there,
though, let’s have some fun and learn how to write imperative code in
Dafny.


\chapter{Indices and tables}
\label{\detokenize{index:indices-and-tables}}\begin{itemize}
\item {} 
\DUrole{xref,std,std-ref}{genindex}

\item {} 
\DUrole{xref,std,std-ref}{modindex}

\item {} 
\DUrole{xref,std,std-ref}{search}

\end{itemize}



\renewcommand{\indexname}{Index}
\printindex
\end{document}