%% Generated by Sphinx.
\def\sphinxdocclass{report}
\documentclass[letterpaper,10pt,english]{sphinxmanual}
\ifdefined\pdfpxdimen
   \let\sphinxpxdimen\pdfpxdimen\else\newdimen\sphinxpxdimen
\fi \sphinxpxdimen=.75bp\relax

\usepackage[utf8]{inputenc}
\ifdefined\DeclareUnicodeCharacter
 \ifdefined\DeclareUnicodeCharacterAsOptional
  \DeclareUnicodeCharacter{"00A0}{\nobreakspace}
  \DeclareUnicodeCharacter{"2500}{\sphinxunichar{2500}}
  \DeclareUnicodeCharacter{"2502}{\sphinxunichar{2502}}
  \DeclareUnicodeCharacter{"2514}{\sphinxunichar{2514}}
  \DeclareUnicodeCharacter{"251C}{\sphinxunichar{251C}}
  \DeclareUnicodeCharacter{"2572}{\textbackslash}
 \else
  \DeclareUnicodeCharacter{00A0}{\nobreakspace}
  \DeclareUnicodeCharacter{2500}{\sphinxunichar{2500}}
  \DeclareUnicodeCharacter{2502}{\sphinxunichar{2502}}
  \DeclareUnicodeCharacter{2514}{\sphinxunichar{2514}}
  \DeclareUnicodeCharacter{251C}{\sphinxunichar{251C}}
  \DeclareUnicodeCharacter{2572}{\textbackslash}
 \fi
\fi
\usepackage{cmap}
\usepackage[T1]{fontenc}
\usepackage{amsmath,amssymb,amstext}
\usepackage{babel}
\usepackage{times}
\usepackage[Bjarne]{fncychap}
\usepackage[dontkeepoldnames]{sphinx}

\usepackage{geometry}

% Include hyperref last.
\usepackage{hyperref}
% Fix anchor placement for figures with captions.
\usepackage{hypcap}% it must be loaded after hyperref.
% Set up styles of URL: it should be placed after hyperref.
\urlstyle{same}
\addto\captionsenglish{\renewcommand{\contentsname}{Contents:}}

\addto\captionsenglish{\renewcommand{\figurename}{Fig.}}
\addto\captionsenglish{\renewcommand{\tablename}{Table}}
\addto\captionsenglish{\renewcommand{\literalblockname}{Listing}}

\addto\captionsenglish{\renewcommand{\literalblockcontinuedname}{continued from previous page}}
\addto\captionsenglish{\renewcommand{\literalblockcontinuesname}{continues on next page}}

\addto\extrasenglish{\def\pageautorefname{page}}

\setcounter{tocdepth}{1}


    \usepackage{amsmath}
    \usepackage{cases}
    \usepackage{turnstile}
    \usepackage{bussproofs}
    \usepackage{prftree}
    

\title{Mathematical Logic in Software Development Documentation}
\date{Feb 25, 2018}
\release{1}
\author{Kevin Sullivan}
\newcommand{\sphinxlogo}{\vbox{}}
\renewcommand{\releasename}{Release}
\makeindex

\begin{document}

\maketitle
\sphinxtableofcontents
\phantomsection\label{\detokenize{index::doc}}



\chapter{1. Requirement, Specifications, and Implementations}
\label{\detokenize{01-reqs-specs-impls:requirement-specifications-and-implementations}}\label{\detokenize{01-reqs-specs-impls::doc}}\label{\detokenize{01-reqs-specs-impls:welcome-to-mathematical-logic-in-software-development}}
Software is an increasingly critical component of major societal
systems, from rockets to power grids to healthcare, etc. Failures are
not always bugs in implementation code. The most critical problems
today are not in implementations but in requirements and
specifications.
\begin{itemize}
\item {} 
\sphinxstylestrong{Requirements:} Statements of the effects that a system is meant to have in a given domain

\item {} 
\sphinxstylestrong{Specification:} Statements of the behavior required of a machine to produce such effects

\item {} 
\sphinxstylestrong{Implementation:} The definition (usually in code) of how a machine produces the specified behavior

\end{itemize}

Avoiding software-caused system failures requires not only a solid
understanding of requirements, specifications, and implementations,
but also great care in both the \sphinxstyleemphasis{validation} of requirements and of
specifications, and \sphinxstyleemphasis{verification} of code against specifications.
\begin{itemize}
\item {} 
\sphinxstylestrong{Validation:} \sphinxstyleemphasis{Are we building the right system?} is the specification right; are the requirements right?

\item {} 
\sphinxstylestrong{Verification:} \sphinxstyleemphasis{Are we building the system right?} Does the implementation behave as its specification requires?

\end{itemize}

You know that the language of implementation is code. What is the
language of specification and of requirements?

One possible answer is \sphinxstyleemphasis{natural language}. Requirements and
specifications can be written in natural languages such as English or
Mandarin. The problem is that natural language is subject to
ambiguity, incompleteness, and inconsistency. This makes it a risky
medium for communicating the precise behaviors required of complex
software artifacts.

The alternative to natural language that we will explore in this class
is the use of mathematical logic, in particular what we call propositional
logic, predicate logic, set theory, and the related field of type theory.

Propositional logic is a language of simple propositions. Propositions
are assertions that might or might not be judged to be true. For
example, \sphinxstyleemphasis{Tennys (the person) plays tennis} is actually a true
proposition (if we interpret \sphinxstyleemphasis{Tennys} to be the person who just played
in the French Open).  So is \sphinxstyleemphasis{Tennys is from Tennessee}. And because
these two propositions are true, so is the \sphinxstyleemphasis{compound} proposition (a
proposition built up from smaller propositions) that Tennys is from
Tennessee \sphinxstylestrong{and} Tennys plans tennis.

Sometimes we want to talk about whether different entities satisfy
give propositions. For this, we introduce propositions with parameters,
which we will call \sphinxstyleemphasis{properties}. If we take \sphinxstyleemphasis{Tennys} out of \sphinxstyleemphasis{Tennys
plays tennis} and replace his name by a variable, \sphinxstyleemphasis{P}, that can take
on the identify of any person, then we end up with a parameterized
proposition, \sphinxstyleemphasis{P plays tennis}. Substituting the name of any particular
person for \sphinxstyleemphasis{P} then gives us a proposition \sphinxstyleemphasis{about that person} that we
can judge to be true or false. A parameterized proposition thus gives
rise to a whole family of propositions, one for each possible value of
\sphinxstyleemphasis{P}.

Sometimes we write parameterized propositions so that they look like
functions, like this: \sphinxstyleemphasis{PlaysTennis(P)}. \sphinxstyleemphasis{PlaysTennis(Tennys)} is thus
the proposition, \sphinxstyleemphasis{Tennys plays Tennis} while \sphinxstyleemphasis{PlaysTennis(Kevin)} is
the proposition \sphinxstyleemphasis{Kevin plays Tennis}. For each possible person name,
\sphinxstyleemphasis{P}, there is a corresponding proposition, \sphinxstyleemphasis{PlaysTennis(P)}.

Some such propositions might be true. For instance,
\sphinxstyleemphasis{PlaysTennis(Tennys)} is true in our example. Others might be false. A
parameterized proposition thus encodes a \sphinxstyleemphasis{property} that some things
(here people) have and that others don’t have (here, the property of
\sphinxstyleemphasis{being a tennis player}).

A property, also sometimes called a \sphinxstyleemphasis{predicate}, thus also serves to
identify a \sphinxstyleemphasis{subset} of elements in a given \sphinxstyleemphasis{domain of discourse}. Here
the domain of discourse is the of all people. The subset of people who
actually do \sphinxstyleemphasis{play tennis} is exactly the set of people, P, for whom
\sphinxstyleemphasis{PlaysTennis(P)} is true.

We note briefly, here, that, like functions, propositions can have
multiple parameters. For example, we can generalize from \sphinxstyleemphasis{Tennys plays
Tennis **and*} Tennys is from Tennessee* to \sphinxstyleemphasis{P plays tennis and P is
from L,} where P ranges over people and L ranges over locations. We
call a proposition with two or more parameters a \sphinxstyleemphasis{relation}. A
relation picks out \sphinxstyleemphasis{combinations} of elements for which corresponding
properties are true. So, for example, the \sphinxstyleemphasis{pair} (Tennys, Tennessee)
is in the relation (set of \sphinxstyleemphasis{P-L} pairs) picked out by this
parameterized proposition. On the other hand, the pair, (Kevin,
Tennessee), is not, because Kevin is actually from New Hampshire, so
the proposition \sphinxstyleemphasis{Kevin plays tennis **and*} Kevin is from Tennessee*
is not true. More on relations later!


\chapter{2. Logical Specifications, Imperative Implementations}
\label{\detokenize{02-logic-and-code::doc}}\label{\detokenize{02-logic-and-code:logical-specifications-imperative-implementations}}
We’ve discussed requirements, specifications, and implementations as
distinct artifacts that serve distinct purposes. For good reasons,
these artifacts are usually written in different languages. Software
implementations are usually written in programming languages, and, in
particular, are usually written in \sphinxstyleemphasis{imperative} programming languages.
Requirements and specifications, on the other hand, are written either
in natural language, e.g., English, or in the language of mathematical
logic.

This unit discusses these different kinds of languages, why they are
used for different purposes, the advantages and disadvantages of each,
and why modern software development requires fluency in and tools for
handling artifacts written in multiple such languages. In particular,
the educated computer scientist and the capable software developer
must be fluent in the language of mathematical logic.


\section{Imperative Languages for Implementations}
\label{\detokenize{02-logic-and-code:imperative-languages-for-implementations}}
The language of implementations is code, usually written in what we
call an \sphinxstyleemphasis{imperative} programming language. Examples of such languages
include Python, Java, C++, and Javascript.

The essential property of an imperative language is that it is
\sphinxstyleemphasis{procedural}. Programs in these languages describe step-by-step
\sphinxstyleemphasis{procedures}, in the form of sequences of \sphinxstyleemphasis{commands}, for solving
given problem instances. Commands in turn operate (1) by reading,
computing with, and updating values stored in a \sphinxstyleemphasis{memory}, and (2) by
interacting with the world outside of the computer by executing input
and output (I/O) commands.

Input (or \sphinxstyleemphasis{read}) commands obtain data from \sphinxstyleemphasis{sensors.} Sensors include
mundane devices such as computer mice, trackpads, and keyboards. They
also include sensors for temperature, magnetism, vibration, chemicals,
biological agents, radiation, and face and license plate recognition,
and much more. Sensors convert physical phenomena in the world into
digital data that programs can manipulate. Computer programs can thus be
made to \sphinxstyleemphasis{compute about reality beyond the computing machine}.

Output (or \sphinxstyleemphasis{write}) commands turn data back into physical phenomena in
the world. The cruise control computer in a car is a good example.  It
periodically senses both the actual speed of the car and the desired
speed set by the driver. It then computes the difference and finally
finally it outputs data representing that difference to an \sphinxstyleemphasis{actuator}
that changes the physical accelerator and transmission settings of the
car to speed it up or slow it down. Computer programs can thus also be
made to \sphinxstyleemphasis{manipulate reality beyond the computing machine}.

A special part of the world beyond of the (core of a) computer is its
\sphinxstyleemphasis{memory}. A memory is to a computer like a diary or a notebook is to a
person: a place to \sphinxstyleemphasis{write} information at one point in time that can
then be \sphinxstyleemphasis{read} back later on. Computers use special actuators to write
data to memory, and special sensors to read it back from memory when
it is needed later on. Memory devices include \sphinxstyleemphasis{random access memory}
(RAM), \sphinxstyleemphasis{flash memory}, \sphinxstyleemphasis{hard drives}, \sphinxstyleemphasis{magnetic tapes}, \sphinxstyleemphasis{compact} and
\sphinxstyleemphasis{bluray} disks, cloud-based data storage systems such as Amazon’s \sphinxstyleemphasis{S3}
and \sphinxstyleemphasis{Glacier} services, and so forth.

Sequential progams describe sequences of actions involving reading of
data from sensors (including from memory devices), computing with this
data, and writing resulting data out to actuators (to memory devices,
display screens, and physical systems controllers). Consider the
simple assignment command, \sphinxstyleemphasis{x := x + 1}. It tells the computer to
first \sphinxstyleemphasis{read} in the value stored in the part of memory designated by
the variable, \sphinxstyleemphasis{x, to add one to that value, and finally to *write} the
result back out to the same location in memory. It’s as if the person
read a number from a notebook, computed a new number, and then erased
the original number and replaced it with the new number. The concept
of an updateable memory is at the very heart of the imperative model
of computation.


\section{Declarative Languages for Specifications}
\label{\detokenize{02-logic-and-code:declarative-languages-for-specifications}}
The language of formal requirements and specifications, on the other
hand, is not imperative code but \sphinxstyleemphasis{declarative} logic.  Expressions in
such logic will state \sphinxstyleemphasis{what} properties or relationships must hold in
given situation without providing a procedures that describes \sphinxstyleemphasis{how}
such results are to be obtained.

To make the difference between procedural and declarative styles of
description clear, consider the problem of computing the positive
square root of any given non-negative number, \sphinxstyleemphasis{x}. We can \sphinxstyleemphasis{specify}
the result we seek in a clear and precise logical style by saying
that, for any given non-negative number \sphinxstyleemphasis{x}, we require a value, \sphinxstyleemphasis{y},
such that \(y^2 = x\). Such a \sphinxstyleemphasis{y}, squared, gives \sphinxstyleemphasis{x}, and this
makes \sphinxstyleemphasis{y} a square root.

We would write this mathematically as \(\forall x \in {\mathbb R}
\mid x >= 0, y \in {\mathbb R} | y >= 0 \land y^2 = x\). In English,
we’d pronounce this expression as, “for any value, \sphinxstyleemphasis{x}, in the real
numbers, where \sphinxstyleemphasis{x} is greater than or equal to zero, the result is a
value, \sphinxstyleemphasis{y}, also in the real numbers, where \sphinxstyleemphasis{y} is greater than or
equal to zero and \sphinxstyleemphasis{y} squared is equal to \sphinxstyleemphasis{x}.” (The word, \sphinxstyleemphasis{where},
here is also often pronounced as \sphinxstyleemphasis{such that}. Repeat it to yourself
both ways until it feels natural to translate the math into spoken
English.)

Let’s look at this expression with care. First, the symbol,
\(\forall\), is read as \sphinxstyleemphasis{for all} or \sphinxstyleemphasis{for any}. Second, the symbol
\({\mathbb R}\), is used in mathematical writing to denote the set
of the \sphinxstyleemphasis{real numbers}, which includes the \sphinxstyleemphasis{integers} (whole numbers,
such as \sphinxstyleemphasis{-1}, \sphinxstyleemphasis{0}, and \sphinxstyleemphasis{2}), the rational numbers (such as \(2/3\)
and \sphinxstyleemphasis{1.5}), and the irrational numbers (such as \sphinxstyleemphasis{pi} and \sphinxstyleemphasis{e}). The
symbol, \(\in\), pronounced as \sphinxstyleemphasis{in}, represents membership of a
value, here \sphinxstyleemphasis{x}, in a given set. The expression, \(\forall x \in
{\mathbb R}\) thus means “for any value, \sphinxstyleemphasis{x}, in the real numbers,” or
just “for any real number, \sphinxstyleemphasis{x}”.

The vertical bar followed by the statement of the property, \sphinxstyleemphasis{x \textgreater{}= 0},
restricts the value being considered to one that satisfies the stated
property. Here the value of \sphinxstyleemphasis{x} is restricted to being greater than or
equal to zero. The formula including this constraint can thus be read
as “for any non-negative real number, \sphinxstyleemphasis{x}.” The set of non-negative
real numbers is thus selected as the \sphinxstyleemphasis{domain} of the function that we
are specifying.

The comma is our formula is a major break-point. It separates the
specification of the \sphinxstyleemphasis{domain} of the function from a formula, after
the comma, that specifies what value, if any, is associated with each
value in the domain.  You can think of the formula after the comma as
the \sphinxstyleemphasis{body} of the function. Here it says, assuming that \sphinxstyleemphasis{x} is any
non-negative real numner, that the associated value, sometimes called
the \sphinxstyleemphasis{image} of \sphinxstyleemphasis{x} under the function, is a value, \sphinxstyleemphasis{y}, also in the
real numbers (the \sphinxstyleemphasis{co-domain} of the function), such that \sphinxstyleemphasis{y} is both
greater than or equal to zero equal \sphinxstyleemphasis{and} \(y^2 = x\). The symbol,
\(\land\) is the logical symbol for \sphinxstyleemphasis{conjunction}, which is the
operation that composes two smaller propositions or properties into a
larger one that is true or satisfied if and only if both constituent
propositions or properties are. The formula to the right of the comma
thus picks out exactly the positive (or more accurate a non-negative)
square root of \sphinxstyleemphasis{x}.

We thus have a precise specification of the positive square root
function for non-negative real numbers. It is defined for every value
in the domain insofar as every non-negative real number has a positive
square root. It is also a \sphinxstyleemphasis{function} in that there is \sphinxstyleemphasis{at most one}
value for any given argument. If we had left out the non-negativity
\sphinxstyleemphasis{constraint} on \sphinxstyleemphasis{y} then for every \sphinxstyleemphasis{x} (except \sphinxstyleemphasis{0}) there would be
\sphinxstyleemphasis{two} square roots, one positive and one negative. We would then no
longer have a \sphinxstyleemphasis{function}, but rather a \sphinxstyleemphasis{relation}. A function must be
\sphinxstyleemphasis{single-valued}, with at most one “result” for any given “argument”.

We now have a \sphinxstyleemphasis{declarative specification} of the desired relationship
between \sphinxstyleemphasis{x} and \sphinxstyleemphasis{y}. The definition is clear (once you understand the
notation), it’s concise, it’s precise. Unfortunately, it isn’t what we
call \sphinxstyleemphasis{effective}. It doesn’t give us a way to actually \sphinxstyleemphasis{compute} the
value of the square root of any \sphinxstyleemphasis{x}. You can’t run a specification in
the language of mathematical logic (at least not in a practical way).


\section{Refining Declarative Specifications into Imperative Implementations}
\label{\detokenize{02-logic-and-code:refining-declarative-specifications-into-imperative-implementations}}
The solution is to \sphinxstyleemphasis{refine} our declarative specification, written in
the language of mathematical logic, into a computer program, written
in an imperative language: one that computes \sphinxstyleemphasis{exactly} the function we
have specified. To refine means to add detail while also preserving
the essential properties of the original. The details to be added are
the procedural steps required to compute the function. The essence to
be preserved is the value of the function at each point in its domain.

In short, we need a step-by-step procedure, in an imperative language,
that, when \sphinxstyleemphasis{evaluated with a given actual parameter value}, computes
exactly the specified value. Here’s a program that \sphinxstyleemphasis{almost} does the
trick. Written in the imperative language, Python, it uses Newton’s
method to compute \sphinxstyleemphasis{floating point} approximations of positive square
roots of given non-negative \sphinxstyleemphasis{floating point} arguments.

\begin{sphinxVerbatim}[commandchars=\\\{\}]
\PYG{k}{def} \PYG{n+nf}{sqrt}\PYG{p}{(}\PYG{n}{x}\PYG{p}{)}\PYG{p}{:}
    \PYG{l+s+sd}{\PYGZdq{}\PYGZdq{}\PYGZdq{}for x\PYGZgt{}=0, return non\PYGZhy{}negative y such that y\PYGZca{}2 = x\PYGZdq{}\PYGZdq{}\PYGZdq{}}
    \PYG{n}{estimate} \PYG{o}{=} \PYG{n}{x}\PYG{o}{/}\PYG{l+m+mf}{2.0}
    \PYG{k}{while} \PYG{n+nb+bp}{True}\PYG{p}{:}
        \PYG{n}{newestimate} \PYG{o}{=} \PYG{p}{(}\PYG{p}{(}\PYG{n}{estimate}\PYG{o}{+}\PYG{p}{(}\PYG{n}{x}\PYG{o}{/}\PYG{n}{estimate}\PYG{p}{)}\PYG{p}{)}\PYG{o}{/}\PYG{l+m+mf}{2.0}\PYG{p}{)}
        \PYG{k}{if} \PYG{n}{newestimate} \PYG{o}{==} \PYG{n}{estimate}\PYG{p}{:}
            \PYG{k}{break}
        \PYG{n}{estimate} \PYG{o}{=} \PYG{n}{newestimate}
    \PYG{k}{return} \PYG{n}{estimate}
\end{sphinxVerbatim}

This procedure initializes and then repeatedly updates the values
stored at two locations in memory, referred to by the two variables,
\sphinxstyleemphasis{estimate} and \sphinxstyleemphasis{newestimate}. It repeats the update process until the
process \sphinxstyleemphasis{converges} on the answer, which occurs when the values of the
two variables become equal. The answer is then returned to the caller
of this procedure.

Note that, following good programming style, we included an English
rendering of the specification as a document string in the second line
of the program.  There are however several problems using English or
other natural language comments to document specifications. First,
natural language is prone to ambiguity, inconsistency, imprecision,
and incompleteness. Second, because the document string is just a
comment, there’s no way for the compiler to check consistency between
the code and this specification. Third, in practice, code evolves (is
changed over time), and developers often forget, or neglect, to update
comments, so, even if an implementation is initially consistent with a
such a comment, inconsistencies can and often do develop over time.

In this case there is, in fact, a real, potentially catastrophic,
mathematical inconsistency between the specification and what the
program computes. The problem is that in Python, as in many everyday
programming languages, so-called \sphinxstyleemphasis{real} numbers are not exactly the
same as the real (\sphinxstyleemphasis{mathematical}) reals!

You can easily see what the problem is by using our procedure to
compute the square root of 2.0 and by then multiplying that number by
itself. The result of the computation is the number \sphinxstyleemphasis{1.41421356237},
which we already know has to be wrong to some degree, as the square
root of two is an \sphinxstyleemphasis{irrational} number that cannot be represented by
any non-terminating, non-repeating decimal. Indeed, if we multiply
this number by itself, we get the number, \sphinxstyleemphasis{1.99999999999}. We end up
in a situation in which \sphinxstyleemphasis{sqrt(2.0) * sqrt(2.0)} isn’t equal to 2.0!

The problem is that in Python, as in most industrial programming
languages, \sphinxstyleemphasis{so-called} real numbers (often called \sphinxstyleemphasis{floating point}
numbers) are represented in just 64 binary digits, and that permits
only a finite number of digits after the decimal to be represented.
And additional \sphinxstyleemphasis{low-order} bits are simply dropped, leading to what
we call \sphinxstyleemphasis{floating-point roundoff errors.} That’s what we’re seeing
here.

In fact, there are problems not only with irrational numbers but with
rational numbers with repeating decimal expansions when represented in
the binary notation of the IEEE-754 (2008) standard for floating point
arithmetic. Try adding \sphinxstyleemphasis{1/10} to itself \sphinxstyleemphasis{10} times in Python. You will
be surprised by the result. \sphinxstyleemphasis{1/10} is rational but its decimal form is
repeating in base-2 arithmetic, so there’s no way to represent \sphinxstyleemphasis{1/10}
precisely as a floating point number in Python, Java, or in many other
such languages.

There are two possible solutions to this problem. First, we could
change the specification to require only that \sphinxstyleemphasis{y} squared be very
close to \sphinxstyleemphasis{x} (within some specified margin of error). The we could
show that the code satisfies this approximate definition of square
root. An alternative would be to restrict our programming language to
represent real numbers as rational numbers, use arbitrarily large
integer values for numerators and denominators, and avoid defining any
functions that produce irrational values as results. We’d represent
\sphinxstyleemphasis{1/10} not as a 64-bit floating point number, for example, but simply
as the pair of integers \sphinxstyleemphasis{(1,10)}.

This is the solution that Dafny uses.  So-called real numbers in Dafny
behave not like \sphinxstyleemphasis{finite-precision floating point numbers that are only
approximate} in general, but like the \sphinxstyleemphasis{mathematical} real numbers they
represent. The limitation is that not all reals can be represented (as
values of the \sphinxstyleemphasis{real} type in Dafny. In particular, irrational numbers
cannot be represented exactly as real numbers. (Of course they can’t
be represented exactly by IEEE-754 floating point numbers, either.) If
you want to learn (a lot) more about floating point, or so-called
\sphinxstyleemphasis{real}, numbers in most programming languages, read the paper by David
Goldberg entitled, \sphinxstyleemphasis{What Every Computer Scientist Should Know About
Floating-Point Arithmetic.} It was published in the March, 1991 issue
of Computing Surveys. You can find it online.


\section{Why Not a Single Language for Programming and Specification?}
\label{\detokenize{02-logic-and-code:why-not-a-single-language-for-programming-and-specification}}
The dichotomy between specification logic and implementation code
raises an important question? Why not just design a single language
that’s good for both?

The answer is that there are fundamental tradeoffs in language design.
One of the most important is a tradeoff between \sphinxstyleemphasis{expressiveness}, on
one hand, and \sphinxstyleemphasis{efficient execution}, on the other.

What we see in our square root example is that mathematical logic is
highly \sphinxstyleemphasis{expressive}. Logic language can be used so say clearly \sphinxstyleemphasis{what}
we want. On the other hand, it’s hard using logic to say \sphinxstyleemphasis{how} to get
it. In practice, mathematical logic is clear but can’t be \sphinxstyleemphasis{run} with
the efficiency required in practice.

On the other hand, imperative code states \sphinxstyleemphasis{how} a computation is to be
carried out, but generally doesn’t make clear \sphinxstyleemphasis{what} it computes. One
would be hard-pressed, based on a quick look at the Python code above,
for example, to explain \sphinxstyleemphasis{what} it does (but for the comment, which is
really not part of the code).

We end up having to express \sphinxstyleemphasis{what} we want and \sphinxstyleemphasis{how} to get it in two
different languages. This situation creates a difficult new problem:
to verify that a program written in an imperative language satisfies,
or \sphinxstyleemphasis{refines}, a specification written in a declarative language.  How
do we know, \sphinxstyleemphasis{for sure}, that a program computes exactly the function
specified in mathematical logic?

This is the problem of program \sphinxstyleemphasis{verification}. We can \sphinxstyleemphasis{test} a program
to see if it produces the specified outputs for \sphinxstyleemphasis{some} elements of the
input domain, but in general it’s infeasible to test \sphinxstyleemphasis{all} inputs. So
how can we know that we have \sphinxstyleemphasis{built a program} right, where right is
defined precisely by a formal (mathematical logic) specification) that
requires that a program work correctly for all (\(\forall\)) inputs?


\chapter{3. Problems with Imperative Code}
\label{\detokenize{03-problems-with-imperative-code::doc}}\label{\detokenize{03-problems-with-imperative-code:problems-with-imperative-code}}
There’s no free lunch: One can have the expressiveness of mathematical
logic, useful for specification, or one can have the ability to run
code efficiently, along with indispensable ability to interact with an
external environment provided by imperative code, but one can not have
all of this at once at once.

A few additional comments about expressiveness are in order here. When
we say that imperative programming languages are not as expressive as
mathematical logic, what we mean is not ony that the code itself is not
very explicit about what it computes. It’s also that it is profoundly
hard to fully comprehend what imperative code will do when run, in large
part due precisely to the things that make imperative code efficient: in
particular to the notion of a mutable memory.

One major problem is that when code in one part of a complex program
updates a variable (the \sphinxstyleemphasis{state} of the program), another part of the
code, far removed from the first, that might not run until much later,
can read the value of that very same variable and thus be affected by
actions taken much earlier by code far away in the program text. When
programs grow to thousands or millions of lines of code (e.g., as in
the cases of the Toyota unintended acceleration accident that we read
about), it can be incredibly hard to understand just how different and
seemingly unrelated parts of a system will interact.

As a special case, one execution of a procedure can even affect later
executions of the same procedure. In pure mathematics, evaluating the
sum of two and two \sphinxstyleemphasis{always} gives four; but if a procedure written in
Python updates a \sphinxstyleemphasis{global} variable and then incoporates its value into
the result the next time the procedure is called, then the procedure
could easily return a different result each time it is called even if
the argument values are the same. The human mind is simply not powerful
enough to see what can happen when computations distant in time and in
space (in the sense of being separated in the code) interact with each
other.

A related problem occurs in imperative programs when two different
variables, say \sphinxstyleemphasis{x} and \sphinxstyleemphasis{y}, refer to the same memory location. When
such \sphinxstyleemphasis{aliasing} occurs, updating the value of \sphinxstyleemphasis{x} will also change the
value of \sphinxstyleemphasis{y}, even though no explicit assignment to \sphinxstyleemphasis{y} was made. A
piece of code that assumes that \sphinxstyleemphasis{y} doesn’t change unless a change is
made explicitly might fail catastrophically under such circumstances.
Aliasing poses severe problems for both human understanding and also
machine analysis of code written in imperative languages.

Imperative code is thus potentially \sphinxstyleemphasis{unsafe} in the sense that it can
not only be very hard to fully understand what it’s going to do, but
it can also have effects on the world, e.g., by producing output
directing some machine to launch a missile, fire up a nuclear reactor,
steer a commercial aircraft, etc.


\chapter{4. Pure Functional Programming as Runnable Mathematics}
\label{\detokenize{04-runnable-math::doc}}\label{\detokenize{04-runnable-math:pure-functional-programming-as-runnable-mathematics}}
What we’d really like would be a language that gives us everything:
the expressiveness and the \sphinxstyleemphasis{safety} of mathematical logic (there’s no
concept of a memory in logic, and thus no possibility for unexpected
interactions through or aliasing of memory), with the efficiency and
interactivity of imperative code. Sadly, there is no such language.

Fortunately, there is an important point in the space between these
extremes: in what we call \sphinxstyleemphasis{pure functional,} as opposed to imperative,
\sphinxstyleemphasis{programming} languages. Pure functional languages are based not on
commands that update memories and perform I/O, but on the definition
of functions and their application to data values. The expressiveness
of such languages is high, in that code often directly refects the
mathematical definitions of functions. And because there is no notion
of an updateable (mutable) memory, aliasing and interactions between
far-flung parts of programs through \sphinxstyleemphasis{global variables} simply cannot
happen. Furthermore, one cannot perform I/O in such languages. These
languages thus provide far greater safety guarantees than imperative
languages.  Finally, unlike mathematical logic, code in functional
languages can be run with reasonable efficiency, though often not with
the same efficiency as in, say, C++.

In this chapter, you will see how functional languages allow one to
implement runnable programs that closely mirror the mathematical
definitions of the functions that they implement.


\section{The identify function (for integers)}
\label{\detokenize{04-runnable-math:the-identify-function-for-integers}}
An \sphinxstyleemphasis{identity function} is a function whose values is simply the value
of the argument to which it is applied. For example, the identify
function applied to an integer value, \sphinxstyleemphasis{x}, just evaluates to the value
of \sphinxstyleemphasis{x}, itself. In the language of mathematical logic, the definition
of the function would be written like this.
\begin{equation*}
\begin{split}\forall x \in \mathbb{Z}, x.\end{split}
\end{equation*}
In English, this would be pronounced, “for all (\(\forall\))
values, \sphinxstyleemphasis{x}, in (\(\in\)) the set of integers
(\(\mathbb{Z}\)), the function simply reduces to value of \sphinxstyleemphasis{x},
itself. The infinite set of integers is usually denoted in
mathematical writing by a script or bold Z. We will use that
convention in these notes.

While such a mathematical definition is not “runnable”, we can
\sphinxstyleemphasis{implement} it as a runnable program in pure functional language. The
code will then closely reflects the abstract mathematical definition.
And it will run!  Here’s an implementation of \sphinxstyleemphasis{id} written in the
functional sub-language of Dafny.

\begin{sphinxVerbatim}[commandchars=\\\{\}]
function method id (x: int): int \PYGZob{} x \PYGZcb{}
\end{sphinxVerbatim}

The code declares \sphinxstyleemphasis{id} to be what Dafny calls a “function method”,
which indicates two things.  First, the \sphinxstyleemphasis{function} keyword states that
the code will be written in a pure functional, not in an imperative,
style. Second, the \sphinxstyleemphasis{method} keyword instructs the compiler to produce
runnable code for this function.

Let’s look at the code in detail. First, the name of the function is
defined to be \sphinxstyleemphasis{id}. Second, the function is defined to take just one
argument, \sphinxstyleemphasis{x}, declared of type \sphinxstyleemphasis{int}.  The is the Dafny type whose
values represent integers (negative, zero, and positive whole number)
of any size. The Dafny type \sphinxstyleemphasis{int} thus represents (or \sphinxstyleemphasis{implements})
the mathematical set, \({\mathbb Z}\), of all integers. The \sphinxstyleemphasis{int}
after the argument list and colon then indicates that, when applied to
an int, the function returns (or \sphinxstyleemphasis{reduces to}) a value of type \sphinxstyleemphasis{int}.
Finally, within the curly braces, the expression \sphinxstyleemphasis{x}, which we call
the \sphinxstyleemphasis{body} of this function definition, specifies the value that this
function reduces to when applied to any \sphinxstyleemphasis{int}. In particular, when
applied to avalue, \sphinxstyleemphasis{x}, the function application simply reduces to the
value of \sphinxstyleemphasis{x} itself.

Compare the code with the abstract mathematical definition and you
will see that but for details, they are basicaly \sphinxstyleemphasis{isomorphic} (a word
that means identical in structure). It’s not too much of a stretch to
say that pure functional programs are basically runnable mathematics.

Finally, we need to know how expressions involving applications of
this function to arguments are evaluated. They fundamental notion at
the heart of functional programming is this: to evaluate a function
application expression, such as \sphinxstyleemphasis{id(4)}, you substiute the value of
the argument (here \sphinxstyleemphasis{4}) for every occurence of the argument variable
(here \sphinxstyleemphasis{x}) in the body of the function definition, the you evaluate
that expression and return the result. In this case, we substite \sphinxstyleemphasis{4}
for the \sphinxstyleemphasis{x} in the body, yielding the literal expression, \sphinxstyleemphasis{4}, which,
when evaluated, yeilds the value \sphinxstyleemphasis{4}, and that’s the result.


\section{Data and function types}
\label{\detokenize{04-runnable-math:data-and-function-types}}
Before moving on to more interesting functions, we must mention the
concepts of \sphinxstyleemphasis{types} and \sphinxstyleemphasis{values} as they pertain to both \sphinxstyleemphasis{data} and
\sphinxstyleemphasis{functions}. Two types appear in the example of the \sphinxstyleemphasis{id} function. The
first, obvious, one is the type \sphinxstyleemphasis{int}. The \sphinxstyleemphasis{values} of this type are
\sphinxstyleemphasis{data} values, namely values representing integers. The second type,
which is less visible in the example, is the type of the the function,
\sphinxstyleemphasis{id}, itself. As the function takes an argument of type \sphinxstyleemphasis{int} and also
returns a value of type \sphinxstyleemphasis{int}, we say that the type of \sphinxstyleemphasis{id} is
\(int \rightarrow int\). You can pronounce this type as \sphinxstyleemphasis{int to
int}.


\section{Other function values of the same type}
\label{\detokenize{04-runnable-math:other-function-values-of-the-same-type}}
There are many (indeed an uncountable infinity of) functions that
convert integer values to other integer values. All such functions
have the same type, namely \(int \rightarrow int\), but they
constitute different function \sphinxstyleemphasis{values}. While the type of a function
is specified in the declaration of the function argument and return
types, a function \sphinxstyleemphasis{value} is defined by the expression comprising the
\sphinxstyleemphasis{body} of the function.

An example of a different function of the same type is what we will
call \sphinxstyleemphasis{inc}, short for \sphinxstyleemphasis{increment}. When applied to an integer value,
it reduces to (or \sphinxstyleemphasis{returns}) that value plus one. Mathematically, it
is defined as \(\forall x \in {\mathbb Z}, x + 1\). For example,
\sphinxstyleemphasis{inc(2)} reduces to \sphinxstyleemphasis{3}, and \sphinxstyleemphasis{inc(-2)}, to \sphinxstyleemphasis{-1}.

Here’s a Dafny functional program that implements this function. You
should be able to understand this program with ease. Once again, take
a moment to see the relationship between the abstract mathematical
definition and the concrete code. They are basically isomorphic. The
pure functional programmer is writing \sphinxstyleemphasis{runnable mathematics}.

\begin{sphinxVerbatim}[commandchars=\\\{\}]
function method inc (x: int): int \PYGZob{} x + 1 \PYGZcb{}
\end{sphinxVerbatim}

Another example of a function of the same type is, \sphinxstyleemphasis{square}, defined
as returing the square of its integer argument. Mathematically it is
the function, \(\forall x \in {\mathbb Z}, x * x\). And here is
a Dafny implementation.

\begin{sphinxVerbatim}[commandchars=\\\{\}]
function method h (x: int): int \PYGZob{} x * x \PYGZcb{}
\end{sphinxVerbatim}

Evaluating expressions in which this function is applied to an
argument happens as previously described. To evaluate \sphinxstyleemphasis{square(4)}, for
example, you rewrite the body, \sphinxstyleemphasis{x * x}, replacing every \sphinxstyleemphasis{x} with a
\sphinxstyleemphasis{4}, yielding the expression \sphinxstyleemphasis{4 * 4}, then you evaluate that
expression and return the result, here \sphinxstyleemphasis{16}. Function evaluation is
done by substituting actual parameter values for all occurrences of
corresponding formal parameters in the body of a function, evaluating
the resulting expression, and returning that result.

Recursive function definitions and implementations
=================================================+

Many mathematical functions are defined \sphinxstyleemphasis{recursively}. Consider the
familiar \sphinxstyleemphasis{factorial} function. An informal explanation of what the
function produces when applied to a natural number (a non-negative
integer), \sphinxstyleemphasis{n}, is the product of natural numbers from \sphinxstyleemphasis{1} to \sphinxstyleemphasis{n}.

That’s a perfectly understandable definition, but it’s not quite
precise (or even correct) enough for a mathematician. There are at
least two problems with this definition. First, it does not define the
value of the function \sphinxstyleemphasis{for all} natural numbers. In particular, it
does not say what the value of the function is for zero. Second, you
can’t just extend the definition by saying that it yields the product
of all the natural numbers from zero to \sphinxstyleemphasis{n}, because that is always
zero!

Rather, if the function is to be defined for an argument of zero, as
we require, then we had better define it to have the value one when
the argument is zero, to preserve the product of all the other numbers
larger than zero that we might have multiplied together to produce the
result. The trick is to write a mathematical definition of factorial
in two cases: one for the value zero, and one for any other number.
\begin{equation*}
\begin{split}factorial(n) := \forall n \in {\mathbb Z} \mid n >= 0, \begin{cases}
\text{if n=0}, & 1,\\ \text{otherwise}, & n *
factorial(n-1).\end{cases}\end{split}
\end{equation*}
To pronounce this mathematical definition in English, one would say
that for any integer, \sphinxstyleemphasis{n}, such that \sphinxstyleemphasis{n} is greater than or equal to
zero, \sphinxstyleemphasis{factorial(n)} is one if \sphinxstyleemphasis{n} is zero and is otherwise \sphinxstyleemphasis{n} times
\sphinxstyleemphasis{factorial(n-1)}.

Let’s analyze this definition. First, whereas in earlier examples we
left mathematical definitions anonymous, here we have given a name,
\sphinxstyleemphasis{factorial}, to the function, as part of its mathematical definition.
We have to do this because we need to refer to the function within its
own definition.  When a definition refers to the thing that is being
defined, we call the definition \sphinxstyleemphasis{recursive.}

Second, we have restricted the \sphinxstyleemphasis{domain} of the function, which is to
say the set of values for which it is defined, to the non-negative
integers only, the set known as the \sphinxstyleemphasis{natural numbers}. The function
simply isn’t defined for negative numbers.  Mathematicians usually use
the symbol, \({\mathbb N}\) for this set. We could have written
the definition a little more concisely using this notation, like this:
\begin{equation*}
\begin{split}factorial(n) := \forall n \in {\mathbb N}, \begin{cases}
\text{if n=0}, & 1,\\ \text{otherwise}, & n *
factorial(n-1).\end{cases}\end{split}
\end{equation*}
Here, then, is a Dafny implementation of the factorial function.

\begin{sphinxVerbatim}[commandchars=\\\{\}]
function method fact(n: int): int
   requires n \PYGZgt{}= 0 // for recursion to be well founded
\PYGZob{}
    if (n==0) then 1
    else n * fact(n\PYGZhy{}1)
\PYGZcb{}
\end{sphinxVerbatim}

This code exactly mirrors our first mathematical definition. The
restriction on the domain is expressed in the \sphinxstyleemphasis{requires} clause of the
program. This clause is not runnable code. It’s a specification: a
\sphinxstyleemphasis{predicate} (a proposition with a parameter) that must hold for the
program to be used. Dafny will insist that this function only ever be
applied to values of \sphinxstyleemphasis{n} that have the \sphinxstyleemphasis{property} of being \(>=
0\). A predicate that must be true for a program to be run is called a
\sphinxstyleemphasis{pre-condition}.

To see how the recursion works, consider the application of
\sphinxstyleemphasis{factorial} to the natural number, \sphinxstyleemphasis{3}. We know that the answer should
be \sphinxstyleemphasis{6. The evaluation of the expression, *factorial(3)}, works as for
any function application expression: first you subsitute the value of
the argument(s) for each occurrence of the formal parameters in the
body of the function; then you evaluate the resulting expression
(recursively!) and return the result. For \sphinxstyleemphasis{factorial(3)}, this process
leads through a sequence of intermediate expressions as follows (leaving
out a few details that should be easy to infer):
\begin{align*}\!\begin{aligned}
factorial\ (3) & \text{ ; a function application expression}\\
if\ (3 == 0)\ then\ 1\ else\ (3 * factorial\ (3-1)) & \text{ ; expand body with  parameter/argument substitution}\\
if\ (3 == 0)\ then\ 1\ else\ (3 * factorial\ (2))  & \text{ ; evaluate $(3-1)$}\\
if\ false\ then\ 1\ else\ (3 * factorial\ (2)) & \text{ ; evaluate $(3==0)$ }\\
(3 * factorial\ (2)) & \text{ ; evaluate $ifThenElse$ }\\
(3 * (if\ (2==0)\ then\ 1\ else\ (2 * factorial\ (1))) & \text{ ; etc }\\
(3 * (2 * factorial\ (1))\\
(3 * (2 * (if\ (1==0)\ then\ 1\ else\ (1 * factorial\ (0)))))\\
(3 * (2 * (1 * factorial\ (0))))\\
(3 * (2 * (1 * (if\ (0==0)\ then\ 1\ else\ (0 * factorial\ (-1))))))\\
(3 * (2 * (1 * (if\ true\ then\ 1\ else\ (0 * factorial\ (-1))))))\\
(3 * (2 * (1 * 1)))\\
(3 * (2 * 1))\\
(3 * 2)\\
6\\
\end{aligned}\end{align*}
The evaluation process continues until the function application expression
is reduced to a data value. That’s the answer!

It’s important to understand how recursive function application
expressions are evaluated. Study this example with care. Once you’re
sure you see what’s going on, go back and look at the mathematical
definition, and convince yourself that you can understand it \sphinxstyleemphasis{without}
having to think about \sphinxstyleemphasis{unrolling} of the recursion as we just did.

Finally we note that the the precondition is essential. If it were not
there in the mathematical definition, the definition would not be what
mathematicians call \sphinxstyleemphasis{well founded}: the recursive definition might
never stop looping back on itself. Just think about what would happen
if you could apply the function to \sphinxstyleemphasis{-1}. The definition would involve
the function applied to \sphinxstyleemphasis{-2}. And the definition of that would involve
the function applied to \sphinxstyleemphasis{-3}. You can see that there will be an
infinite regress.

Similarly, if Dafny would allow the function to be applied to \sphinxstyleemphasis{any}
value of type \sphinxstyleemphasis{int}, it would be possible, in particular, to apply the
function to negative values, and that would be bad!  Evaluating the
expression, \sphinxstyleemphasis{factorial(-1)} would involve the recursive evaluation of
the expression, \sphinxstyleemphasis{factorial(-2)}, and you can see that the evaluation
process would never end. The program would go into an “infinite loop”
(technically an unbounded recursion). By doing so, the program would
also violate the fundamental promise made by its type: that for \sphinxstyleemphasis{any}
integer-valued argument, an integer result will be produced. That can
not happen if the evaluation process never returns a result. We see
the precondition in the code, implementing the domain restriction in
the mathematical definition, is indispensible. It makes the definition
sound and it makes the code correct!


\section{Dafny is a Program Verifier}
\label{\detokenize{04-runnable-math:dafny-is-a-program-verifier}}
Restricting the domain of factorial to non-negative integers is
critical. Combining the non-negative property of ever value to which
the function is applied with the fact that every recursive application
is to a smaller value of \sphinxstyleemphasis{n}, allows us to conclude that no \sphinxstyleemphasis{infinite
decreasing chains} are possible. Any application of the function to a
non-negative integer \sphinxstyleemphasis{n} will terminate after exactly \sphinxstyleemphasis{n} recursive
calls to the function. Every non-negative integer, \sphinxstyleemphasis{n} is finite. So
every call to the function will terminate.

Termination is a critical \sphinxstyleemphasis{property} of programs. The proposition that
our factorial program with the precondition in place always terminates
is true as we’ve argued. Without the precondition, the proposition is
false.

Underneath Dafny’s “hood,” it has a system for proving propositions
about (i.e., properties of) programs. Here we see that It generates a
propostion that each recursive function terminates; and it requires a
proof that each such proposition is true.

With the precondition in place, there not only is a proof, but Dafny
can find it on its own. If you remove the precondition, Dafny won’t be
able to find a proof, because, as we just saw, there isn’t one: the
proposition that evaluation of the function always terminates is not
true. In this case, because it can’t prove termination, Dafny will
issue an error stating, in effect, that there is the possibility that
the program will infinitely loop. Try it in Dafny.  You will see.

In some cases there will be proofs of important propositions that
Dafny nevertheless can’t find it on its own. In such cases, you may
have to help it by giving it some additional propositions that it
can verify and that help point it in the right direction. We’ll see
more of this later.

The Dafny language and verification system is powerful mechansim for
finding subtle bugs in code, but it require a knowledge of more than
just programming. It requires an understanding of specification, and
of the languages of logic and proofs in which specifications of code
are expressed and verified.


\chapter{5. Formal Verification of Imperative Programs}
\label{\detokenize{05-verifying-logical-specifications::doc}}\label{\detokenize{05-verifying-logical-specifications:formal-verification-of-imperative-programs}}
In this chapter, we first elaborate on the idea that pure functional
programming make for mathematically clear but potentially inefficient
specifications, while imperative code makes for efficient code but is
hardly clear as to its purpose, and is thus hard to reason about. To
get the benefits of both, we use functional programming to write key
parts of specifications for imperative code, and then we use tools or
manual methods to \sphinxstyleemphasis{prove} that the imperative code does what such a
speification requires.


\section{Performance vs. Undertandability}
\label{\detokenize{05-verifying-logical-specifications:performance-vs-undertandability}}
To get a clearer sense of the potential differences in performance
between a pure functional program and an imperative program that
compute the same function, and tradeoffs one makes between clarity of
intent and execution speed, consider our recursive definition of the
Fibonacci function.

We start off knowing that if the argument to the function, \sphinxstyleemphasis{n}, is \sphinxstyleemphasis{0}
or \sphinxstyleemphasis{1}, the value of the function for that \sphinxstyleemphasis{n} is just \sphinxstyleemphasis{n} itself.  In
other words, the sequence, \sphinxstyleemphasis{fib(i)} of \sphinxstyleemphasis{Fibonacci numbers indexed by
i}, starts with, \([0, 1, \ldots ]\).  For any \sphinxstyleemphasis{n \textgreater{}= 2}, \sphinxstyleemphasis{fib(n)},
is the sum of the previous two values.  To compute the \sphinxstyleemphasis{n’th (n \textgreater{}= 2)}
Fibonacci number, we can thus start with the first two, sum them up to
get the next one, then iterate this process, computing the next value
on each iteration, until we’ve got the result.

Footnote: by convention we index sequences starating at zero rather
than one. The first element in such a sequence thus has index \sphinxstyleemphasis{0}, the
second has index \sphinxstyleemphasis{1}, and the \sphinxstyleemphasis{n’th} has index \sphinxstyleemphasis{n - 1}. For example,
\sphinxstyleemphasis{fib(6)} refers to the \sphinxstyleemphasis{7th} Fibonacci number. You should get used to
thinking in terms of zero-indexed sequences.

Now consider our recursive definition, \sphinxstyleemphasis{fib(n)}. It’s \sphinxstyleemphasis{pure math}:
concise, precise, elegant.  And because we’ve written it in a
functional language, we can even run it. However, it might not give us
the performance we require. An imperative program, by constrast, is
\sphinxstyleemphasis{code}. It’s cryptic but it can be very efficient when run.

To get a sense of performance diferences, consider the evaluation of
each of two programs to compute \sphinxstyleemphasis{fib(5)}: our functional program and
an imperative one that we will develop in this chapter.

Consider the imperative program. If the argument, \sphinxstyleemphasis{n}, is either zero
or one, the answer is just returned. If \sphinxstyleemphasis{n \textgreater{}= 2} an answer has to be
computed. In this case, the program will repeatedly add together the
previous two values of the function, starting with \sphinxstyleemphasis{0} and \sphinxstyleemphasis{1}, until
it computes the result for \sphinxstyleemphasis{n}.  The program returns that value.

For a given value of \sphinxstyleemphasis{n}, what is the cost of computing an answer?
The cost will be dominated by the work done inside the loop body; and
on each iteration of the loop, a fixed amount of work is done; so it’s
not a bad idea to use the number of loop body executions as a measure
of the cost of computing an answer for an argument, \sphinxstyleemphasis{n}.

So, what does it cost to compute \sphinxstyleemphasis{fib(5)}? Well, we need to execute
the loop body to compute \sphinxstyleemphasis{fib(i)} for values of \sphinxstyleemphasis{i} of \sphinxstyleemphasis{2, 3, 4,} and
\sphinxstyleemphasis{5}. It thus takes \sphinxstyleemphasis{4} loop body iterations to compute \sphinxstyleemphasis{fib(5)}. To
compute the 10th element requires that the loop body execute for \sphinxstyleemphasis{i}
in the range of \sphinxstyleemphasis{{[}2, 3, …, 10{]}}. That’s nine iterations.  It’s easy
to see that for any value of \sphinxstyleemphasis{n}, the cost to compute \sphinxstyleemphasis{fib(n)} will be
\sphinxstyleemphasis{n-1} loop body iterations. We can compute the \sphinxstyleemphasis{100,000th} Fibonacci
number by running a simple loop body \sphinxstyleemphasis{about} that many times. On a
modern computer, the computation will be completed very quickly.

The functional program, on the other hand, is evaluated by repeated
evaluation of nested recursive function applications until base cases
are reached.  Let’s think about the cost of evaluation for increasing
values of \sphinxstyleemphasis{n} and try to see a pattern. We’ll measure computational
complexity now in terms of the number of function evaluations (rather
than loop bodies executed) required to produce a final answer.

To compute \sphinxstyleemphasis{fib(0)} or \sphinxstyleemphasis{fib(1)} requires just \sphinxstyleemphasis{1} function evaluation
(the first and only call to the function), as these are base cases
requiring no further recursion. To compute \sphinxstyleemphasis{fib(2)} however requires
\sphinxstyleemphasis{3} evalations of \sphinxstyleemphasis{fib}: one for each of \sphinxstyleemphasis{fib(1)} and \sphinxstyleemphasis{fib(0)} plus
the evaluation of the top-level function. The relationship between \sphinxstyleemphasis{n}
and the number of function evaluations currently looks like this:
\(\{ (0,1), (1,1), (2,3), ... \}.\) The first element of each pair
is \sphinxstyleemphasis{n} and the second element is the cost to compute \sphinxstyleemphasis{fib(n)}.

What about when \sphinxstyleemphasis{n} is \sphinxstyleemphasis{3}?  Computing this requires answers for
\sphinxstyleemphasis{fib(2)}, which by the resuts we just computed costs \sphinxstyleemphasis{3} evaluations,
and for \sphinxstyleemphasis{fib(1)}, which costs \sphinxstyleemphasis{1}, for a total of \sphinxstyleemphasis{5} evaluations
including the top-level evaluation. Computing \sphinxstyleemphasis{fib(4)} requires that
we compute \sphinxstyleemphasis{fib(3)} and \sphinxstyleemphasis{fib(2)}, costing \sphinxstyleemphasis{5 + 3}, or \sphinxstyleemphasis{8} evaluations,
plus the original, top-level call, for a total of 9. For \sphinxstyleemphasis{fib(5)} we
need \sphinxstyleemphasis{9} + \sphinxstyleemphasis{5}, or \sphinxstyleemphasis{14} plus one more, making \sphinxstyleemphasis{15} evaluations.  The
relation of cost to \sphinxstyleemphasis{n} (the problem size) is now like this: \(\{
(0,1), (1,1), (2,3), (3,5), (4,9), (5, 15), ... \}.\)

In general, the number of evaluations needed to evaluate \sphinxstyleemphasis{fib(i+1)} is
the sum of the numbers required to evaluate \sphinxstyleemphasis{fib(i)} plus the number
to evaluate \sphinxstyleemphasis{fib(i-1)} plus \sphinxstyleemphasis{1.} If we use \sphinxstyleemphasis{C} to represent the cost
function, then we could say, \(C(n) = C(n-1) + C(n-2) + 1\). This
kind of function is called a recurrence relation, and there are clever
ways to solve such functions to determine what function \sphinxstyleemphasis{C} me be. Of
course we can also write a recursive function to compute \sphinxstyleemphasis{C(n)}, if
we need only to compute it for relatively small values of \sphinxstyleemphasis{n}.

Now that we have the formula, we can quickly compute the costs to
compute \sphinxstyleemphasis{fib(n)} for numerous values of \sphinxstyleemphasis{n}. The number of evaluations
needed to compute \sphinxstyleemphasis{fib(6)} is \sphinxstyleemphasis{15 + 9 + 1}, i.e., 25. For \sphinxstyleemphasis{fib(7)}
it’s \sphinxstyleemphasis{41}.  For \sphinxstyleemphasis{fib(8), *67}; for \sphinxstyleemphasis{fib(9), 109}; for \sphinxstyleemphasis{fib(10), 177};
and for \sphinxstyleemphasis{fib(11), 286} function evaluations.

One thing is clear: The cost to compute the \sphinxstyleemphasis{n’th} Fibonacci number,
as measured by the number of function evaluations, using our beautiful
functional program, is growing much more quickly than \sphinxstyleemphasis{n} itself, and
indeed it is growing faster and faster as \sphinxstyleemphasis{n} increases. We would say
the cost is \sphinxstyleemphasis{super-linear}, whereas with our imperative program, the
number of loop body interations grows \sphinxstyleemphasis{linearly} in \sphinxstyleemphasis{n}.

How exactly does the cost of the pure functional program compare? One
thing to notice is that the cost of computing a Fibonacci element with
our functional program is related to the Fibonacci sequence itself!
The first two values in the \sphinxstyleemphasis{cost} sequence are \sphinxstyleemphasis{1} and \sphinxstyleemphasis{1}, and each
subsequence element is the sum of the previous two \sphinxstyleemphasis{plus 1}.  It’s not
exactly the Fibonacci sequence, but it turns out to grow at a very
similar rate. The Fibonacci sequence, thus also the cost of computing
it recursively, grows at what turns out to be a rate \sphinxstyleemphasis{exponential} in
\sphinxstyleemphasis{n}, with an exponent of about \sphinxstyleemphasis{1.6}. Increasing \sphinxstyleemphasis{n} by \sphinxstyleemphasis{1} doesn’t
just add a little to the cost; it almost doubles it (multiplying it by
a factor of \sphinxstyleemphasis{1.6}).

No matter how small the exponent (with any exponent greater than one),
exponential functions eventually grow very quickly. In the limit, any
exponential function grows faster than any polynomial no matter how
high in rank it is and no matter how large its coefficients are.

The exponential-in-\sphinxstyleemphasis{n} cost of our clear but inefficient functional
program grows far faster than the cost of our ugly but efficient
imperative program as we increase \sphinxstyleemphasis{n}.  For any even modestly large
value of \sphinxstyleemphasis{n} (e.g., greater than 50 or so), it will be impractical to
use the pure functional program, whereas the imperative program will
reasonably run quickly even on a small personal computer for values of
\sphinxstyleemphasis{n} well into in the millions.  What eventually slows it down is not
the number of additions that it has to do but the sizes of the numbers
that it has to add.

You can already see that the cost to compute \sphinxstyleemphasis{fib(n)} recursively for
values of \sphinxstyleemphasis{n} larger than just ten or so is much greater than the cost
to compute it iteratively. Our mathematical/functional definition is
clear (“intellectually tractable”) but inefficient. The imperative
program, on the other hand, is efficient, but not at all transparent.
We need the latter program for practical computation. But how do we
ensure that it implements the same function that we expressed in our
elegant mathematical definition?


\section{Specification, Implementation, and Verification}
\label{\detokenize{05-verifying-logical-specifications:specification-implementation-and-verification}}
We address such problems by combining a few ideas. First, we use
logic, including mathematical specifications written in part using
functional programming, to express \sphinxstyleemphasis{declarative} specifications.  Such
specification precisely define \sphinxstyleemphasis{what} a given imperative program must
compute, and in particular what results it must return as a function
of the arguements it receives, without saying \sphinxstyleemphasis{how} the computation
should be done.

We can use functions defined in the pure functional programming style
as parts of specifications, e.g., as giving a mathematical definition
of the \sphinxstyleemphasis{factorial} function that an imperative program will then have
to implement.

Second, we implement the specified program in an imperative language.
Ideally we do so in a way that supports logical reasoning about its
behavior. For example, we have to specify not only the relationship
between argument and result values that are required, but also how
loops are designed to work in our code. We then need to design loops
in ways that make it easier to explain, in formal logic, how they do
what they are meant to do.

Finally, we use logical proofs to \sphinxstyleemphasis{verify} that the program satisifies
its specification. Later in this course, we’ll see how to create such
proofs ourselves. For now we’ll be happy to let Dafny generate them
for us mostly automatically!

The rest of this chapter develops these ideas in more depth with
concrete examples.  First we explain how formal specifications in
mathematical logic for imperative programs are often organized. Next
we explore how writing imperative programs without the benefits of
specification languages and verifications tools can make it hard to
spot bugs in code. Next we enhance our implementation of the factorial
function with specifications, show how Dafny flags the bug, and fix
the program. Doing this requires that we deepen the way we understand
loops. We end with a detailed presentation of the verification of an
imperative program to compute values in the Fibonacci sequence. Given
any natural number \sphinxstyleemphasis{n}, our program must return the value of \sphinxstyleemphasis{fib(n)},
but it must also do it efficiently.  The design and precise, logical
description of key properties of a loop is once again the heart of the
problem.  We will see how Dafny can help us to reason rigorously about
loops, and that giving it a little help enables it to reason about
them for us.


\section{Declarative Input-Output Specifications}
\label{\detokenize{05-verifying-logical-specifications:declarative-input-output-specifications}}
First, we use mathematical logic to \sphinxstyleemphasis{declaratively specify} properties
of the behaviors that we require of programs written in \sphinxstyleemphasis{imperative}
languages. For example, we might require that, when given \sphinxstyleemphasis{any}
natural number, \sphinxstyleemphasis{n}, a program compute and return the value of the
\sphinxstyleemphasis{factorial} of \sphinxstyleemphasis{n}, the mathematical definition of which we’ve given
as \sphinxstyleemphasis{fact(n)}.  In general, we want to specify how the results returned
by an imperative program relate to the arguments on which it was run.
We call such a specification an \sphinxstyleemphasis{input-output} specification. (Here
we ignore \sphinxstyleemphasis{side-effect} behaviors such as reading from and writing
to input and output devices.)

Specifications about required relationships between argument values
and return results specify \sphinxstyleemphasis{what} a program must compute without
specifying how it should be done. Specifications are thus \sphinxstyleemphasis{abstract}:
they omit \sphinxstyleemphasis{implementation details}, leaving it to the programmer to
decide how best to \sphinxstyleemphasis{refine} the specification into efficient code.

For example we might specify that a program (1) must accept any
integer valued argument greater than or equal to zero (a piece of a
specification that we call a \sphinxstyleemphasis{precondition}), and (2) that as long as
the precondition holds, then it must return the factorial of the given
argument value (a \sphinxstyleemphasis{postcondition}).


\subsection{Input-Output Relations}
\label{\detokenize{05-verifying-logical-specifications:input-output-relations}}
In purely mathematical terms, a specification of this kind defines a
\sphinxstyleemphasis{binary relation} between argument (input) and return (output) values,
and imposes on the program a requirement that whenever it is given the
first value in such an \sphinxstyleemphasis{input-output} pair, it must compute a second
(output) value so that the pair, \((input, output)\), is in the
specified relation.


\subsection{Relations and Functions}
\label{\detokenize{05-verifying-logical-specifications:relations-and-functions}}
A binary relation in ordinary mathematics is just a set of pairs of
values. A function is a binary relation with at most one pair with a
given first value. A function is a \sphinxstyleemphasis{single-valued} relation. What we
often need to specify, in particular, is an input-output \sphinxstyleemphasis{function}.

For example, pairs in the factorial relation include \((0,1),
(1,1), (2,2), (3,6), (4,24)\) and \((5,120)\), but not the pair
\((5,25)\). Some of the pairs in the Fibonacci relation include
\(\{ (0,0), (1,1), (2,1), (3,2), (5,3)\) and \sphinxstyleemphasis{(6,5)}. These
relations are also \sphinxstyleemphasis{functions} because there is \sphinxstyleemphasis{at most} one pair
with a given first element. Finally, these functions are also said to
be \sphinxstyleemphasis{total} because for \sphinxstyleemphasis{every} natural number, there is a pair with
that number as its first element.

On the other hand, square root is a \sphinxstyleemphasis{relation}, a set of pairs of real
numbers, but not a \sphinxstyleemphasis{function}, because it is not singled valued. Both
of the pairs, \((4,2)\) and \((4,-2)\), which are distinct but
have same first element, are in the relation. That is so because both
\sphinxstyleemphasis{2} and \sphinxstyleemphasis{-2} are square roots of \sphinxstyleemphasis{4}.


\subsection{Total and Partial Functions}
\label{\detokenize{05-verifying-logical-specifications:total-and-partial-functions}}
We also note that the square root relation \sphinxstyleemphasis{on the real numbers} is
what we call \sphinxstyleemphasis{partial} rather than total: in that it is not defined
for some real numbers. In particular, it is not defined for (i.e., it
does not have any pairs where the first element is) any negative real
number.


\subsection{Turning Partial Functions into Total Functions}
\label{\detokenize{05-verifying-logical-specifications:turning-partial-functions-into-total-functions}}
Partial functions and non-function relations both present problems for
programmers. Let’s first consider relations that sometimes have \sphinxstyleemphasis{more}
than one value of a given type for a given argument. What value should
a program return?

The square root function is a good example. Given a positive argument
there will be \sphinxstyleemphasis{two} square roots, one positive and one negative. If
the function is require to return a single number as an answer, which
one should it return?

There is really no good answer. Rather, the solution is usually to
change the program specification slightly. For example, rather than
promising to return \sphinxstyleemphasis{the} square root (a concept that is not well
defined when there are two square roots for the same number) such a
program might promise to return the non-negative square root, of which
there is always just one (given a non-negative argument. What we have
done here is to implement a different relation, and one that is now
also a function.

A different way to re-formulate the square root \sphinxstyleemphasis{relation} as a
\sphinxstyleemphasis{function} would be to view it as returning a single \sphinxstyleemphasis{set} of values
as a result: a set containing all of the square roots of a given
argument.  The pair \((4, \{2, -2\}\) is in this relation, for
example, and the relation is also a function in that there is only one
such pair with any given first element.

So far we have dealt with the situation where a relation holds more
than one result for a given argument. The other difficult situation
occurs when there is no result or a given argument, i.e., when the
function or relation is undefined for some argument values. What
should a program return then?

Once again, there’s no good answer. Rather, we generally tweak the
specification to require the implementation of a slightly different
relation. One approach would be to narrow the domain of values that
the \sphinxstyleemphasis{program} can take to the domain on which the actual mathematical
function is defined. So instead of specifying a square root function
as taking any real number, we could speficy that it requires that an
argument value be non-negative. When we add such a precondition to a
method or function specification in Daphy, the effect is that Dafny
checks every place in the code where the method or function is called
to verify that the argument values satisfy that pre-condition.

Alternately, we might “tweak” the type of the return value, so that
the program can return some value of the promised type, even if the
underlying mathematical function is not defined for the arguments. So,
for example, if instead of promising to return a single number as a
square root we promise to return a set of numbers, then in cases where
the function is undefined, we just return the empty set of numbers.
In this case, the empty set as a return value can be interpreted as
signifying that no numerical answer could be returned.

Finally, in languages such as Java and Python, when a program
encounters a state where a valid value cannot be computed and
returned, it can invoke an error handling routine to take some kind of
“exceptional” action. This is the purpose of exceptions in Java,
Python, etc. We will not entertain the use of exceptions in this
course.


\section{Imperative Implementation}
\label{\detokenize{05-verifying-logical-specifications:imperative-implementation}}
Having written a formal specification of the required \sphinxstyleemphasis{input-output}
behavior of a program, we next write imperative code in a manner, and
in a language, that supports the use of formal logic to \sphinxstyleemphasis{reason} about
whether the program refines (implements) its formal specification. One
can use formal specifications when programming in any language, but it
helps greatly if the language has strong, static type checking. It is
even better if the language supports formal specification and logical
reasoning mechanisms right alongside of its imperative and functional
programming capabilities. Dafny is such a language and system. It is
not just a language, but a verifier.

In addition to choosing a language with features that help to support
formal reasoning, we sometimes also aim to write imperative code in a
way that makes it easier to reason about formally. As we’ll see below,
for example, the way that we write our while loops can make it easier
or harder to reason about their correctness. Even whether we iterate
from zero up to \sphinxstyleemphasis{n} or from \sphinxstyleemphasis{n} down to zero can affect the difficulty
of writing specification elements for a program.


\section{Formal Verification}
\label{\detokenize{05-verifying-logical-specifications:formal-verification}}
The aim of formal verification is to deduce (to use deductive logic to
\sphinxstyleemphasis{prove}) that, as written, a program satisfies its specification.  In
more detail, if we’re given a program, \sphinxstyleemphasis{C} with a precondition, \sphinxstyleemphasis{P},
and a postcondition \sphinxstyleemphasis{Q}, we want a proof that verifies that if \sphinxstyleemphasis{C} is
started in a state that satisfies \sphinxstyleemphasis{P} and if it terminates (doesn’t go
into an infinite loop), that it ends in a state that satisfies \sphinxstyleemphasis{Q}. We
call this property \sphinxstyleemphasis{partial correctness}.

We write the proposition that \sphinxstyleemphasis{C} is partially correct (that if it’s
started in a state that satisfies the assertion, \sphinxstyleemphasis{P}, and that if it
terminates, then it will do so in a state that satisfies assertion
\sphinxstyleemphasis{Q}) as \(P \{ C \} Q.\) This is a so-called \sphinxstyleemphasis{Hoare triple}, named
after the famous computer scientist, Sir Anthony (Tony) Hoare. It is
nothing other than a proposition that claims that \sphinxstyleemphasis{C} satisfies its
\sphinxstyleemphasis{pre-condition/post-condition} specification. Another way to read it
is as saying that the combination of the pre-condition being satisfied
and the the program being run implies that the post-condition will be
satisfied.

In addition to a proof of partial correctness, we usually do want to
know that a program also does always terminate. When we have a proof
of both \(P \{ C \} Q\) and that the program always terminates,
then we have a proof of \sphinxstyleemphasis{total correctness}. Dafny is a programming
system that allows us to specify \sphinxstyleemphasis{P} and \sphinxstyleemphasis{Q} and it then formally, and
to a considerable extent automatically, verifies both \sphinxtitleref{P \{ C \} Q}
and termination.  That is, Dafny produces proofs of total correctness.

It is important to bear in mind that a proof that a program refines
(implements) its formal specification does not necessarily mean that
it is fit for its intended purpose! If the specification is wrong,
then all bets are off, even if the program is correct relative to its
specification.  The problem of \sphinxstyleemphasis{validating} specification againts
real-world needs is separate from that of \sphinxstyleemphasis{verifying} that a given
program implements its specification correctly. Formal methods can
help here, as well, by verifying that \sphinxstyleemphasis{specifications} have certain
desired properties, but formal validation of specifications is not
our main concern at the moment.


\section{Case Study: The Factorial Function}
\label{\detokenize{05-verifying-logical-specifications:case-study-the-factorial-function}}
So far the material in this chapter has been pretty abstract. Now
we’ll see what it means in practice.


\subsection{A Buggy Implementation}
\label{\detokenize{05-verifying-logical-specifications:a-buggy-implementation}}
To start, let’s consider an ordinary imperative program, as you might
have written in Python or Java, for computing values of the factorial
function. The name of the function is the only indication here of the
intended behavior of this program. There is no clear specification.

The program takes an argument of type nat (which guarantees that the
argument has the property of being non-negative). It then returns a
nat which the programmer implicitly claims (given the function name)
is the factorial of the argument.

\begin{sphinxVerbatim}[commandchars=\\\{\}]
method factorial(n: nat) returns (f: nat)
\PYGZob{}
    if (n == 0)
    \PYGZob{}
        return 1;
    \PYGZcb{}
    var t: nat := n;
    var a: nat := 1;
    while (t !=  0)
    \PYGZob{}
        a := a * n;
        t := t \PYGZhy{} 1;
    \PYGZcb{}
    f := a;
\PYGZcb{}
\end{sphinxVerbatim}

It’s not immediately obvious whether this code is correct or not,
relative to what we know it’s meant to do. Sadly, this program also
contains a bug. Try to find it. Reason about the behavior of the
program when the argument is 0, 1, 2, 3, etc.  Does it always compute
the right result? Where is the bug? What is wrong? And how could this
logical error have been detected automatically?


\subsection{Specifications Establish Correctness Criteria}
\label{\detokenize{05-verifying-logical-specifications:specifications-establish-correctness-criteria}}
A key problem is that the program lacks a precise specification. The
program does \sphinxstyleemphasis{something}, taking a nat and possibly returning a \sphinxstyleemphasis{nat}
(unless it goes into an infinite loop) but there’s no way to analyze
its correctness in the absence of a specification that defines what it
even means to be correct.

Now let’s see what happens when we add a formal specification.  Look
at the following code block. That \sphinxstyleemphasis{n \textgreater{}= 0} continues to be expressed
by the \sphinxstyleemphasis{type} of the argument, \sphinxstyleemphasis{n}, being \sphinxstyleemphasis{nat}. However, we have now
added a postcondition that \sphinxstyleemphasis{ensures} that the return result will be
the factorial of \sphinxstyleemphasis{n} as defined by our functional program!  What we
assert is that the result produced by our imperative code is the same
result that \sphinxstyleemphasis{would have been produced} if we had run the functional
program.

\begin{sphinxVerbatim}[commandchars=\\\{\}]
method factorial(n: nat) returns (f: nat)
    ensures f == fact(n)
\PYGZob{}
    if (n == 0)
    \PYGZob{}
        return 1;
    \PYGZcb{}
    var t := n;
    var a := 1;
    while (t !=  0)
    \PYGZob{}
        a := a * n;
        t := t \PYGZhy{} 1;
    \PYGZcb{}
    return a;
\PYGZcb{}
\end{sphinxVerbatim}

With a specification in place, Dafny now reports that it cannot
guarantee—formally prove to itself—that the \sphinxstyleemphasis{postcondition} is
guaranteed to hold. Generating proofs is hard, not only for people but
also for machines. In fact, one of seminal results of 20th century
mathematical logic was to prove that there is no general-purpose
algorithm for proving propositions in mathematical logic. That’s good
news for mathematicians!  If this weren’t true, we wouldn’t need them!

So, the best that a machine can do is to try to find a proof for any
given proposition. Sometimes proofs are easy to generate. For example,
it’s easy to prove \sphinxstyleemphasis{1 = 1} by the \sphinxstyleemphasis{reflexive} propery of equality.
Other propositions can be hard to prove. Proving that programs in
imperative languages satisfy declarative specifications can be hard.

When Dafny fails to verify a program (find a proof that it satisfies
its specification), there is one of two reasons. Either the program
really does fail to satisfy its specification; or the program is good
but Dafny does not have enough information to know how to prove it.

With the preceding program, the postcondition really isn’t satisfied
due to the bug in the program. When Dafny fails to verify, it gives
us a strong reason to double-check our code to be sure we have not
made some kind of mistake in reasoning.

But even if the program were correct, Dafny would still need a little
more than is given here to prove it. In particular, Dafny would need a
litte more information about how the while loop behaves. It turns out
that providing such extra information about while loops is where much
of the difficulty lies.


\subsection{A Verified Implementation of the Factorial Function}
\label{\detokenize{05-verifying-logical-specifications:a-verified-implementation-of-the-factorial-function}}
Here, then, is a verified imperative program for computing
factorial. We start by documenting the overall program specification.
The key element here is the ensures clause. This clause links our
imperative program with our functional specification and tells Dafny
to make sure that the reuqired relationship holds.

\begin{sphinxVerbatim}[commandchars=\\\{\}]
method verified\PYGZus{}factorial(n: nat) returns (f: nat)
    ensures f == fact(n)
\end{sphinxVerbatim}

Now for the body of the method. First, if we’re looking at the case
where \sphinxstyleemphasis{n==0} we just return the right answer immediately. There is
no need for any further computation.

\begin{sphinxVerbatim}[commandchars=\\\{\}]
if (n == 0)
\PYGZob{}
    return 1;
\PYGZcb{}
\end{sphinxVerbatim}

The rest of the code handles the case where \sphinxstyleemphasis{n \textgreater{} 1}. At this point in
the program execution, we believe that \sphinxstyleemphasis{n} must be greater than zero.
We would have just returned if it were zero, and it can’t be negative
because its type is \sphinxstyleemphasis{nat}. We can nevertheless formally assert (write
a proposition about the state of the program) that \sphinxstyleemphasis{n} is greater than
zero. Dafny will try to (and here will successfully) verify that the
assertion is true at this point in the program, no matter what path
through conditionals, while loops, commands led to this point in the
program.

\begin{sphinxVerbatim}[commandchars=\\\{\}]
assert n \PYGZgt{} 0;
\end{sphinxVerbatim}

To compute an answer for the non-zero input case, we will use a loop.
We can do this by using a variable, a, to hold a “partial factorial
value” in the form of a product of the numbers from n down to a loop
index, “i,” that we start at n and decrement, terminating the loop
when \sphinxstyleemphasis{n==0}.

At each point just before, during, and right after the loop, \sphinxstyleemphasis{a} is a
product of the numbers from \sphinxstyleemphasis{n} down to but not including \sphinxstyleemphasis{i}, and the
value of \sphinxstyleemphasis{i} represents how much product-computing work remains to be
done. So, for example, if we’re computing factorial(10) and a holds
the value \sphinxstyleemphasis{10 * 9}, then \sphinxstyleemphasis{i} must be \sphinxstyleemphasis{8} because multiplying \sphinxstyleemphasis{a} by
the factors from \sphinxstyleemphasis{8} to \sphinxstyleemphasis{1} remains to be done.

A critical “invariant” then is that if you multiply \sphinxstyleemphasis{a} by the
factorial of \sphinxstyleemphasis{i} you get the the factorial of \sphinxstyleemphasis{n}.  When we say that
this is an invariant, we mean that it holds before and also after any
execution of the loop body, but not necessarily within the loop
body. In particular, when \sphinxstyleemphasis{i} gets down to \sphinxstyleemphasis{0}, this relation means
that \sphinxstyleemphasis{a} must contain the final result, because \sphinxstyleemphasis{a * fact(0)} will
then equal \sphinxstyleemphasis{fact(n)} and \sphinxstyleemphasis{fact(0)} is just \sphinxstyleemphasis{1}, so \sphinxstyleemphasis{a} must equal
\sphinxstyleemphasis{fact(n)}.

This is how we design loops so that we can be confident that they do
what we want tem to do. So now let’s go through the steps required to
implement our looping strategy.

Step 1. Set up state for the loop to work. We first initializie a := 1
and i := n.

\begin{sphinxVerbatim}[commandchars=\\\{\}]
var i: nat := n;    // nat type of i explicit
var a := 1;         // can let Dafny infer it
\end{sphinxVerbatim}

It would now be a good idea to ask Dafny to check that the invariant
holds. See the next bit of code, below. Note that we are again using
our pure functional definition, \sphinxstyleemphasis{fact}, as a \sphinxstyleemphasis{specification} of the
function we’re implementing.

In Dafny, we can use matnematical logic to express what must be true
at any given point in the execution of a program in the form of an
“assertion.” Here we assert that our loop invariant holds. The Dafny
verifier tries to prove that the assertion is a true propsition about
the state of the program when control reaches this point, no matter
what path might have been taken to arrive at this point.

\begin{sphinxVerbatim}[commandchars=\\\{\}]
assert a * fact(i) == fact(n); // \PYGZdq{}invariant\PYGZdq{}
\end{sphinxVerbatim}

Step 2: Now we write the actual loop command. Recall how a \sphinxstyleemphasis{while}
loop works. To evaluate a loop, one evaluates the loop condition. If
the result is false, the loop body does not execute and the loop
terminates.  Otherwise, the loop body is executed once and then the
whole loop is run again (starting with a new evaluation of the loop
condition).

We want our loop body to run at least once, as we already handled the
case where it doesn’t need to run at all. It will run if i \textgreater{} 0. What
is i? We initialized it to n and haven’t change it since then so it
must still be equal to n. Do we know that n is greater than 0? We do,
because (1) it can’t be negative owning to its type, and (2) it can’t
be 0 because if it were 0 the program would already have returned.

We can now do better than just reasoning in our heads. We can also use
logic to express what we believe to be true and let Dafny try to check
it for us automatically.

\begin{sphinxVerbatim}[commandchars=\\\{\}]
assert i \PYGZgt{} 0;
\end{sphinxVerbatim}

Now if \sphinxstyleemphasis{i} is one, then the loop body will run once. The value of \sphinxstyleemphasis{a},
which starts at 1, will be multiplied by i, which is 1, then i will be
decremented, taking the the value 0. The loop will be run again, but
the loop condition will be found to be false, and to the loop body
will not be executed and the loop will terminate. When it does, it
will leave \sphinxstyleemphasis{a} with the value 1, which is the right answer.

\begin{sphinxVerbatim}[commandchars=\\\{\}]
while (i \PYGZgt{}  0)
    invariant 0 \PYGZlt{}= i \PYGZlt{}= n
    invariant fact(n) == a * fact(i)
\PYGZob{}
    a := a * i;
    i := i \PYGZhy{} 1;
\PYGZcb{}
\end{sphinxVerbatim}

If \sphinxstyleemphasis{i} is greater than 1, the loop body will execute, multiplying \sphinxstyleemphasis{a}
by the current value of \sphinxstyleemphasis{i} and \sphinxstyleemphasis{i} will be decremented. The vaue of
\sphinxstyleemphasis{a} will be the partial value of the factorial computed so far, and
the value of \sphinxstyleemphasis{i} will represent the work that remains to be done. When
\sphinxstyleemphasis{i} reaches zero, all the work will be done, and \sphinxstyleemphasis{a} will contain the
final result.

However, Dafny cannot determine on its own that this will be the case.
What it needs to know to reason “mechanically” about the program is a
bit of additional information about what remains true no matter \sphinxstyleemphasis{how}
many times the loop body executes (zero or more). That information is
expressed in the loop \sphinxstyleemphasis{invariants}. The first one is true but not of
much use. The second one is the key to enabling Dafny to verify that
after the loop, \sphinxstyleemphasis{a == fact(n)}.

The invariant itself just says that at all points before and after the
loop body executes, that partial factorial value computed so far times
the factorial of \sphinxstyleemphasis{i} (which remains to be computed) is the answer that
we seek. Once the loop is done we (and Dafny) \sphinxstyleemphasis{also} know that \sphinxstyleemphasis{i == 0}.
It is the combination of the invariant and this fact that enables Dafny
to see that it must be the case that \sphinxstyleemphasis{a == fact(n)}.

We can verify by using asserts after the loop that our beliefs about
what the state of the program must be are correct. First, let’s have
Dafny check that the loop condition is now false.

\begin{sphinxVerbatim}[commandchars=\\\{\}]
assert !(i \PYGZgt{} 0);
\end{sphinxVerbatim}

We can also have Dafny check that our loop invariant still holds.

\begin{sphinxVerbatim}[commandchars=\\\{\}]
assert a * fact(i) == fact(n);
\end{sphinxVerbatim}

And now comes the most crucial step of all in our reasoning. We can
deduce that \sphinxstyleemphasis{a} now holds the correct answer. That this is so follows
from the conjunction of the two assertions we just made. First, that
\sphinxstyleemphasis{i} is not greater than \sphinxstyleemphasis{0} and given that its type is \sphinxstyleemphasis{nat}, the only
possible value it can have now is \sphinxstyleemphasis{0}. That’s what we’d expect, as it
is the condition on which the loop terminates (which it just did). But
better than just saying all of this, let us also formalize, document,
and check it using the Dafny verifier.

\begin{sphinxVerbatim}[commandchars=\\\{\}]
assert i == 0;
\end{sphinxVerbatim}

Now it’s easy to see. No matter what value \sphinxstyleemphasis{i} has, we know that the
loop invariant holds: \(a * fact(i) == fact(n),\) and we also know
that \sphinxstyleemphasis{i == 0}. So it must be that \(a * fact(0) == fact(n).\) And
fact(0) is \sphinxstyleemphasis{1} (from its mathematical definition). So it must be that
\sphinxstyleemphasis{a == fact(n)}. And Dafny confirms it!

\begin{sphinxVerbatim}[commandchars=\\\{\}]
assert a == fact(n);
\end{sphinxVerbatim}

We thus have the answer we need to return.  Dafny verifies that our
program satisfies its formal specification. We no longer have to
pray. We \sphinxstyleemphasis{know} that our program is right and Dafny confirms our
belief.

\begin{sphinxVerbatim}[commandchars=\\\{\}]
return a;
\end{sphinxVerbatim}

Mathematical logic is to software as the calculus is to physics and
engineering.  It’s not just an academic curiosity. It is a critical
intellectual tool, inceasingly used for precise specification and
semi-automated reasoning about and verification of real programs.


\section{Case Study: The Fibonacci Function}
\label{\detokenize{05-verifying-logical-specifications:case-study-the-fibonacci-function}}
Similarly, here is a verified imperative implementation of the
Fibonacci function. We start by adding a specification in the
form of an ensures clause, appealing to our functional program,
to tell Dafny what the imperative program must compute.

\begin{sphinxVerbatim}[commandchars=\\\{\}]
method fibonacci(n: nat) returns (r: nat)
    ensures r == fib(n)
\end{sphinxVerbatim}

Now for the body. First we represent values for the two cases where
the result requires no further computation.  Initially, \sphinxstyleemphasis{fib0} will
store the value of \sphinxstyleemphasis{fib(0)}, namely \sphinxstyleemphasis{0}, and \sphinxstyleemphasis{fib1} will store the
value of \sphinxstyleemphasis{fib(1)}, namely \sphinxstyleemphasis{1}.

\begin{sphinxVerbatim}[commandchars=\\\{\}]
var fib0, fib1 := 0, 1; //parallel assignment
\end{sphinxVerbatim}

Next, we test to see if either of these cases applies,
and if so we just return the appropriate result.

\begin{sphinxVerbatim}[commandchars=\\\{\}]
if (n == 0) \PYGZob{} return fib0; \PYGZcb{}
if (n == 1) \PYGZob{} return fib1; \PYGZcb{}
\end{sphinxVerbatim}

At this point, we know something more about the state of the program
than was the case when we started. We can deduce that \sphinxstyleemphasis{n} has to be
greater than or equal to \sphinxstyleemphasis{2}. This is because it initially had to be
greater than or equal to zero due to its type, and we would already
have returned if it were \sphinxstyleemphasis{0} or \sphinxstyleemphasis{1}. It must now be \sphinxstyleemphasis{2} or greater. We
can assert this proposition about the state of the program at this
point, and Dafny will verify it for us.

\begin{sphinxVerbatim}[commandchars=\\\{\}]
assert n \PYGZgt{}= 2;
\end{sphinxVerbatim}

So now we have to deal with the case where \sphinxstyleemphasis{n \textgreater{}= 2}. Our strategy for
computing \sphinxstyleemphasis{fib(n)} in this case is, again, to use a while loop. We
will establish a loop index \sphinxstyleemphasis{i}.  Our design will be based on the idea
that at the beginning and end of each loop iteration (we are currently
at the beginning), we will have already computed \sphinxstyleemphasis{fib(i)} and that its
value is in \sphinxstyleemphasis{fib1}. We’ve already assigned the value of \sphinxstyleemphasis{fib(0)} to
\sphinxstyleemphasis{fib0}, and \sphinxstyleemphasis{fib(1)} to \sphinxstyleemphasis{fib1}, so to set up the desired state, we
initialize \sphinxstyleemphasis{i} to be \sphinxstyleemphasis{1}.

\begin{sphinxVerbatim}[commandchars=\\\{\}]
var i := 1;
\end{sphinxVerbatim}

We can now assert and Dafny can verify a number of conditions that we
expect and require to hold. First, \sphinxstyleemphasis{fib1} equals \sphinxstyleemphasis{fib(i)}. To compute
the next (\sphinxstyleemphasis{i+1st}) Fibonacci number, we need not only the value of
\sphinxstyleemphasis{fib(i)} but also \sphinxstyleemphasis{fib(i-1)}. We will thus also want \sphinxstyleemphasis{fib0} to hold
this value at the start and end of each loop iteration. Indeed we do
have this state of affairs right now.

\begin{sphinxVerbatim}[commandchars=\\\{\}]
assert fib1 == fib(i);
assert fib0 == fib(i\PYGZhy{}1);
\end{sphinxVerbatim}

To compute \sphinxstyleemphasis{fib(n)} for any \sphinxstyleemphasis{n} greater than or equal to \sphinxstyleemphasis{2} will
require at least one execution of the loop body. We’ll thus set our
loop condition to be \sphinxstyleemphasis{i \textless{} n}. This ensures that the loop body will run
at least once, to compute \sphinxstyleemphasis{fib(2)}, as \sphinxstyleemphasis{i} is \sphinxstyleemphasis{1} and \sphinxstyleemphasis{n} is at least
\sphinxstyleemphasis{2}; so the loop condition \sphinxstyleemphasis{i \textless{} n} is \sphinxstyleemphasis{true}, which dictates that the
loop body must be evaluated at least once.

Within the loop body we’ll compute \sphinxstyleemphasis{fib(i+1)} (we call it \sphinxstyleemphasis{fib2}) by
adding together \sphinxstyleemphasis{fib0} and \sphinxstyleemphasis{fib1}; then we increment \sphinxstyleemphasis{i}; then we
update \sphinxstyleemphasis{fib0} and \sphinxstyleemphasis{fib1} so that for the \sphinxstyleemphasis{new} value of \sphinxstyleemphasis{i} they hold
\sphinxstyleemphasis{fib(i-1)} and \sphinxstyleemphasis{fib(i)}. To do this we assign the initial value of
\sphinxstyleemphasis{fib1} to \sphinxstyleemphasis{fib0} and the value of \sphinxstyleemphasis{fib2} to \sphinxstyleemphasis{fib1}. Think hard so as
to confirm for yourself that this sequence of actions re-establishes
the loop invariant.

Let’s work an example. Suppose \sphinxstyleemphasis{n} happens to be \sphinxstyleemphasis{2}. The loop body
will run, and after the one execution, \sphinxstyleemphasis{i} will have the value, \sphinxstyleemphasis{2};
\sphinxstyleemphasis{fib1} will have the value of \sphinxstyleemphasis{fib(2)}, and \sphinxstyleemphasis{fib0} will have the value
of \sphinxstyleemphasis{fib(1)}. Because \sphinxstyleemphasis{i} is now \sphinxstyleemphasis{2} and \sphinxstyleemphasis{n} is \sphinxstyleemphasis{2}, the loop condition
will now be false and the loop will terminate. The value of \sphinxstyleemphasis{fib1}
will of course be \sphinxstyleemphasis{fib(i)} but now we also have the negation of the
loop condition, i.e., \sphinxstyleemphasis{i == n}. So \sphinxstyleemphasis{fib(i)} will be \sphinxstyleemphasis{fib(n)}, which is
the result we want and that we return.

We can also informally prove to ourself that this strategy gives us a
program that always terminates and returns a value. That is, it does
not go into an infinite loop. To see this, note that the value of \sphinxstyleemphasis{i}
is initally less than or equal to \sphinxstyleemphasis{n}, and it increases by only \sphinxstyleemphasis{1} on
each time through the loop. The value of \sphinxstyleemphasis{n} is finite, so the value
of \sphinxstyleemphasis{i} will eventually equal the value of \sphinxstyleemphasis{n} at which point the loop
condition will be falsified and the looping will end.

What Dafny looks for to verify that a given loop terminates is a value
that \sphinxstyleemphasis{decreases} each time the loop runs and that is bounded below so
that it cannot decrease forever. As \sphinxstyleemphasis{i} increases in this loop, it can
not be the decreasing quantity. What Dafny takes instead is \sphinxstyleemphasis{n - i}.
When \sphinxstyleemphasis{i} is \sphinxstyleemphasis{0} this value is large, and as \sphinxstyleemphasis{i} gets closer to \sphinxstyleemphasis{n} it
decreases until when \sphinxstyleemphasis{i == n}, the difference is zero, and that is the
bound at which the loop terminates.

That’s our strategy. Here’s the while loop that we have designed. Now
for the first time, we see something crucial. In general, Dafny has no
idea how many times a loop body will execute. Intead, what it needs to
know are properties of the state of the program that hold no matter
how many times the loop executes, that, when combined with the fact
that the has terminated allows one to conclude that the loop does what
it’s meant to do. We call such properties \sphinxstyleemphasis{loop invariants}.

\begin{sphinxVerbatim}[commandchars=\\\{\}]
while (i \PYGZlt{} n)
    invariant i \PYGZlt{}= n;
    invariant fib0 == fib(i\PYGZhy{}1);
    invariant fib1 == fib(i);
\PYGZob{}
    var fib2 := fib0 + fib1;
    fib0 := fib1;
    fib1 := fib2;
    i := i + 1;
\PYGZcb{}
\end{sphinxVerbatim}

The invariants are just the conditions that we required to hold for
our design of the loop to work. First, \sphinxstyleemphasis{i} must never exceed \sphinxstyleemphasis{n}. If
it did, the loop would spin off into infinity. Second, to compute the
next (the \sphinxstyleemphasis{i+1st)} Fibonacci number we have to have the previous \sphinxstyleemphasis{two}
in memory. So \sphinxstyleemphasis{fib0} better hold \sphinxstyleemphasis{fib(i-1)} and \sphinxstyleemphasis{fib1}, \sphinxstyleemphasis{fib(i)}. Note
that these conditions do not have to hold at all times \sphinxstyleemphasis{within} the
execution of the loop body, where things are being updated, but they
do have to hold before before and after each execution.

The body of the loop is just as we described it above. We can use our
minds to deduce that if the invariants hold before each loop body runs
(and they do), then they will also hold after it runs. We can also see
that after the loop terminates, it must be that \sphinxstyleemphasis{i==n}, because we
know that it’s always true that \sphinxstyleemphasis{i \textless{}= n} and the loop condition must
now be false, which is to say that \sphinxstyleemphasis{i} can no longer be strictly less
than \sphinxstyleemphasis{n}, so \sphinxstyleemphasis{i} must now equal \sphinxstyleemphasis{n}. Logic says so.

What is amazing is that we can write these assertions in Dafny if we
wish to, and Dafny will verify that they are true statements about the
state of the program after the loop has run. We have \sphinxstyleemphasis{proved} (or
rather Dafny has proved) that our loop always terminates with the
right answer. We have a formal proof of \sphinxstyleemphasis{total correctness} for this
program.

\begin{sphinxVerbatim}[commandchars=\\\{\}]
assert i \PYGZlt{}= n;      // invariant
assert !(i \PYGZlt{} n);    // loop condition is false
assert (i \PYGZlt{}= n) \PYGZam{}\PYGZam{} !(i \PYGZlt{} n) ==\PYGZgt{} (i == n);
assert i == n;      // deductive conclusion
assert fib1 == fib(i); // invariant
assert fib1 == fib(i) \PYGZam{}\PYGZam{} (i==n) ==\PYGZgt{} fib1 == fib(n);
assert fib1 == fib(n);
return fib1;
\end{sphinxVerbatim}


\section{What is Dafny, Again?}
\label{\detokenize{05-verifying-logical-specifications:what-is-dafny-again}}
Dafny is a cutting-edge software language and tooset for verification
of imperative code. It was developed at Microsoft Research—one of
the top computer science research labs in the world. We are exploring
Dafny and the ideas underlying it in the first part of this course to
give a sense of why it’s vital for a computer scientist today to have
a substantial understanding of logic and proofs along with the ability
to \sphinxstyleemphasis{code}.

Tools such as TLA+, Dafny, and others of this variety give us a way
both to express formal specifications and imperative code in a unified
way (albeit in different sub-languages), and to have some automated
checking done in an attempt to verify that code satisfies its spec.

We say \sphinxstyleemphasis{attempt} here, because in general verifying the consistency of
code and a specification is a literally unsolvable problem. In cases
that arise in practice, much can often be done. It’s not always easy,
but if one requires ultra-high assurance of the consistency of code
and specification, then there is no choice but to employ the kinds of
\sphinxstyleemphasis{formal methods} introduced here.

To understand how to use such state-of-the-art software development
tools and methods, one must understand not only the language of code,
but also the languages of mathematical logic, including set and type
theory. One must also understand precisely what it means to \sphinxstyleemphasis{prove}
that a program satisfies its specification. And for that, one must
develop a sense for propositions and proofs: what they are and how
they are built and evaluated.

The well educated computer scientist and the professional software
engineer must understand logic and proofs as well as coding, and how
they work together to help build \sphinxstyleemphasis{trustworthy} systems. Herein lies
the deep relevance of logic and proofs, which might otherwise seem
like little more than abstract nonsense and a distraction from the
task of learning how to program.


\chapter{6. Dafny Language: Types, Statements, Expressions}
\label{\detokenize{06-dafny-language::doc}}\label{\detokenize{06-dafny-language:dafny-language-types-statements-expressions}}

\section{Built-In Types}
\label{\detokenize{06-dafny-language:built-in-types}}
Dafny natively supports a range of abstract data types akin to those
found in widely used, industrial imperative programming languages and
systems, such as Python and Java. In this chapter, we introduce and
briefly illustrate the use of these types. The types we discuss are
as follow:
\begin{itemize}
\item {} 
bool, supporting Boolean algebra

\item {} 
int, nat, and real types, supporting \sphinxstyleemphasis{exact} arithmetic (unlike
the numerical types found in most industrial languages

\item {} 
char, supporting character types

\item {} 
set\textless{}T\textgreater{} and iset\textless{}T\textgreater{}, polymorphic set theory for finite and infinite sets

\item {} 
seq\textless{}T\textgreater{} and iseq\textless{}T\textgreater{}, polymorphic finite and infinite sequences

\item {} 
string, supporting character sequences (with addtional helpful functions)

\item {} 
map\textless{}K,V\textgreater{} and imap\textless{}K,V\textgreater{}, polymorphic finite and infinite partial functions

\item {} 
array\textless{}T\textgreater{}, polymorphic 1- and multi-dimensional arrays

\end{itemize}


\subsection{Booleans}
\label{\detokenize{06-dafny-language:booleans}}
The bool abstract data type (ADT) in Dafny provides a bool data type
with values, \sphinxstyleemphasis{true} and \sphinxstyleemphasis{false}, along with the Boolean operators that
are supported by most programming langauges, along with a few that are
not commonly supported.

Here’s a method that computes nothing useful and returns no values,
but that illustrates the range of Boolean operators in Dafny. We also
use the examples in this chapter to discuss a few other aspects of the
Dafny language.

\begin{sphinxVerbatim}[commandchars=\\\{\}]
method BoolOps(a: bool) returns (r: bool)  // bool \PYGZhy{}\PYGZgt{} bool
\PYGZob{}
    var t: bool := true;    // explicit type declaration
    var f := false;         // type inferred automatically
    var not := !t;          // negation
    var conj := t \PYGZam{}\PYGZam{} f;     // conjunction, short\PYGZhy{}circuit evaluation
    var disj := t \textbar{}\textbar{} f;     // disjunction, short\PYGZhy{}circuit (sc) evaluation
    var impl := t ==\PYGZgt{} f;    // implication, right associative, sc from left
    var foll := t \PYGZlt{}== f;    // follows, left associative, sc from right
    var equv := t \PYGZlt{}==\PYGZgt{} t;   // iff, bi\PYGZhy{}implication
    return true;            // returning a Boolean value
 \PYGZcb{}
\end{sphinxVerbatim}


\subsection{Numbers}
\label{\detokenize{06-dafny-language:numbers}}
Methods aren’t required to return results. Such methods do their jobs
by having side effects, e.g., doing output or writing data into global
variables (usually a bad idea).  Here’s a method that doesn’t return a
value. It illustrates numerical types, syntax, and operations.

\begin{sphinxVerbatim}[commandchars=\\\{\}]
method NumOps()
\PYGZob{}
    var r1: real := 1000000.0;
    var i1: int := 1000000;
    var i2: int := 1\PYGZus{}000\PYGZus{}000;   // underscores for readiability
    var i3 := 1\PYGZus{}000;            // Dafny can often infer types
    var b1 := (10 \PYGZlt{} 20) \PYGZam{}\PYGZam{} (20 \PYGZlt{}= 30); // a boolean expression
    var b2 := 10 \PYGZlt{} 20 \PYGZlt{}= 30;    // equivalent, with \PYGZdq{}chaining\PYGZdq{}
    var i4: int := (5.5).Floor; // 5
    var i5 := (\PYGZhy{}2.5).Floor;     // \PYGZhy{}3
    var i6 := \PYGZhy{}2.5.Floor;        // \PYGZhy{}2 = \PYGZhy{}(2.5.Floor); binding!
\PYGZcb{}
\end{sphinxVerbatim}


\subsection{Characters}
\label{\detokenize{06-dafny-language:characters}}
Characters (char) are handled sort of as they are in C, etc.

\begin{sphinxVerbatim}[commandchars=\\\{\}]
method CharFun()
\PYGZob{}
    var c1: char := \PYGZsq{}a\PYGZsq{};
    var c2 := \PYGZsq{}b\PYGZsq{};
    // var i1 := c2 \PYGZhy{} c1;
    var i1 := (c2 as int) \PYGZhy{} (c1 as int);    // type conversion
    var b1 := c1 \PYGZlt{} c2;  // ordering operators defined for char
    var c3 := \PYGZsq{}\PYGZbs{}n\PYGZsq{};     // c\PYGZhy{}style escape for non\PYGZhy{}printing chars
    var c4 := \PYGZsq{}\PYGZbs{}u265B\PYGZsq{}; // unicode, hex, \PYGZdq{}chess king\PYGZdq{} character
\PYGZcb{}
\end{sphinxVerbatim}


\subsection{Sets}
\label{\detokenize{06-dafny-language:sets}}
Polymorphic finite and infinite set types:
set\textless{}T\textgreater{} and iset\textless{}T\textgreater{}. T must support equality.
Values of these types are immutable.

\begin{sphinxVerbatim}[commandchars=\\\{\}]
method SetPlay()
\PYGZob{}
    var empty: set\PYGZlt{}int\PYGZgt{} := \PYGZob{}\PYGZcb{};
    var primes := \PYGZob{}2, 3, 5, 7, 11\PYGZcb{};
    var squares := \PYGZob{}1, 4, 9, 16, 25\PYGZcb{};
    var b1 := empty \PYGZlt{} primes;    // strict subset
    var b2 := primes \PYGZlt{}= primes;   // subset
    var b3: bool := primes !! squares; // disjoint
    var union := primes + squares;
    var intersection := primes * squares;
    var difference := primes \PYGZhy{} \PYGZob{}3, 5\PYGZcb{};
    var b4 := primes == squares;    // false
    var i1 := \textbar{} primes \textbar{};   // cardinality (5)
    var b5 := 4 in primes;  // membership (false)
    var b6 := 4 !in primes; // non\PYGZhy{}membership
\PYGZcb{}
\end{sphinxVerbatim}


\subsection{Sequences}
\label{\detokenize{06-dafny-language:sequences}}
Polymorphic sequences (often called “lists”): seq\textless{}T\textgreater{}. These can be
understood as functions from indices to values. Some of the operations
require that T support equality. Values of this type are immutable.

\begin{sphinxVerbatim}[commandchars=\\\{\}]
method SequencePlay()
\PYGZob{}
    var empty\PYGZus{}seq: seq\PYGZlt{}char\PYGZgt{} := [];
    var hi\PYGZus{}seq: seq\PYGZlt{}char\PYGZgt{} := [\PYGZsq{}h\PYGZsq{}, \PYGZsq{}i\PYGZsq{}];
    var b1 := hi\PYGZus{}seq == empty\PYGZus{}seq; // equality; !=
    var hchar := hi\PYGZus{}seq[0];        // indexing
    var b2 := [\PYGZsq{}h\PYGZsq{}] \PYGZlt{} hi\PYGZus{}seq;   // proper prefix
    var b3 := hi\PYGZus{}seq \PYGZlt{} hi\PYGZus{}seq;  // this is false
    var b4 := hi\PYGZus{}seq \PYGZlt{}= hi\PYGZus{}seq; // prefix, true
    var sum := hi\PYGZus{}seq + hi\PYGZus{}seq; // concatenation
    var len := \textbar{} hi\PYGZus{}seq \textbar{};
    var Hi\PYGZus{}seq := hi\PYGZus{}seq[0 := \PYGZsq{}H\PYGZsq{}]; // update
    var b5 := \PYGZsq{}h\PYGZsq{} in hi\PYGZus{}seq; // member, true, !in
    var s := [0,1,2,3,4,5];
    var s1 := s[0..2];  // subseqence
    var s2 := s[1..];   // \PYGZdq{}drop\PYGZdq{} prefix of len 1
    var s3 := s[..2];   // \PYGZdq{}take\PYGZdq{} prefix of len 2
    // there\PYGZsq{}s a slice operator, too; later
 \PYGZcb{}
\end{sphinxVerbatim}


\subsection{Strings}
\label{\detokenize{06-dafny-language:strings}}
Dafny has strings. Strings are literally just sequences of characters
(of type seq\textless{}char\textgreater{}), so you can use all the sequence operations on
strings.  Dafny provides additional helpful syntax for strings.

\begin{sphinxVerbatim}[commandchars=\\\{\}]
method StringPlay()
 \PYGZob{}
     var s1: string := \PYGZdq{}Hello CS2102!\PYGZdq{};
     var s2 := \PYGZdq{}Hello CS2102!\PYGZbs{}n\PYGZdq{};   // return
     var s3 := \PYGZdq{}\PYGZbs{}\PYGZdq{}Hello CS2102!\PYGZbs{}\PYGZdq{}\PYGZdq{}; // quotes
 \PYGZcb{}
\end{sphinxVerbatim}


\subsection{Maps (Partial Functions)}
\label{\detokenize{06-dafny-language:maps-partial-functions}}
Dafny also supports polymorphic maps, both finite (map\textless{}K,V\textgreater{}) and
infinite (imap\textless{}K,V\textgreater{}).  The key type, K, must support equality (==).
In mathematical terms, a map really represents a binary relation,
i.e., a set of \textless{}K,V\textgreater{} pairs, which is to say a subset of the product
set, K * V, where we view the types K and V as defining sets of
values.

\begin{sphinxVerbatim}[commandchars=\\\{\}]
method MapPlay()
\PYGZob{}
    // A map literal is keyword map + a list of maplets.
    // A maplet is just a single \PYGZlt{}K,V\PYGZgt{} pair (or \PYGZdq{}tuple\PYGZdq{}).
    // Here\PYGZsq{}s an empty map from strings to ints
    var emptyMap: map\PYGZlt{}string,int\PYGZgt{} := map[];

    // Here\PYGZsq{}s non empty map from strings to ints
    // A maplet is \PYGZdq{}k := v,\PYGZdq{} k and v being of types K and V
    var aMap: map\PYGZlt{}string,int\PYGZgt{}  := map[\PYGZdq{}Hi\PYGZdq{} := 1, \PYGZdq{}There\PYGZdq{} := 2];

    // Map domain (key) membership
    var isIn: bool := \PYGZdq{}There\PYGZdq{} in aMap; // true
    var isntIn := \PYGZdq{}Their\PYGZdq{} !in aMap;    // true

    // Finite map cardinality (number of maplets in a map)
    var card := \textbar{}aMap\textbar{};

    //Map lookup
    var image1 := aMap[\PYGZdq{}There\PYGZdq{}];
    // var image2 := aMap[\PYGZdq{}Their\PYGZdq{}]; // error! some kind of magic
    var image2: int;
    if (\PYGZdq{}Their\PYGZdq{} in aMap) \PYGZob{} image2 := aMap[\PYGZdq{}Their\PYGZdq{}]; \PYGZcb{}

    // map update, maplet override and maplet addition
    aMap := aMap[\PYGZdq{}There\PYGZdq{} := 3];
    aMap := aMap[\PYGZdq{}Their\PYGZdq{} := 10];
\PYGZcb{}
\end{sphinxVerbatim}


\subsection{Arrays}
\label{\detokenize{06-dafny-language:arrays}}
Dafny supports arrays. Here’s we’ll see simple 1-d arrays.

\begin{sphinxVerbatim}[commandchars=\\\{\}]
method ArrayPlay()
\PYGZob{}
    var a := new int[10]; // in general: a: array\PYGZlt{}T\PYGZgt{} := new T[n];
    var a\PYGZsq{} := new int[10];   // type inference naturally works here
    var i1 := a.Length;      // Immutable \PYGZdq{}Length\PYGZdq{} member holds length of array
    a[3] := 3;           // array update
    var i2 := a[3];          // array access
    var seq1 := a[3..8];    // take first 8, drop first 3, return as sequence
    var b := 3 in seq1;     // true! (see sequence operations)
    var seq2 := a[..8];     // take first 8, return rest as sequence
    var seq3 := a[3..];     // drop first 3, return rest as sequence
    var seq4 := a[..];      // return entire array as a sequence
\PYGZcb{}
\end{sphinxVerbatim}

Arrays, objects (class instances), and traits (to be discussed) are of
“reference” types, which is to say, values of these types are stored
on the heap. Values of other types, including sets and sequences, are
of “value types,” which is to say values of these types are stored on
the stack; and they’re thus always treated as “local” variables. They
are passed by value, not reference, when passed as arguments to
functions and methods. Value types include the basic scalar types
(bool, char, nat, int, real), built-in collection types (set,
multiset, seq, string, map, imap), tuple, inductive, and co-inductive
types (to be discussed).  Reference type values are allocated
dynamically on the heap, are passed by reference, and therefore can be
“side effected” (mofified) by methods to which they are passed.


\section{Statements}
\label{\detokenize{06-dafny-language:statements}}

\subsection{Block}
\label{\detokenize{06-dafny-language:block}}
In Dafny, you can make one bigger command from a sequence of smaller
ones by enclosing the sequence in braces. You typically use this only
for the bodies of loops and the parts of conditionals.

\begin{sphinxVerbatim}[commandchars=\\\{\}]
\PYGZob{}
    print \PYGZdq{}Block: Command1\PYGZbs{}n\PYGZdq{};
    print \PYGZdq{}Block: Command2\PYGZbs{}n\PYGZdq{};
\PYGZcb{}
\end{sphinxVerbatim}


\subsection{Break}
\label{\detokenize{06-dafny-language:break}}
The break command is for prematurely breaking out of loops.

\begin{sphinxVerbatim}[commandchars=\\\{\}]
var i := 5;
while (i \PYGZgt{} 0)
\PYGZob{}
    if (i == 3)
    \PYGZob{}
        break;
    \PYGZcb{}
    i := i \PYGZhy{} 1;
\PYGZcb{}
print \PYGZdq{}Break: Broke when i was \PYGZdq{}, i, \PYGZdq{}\PYGZbs{}n\PYGZdq{};
\end{sphinxVerbatim}


\subsection{Update (Assignment)}
\label{\detokenize{06-dafny-language:update-assignment}}
There are several forms of the update command. The first is the usual
assignment that you see in many languages. The second is “multiple
assignment”, where you can assign several values to several variables
at once. The final version is not so familar. It \sphinxstyleemphasis{chooses} a value
that satisfies some property and assigns it to a variable.

\begin{sphinxVerbatim}[commandchars=\\\{\}]
var x := 3;         // typical assignment
var y := 4;         // typical assignment
print \PYGZdq{}Update: before swap, x and y are \PYGZdq{}, x, \PYGZdq{}, \PYGZdq{}, y, \PYGZdq{}\PYGZbs{}n\PYGZdq{};
x, y := y, x;       // one\PYGZhy{}line swap using multiple assignment
print \PYGZdq{}Update: after swap, x and y are \PYGZdq{}, x, \PYGZdq{}, \PYGZdq{}, y, \PYGZdq{}\PYGZbs{}n\PYGZdq{};
var s: set\PYGZlt{}int\PYGZgt{} := \PYGZob{} 1, 2, 3 \PYGZcb{}; // typical: assign set value to s
var c :\textbar{} c in s;    // update c to a value such that c is in s
print \PYGZdq{}Update: Dafny chose this value from the set: \PYGZdq{}, c, \PYGZdq{}\PYGZbs{}n\PYGZdq{};
\end{sphinxVerbatim}


\subsection{Var (variable declaration)}
\label{\detokenize{06-dafny-language:var-variable-declaration}}
A variable declaration stsatement is used to declare one or more local
variables in a method or function. The type of each local variable
must be given unless the variable is given an initial value in which
case the type will be inferred. If initial values are given, the
number of values must match the number of variables declared. Note
that the type of each variable must be given individually. This “var
x, y : int;” does not declare both x and y to be of type int. Rather
it will give an error explaining that the type of x is underspecified.

\begin{sphinxVerbatim}[commandchars=\\\{\}]
var l: seq\PYGZlt{}int\PYGZgt{} := [1, 2, 3]; // explicit type (sequence of its)
var l\PYGZsq{}          := [1, 2, 3]; // Dafny infers type from [1, 2, 3]
\end{sphinxVerbatim}


\subsection{If (conditional)}
\label{\detokenize{06-dafny-language:if-conditional}}
There are several forms of the if statement in Dafny.  The first is
“if (Boolean) block-statement.” The second is “if (Boolean)
block-statement else block-statement” A block is a sequence of
commands enclosed by braces (see above).

In addition, there is a multi-way if statement similar to a case
statement in C or C++. The conditions for the cases are evaluated in
an unspecified order. The first to match results in evaluation of the
corresponding command. If no case matches the overall if command does
nothing.

\begin{sphinxVerbatim}[commandchars=\\\{\}]
if (0==0) \PYGZob{} print \PYGZdq{}If: zero is zero\PYGZbs{}n\PYGZdq{}; \PYGZcb{}   // if (bool) \PYGZob{}block\PYGZcb{}
if (0==1)
    \PYGZob{} print \PYGZdq{}If: oops!\PYGZbs{}n\PYGZdq{}; \PYGZcb{}
else
    \PYGZob{} print \PYGZdq{}If: oh good, 0 != 1\PYGZbs{}n\PYGZdq{}; \PYGZcb{}

var q := 1;
if \PYGZob{}
    case q == 0 =\PYGZgt{} print \PYGZdq{}Case: q is 0\PYGZbs{}n\PYGZdq{};
    case q == 1 =\PYGZgt{} print \PYGZdq{}Case: q is 1\PYGZbs{}n\PYGZdq{};
    case q == 2 =\PYGZgt{} print \PYGZdq{}Case: q is 2\PYGZbs{}n\PYGZdq{};
\PYGZcb{}
\end{sphinxVerbatim}


\subsection{While (iteration)}
\label{\detokenize{06-dafny-language:while-iteration}}
While statements come in two forms. The first is a typical Python-like
statement “while (Boolean) block-command”. The second involves the use
of a case-like construct instead of a single Boolean expression to
control the loop. This form is typically used when a loop has to
either run up or down depending on the initial value of the index. An
example of the first form is given above, for the BREAK
statement. Here is an example of the second form.

\begin{sphinxVerbatim}[commandchars=\\\{\}]
var r: int;
while
    decreases if 0 \PYGZlt{}= r then r else \PYGZhy{}r;
\PYGZob{}
    case r \PYGZlt{} 0 =\PYGZgt{} \PYGZob{} r := r + 1; \PYGZcb{}
    case 0 \PYGZlt{} r =\PYGZgt{} \PYGZob{} r := r \PYGZhy{} 1; \PYGZcb{}
\PYGZcb{}
\end{sphinxVerbatim}

Dafny insists on proving that all while loops and all recursive
functions actually terminate \textendash{} do not loop forever. Proving such
properties is (infinitely) hard in general. Dafny often makes good
guesses as to how to do it, in which case one need do nothing more. In
many other cases, however, Dafny needs some help. For this, one writes
“loop specifications.” These include clauses called “decreases”,
“invariant”, and “modifies”, which are written after the while and
before the left brace of the loop body. We discuss these separately,
but in the meantime, here are a few examples.

\begin{sphinxVerbatim}[commandchars=\\\{\}]
// a loop that counts down from 5, terminating when i==0.
i := 5;                 // already declared as int above
while 0 \PYGZlt{} i
    invariant 0 \PYGZlt{}= i    // i always \PYGZgt{}= 0 before and after loop
    decreases i         // decreasing value of i bounds the loop
\PYGZob{}
    i := i \PYGZhy{} 1;
\PYGZcb{}

// this loop counts *up* from i=0 ending with i==5
// notice that what decreases is difference between i and n
var n := 5;
i := 0;
while i \PYGZlt{} n
    invariant 0 \PYGZlt{}= i \PYGZlt{}= n
    decreases n \PYGZhy{} i
\PYGZob{}
    i := i + 1;
\PYGZcb{}
\end{sphinxVerbatim}


\subsection{Assert (assert a proposition about the state of the program)}
\label{\detokenize{06-dafny-language:assert-assert-a-proposition-about-the-state-of-the-program}}
Assert statements are used to express logical proposition that are
expected to be true. Dafny will attempt to prove that the assertion is
true and give an error if not. Once it has proved the assertion it can
then use its truth to aid in following deductions. Thus if Dafny is
having a difficult time verifying a method the user may help by
inserting assertions that Dafny can prove, and whose true may aid in
the larger verification effort.  (From reference manual.)

\begin{sphinxVerbatim}[commandchars=\\\{\}]
assert i == 5;      // true because of preceding loop
assert !(i == 4);   // similarly true
// assert i == 4;   // uncomment to see static assertion failure
\end{sphinxVerbatim}


\subsection{Print (produce output on console)}
\label{\detokenize{06-dafny-language:print-produce-output-on-console}}
From reference manual: The print statement is used to print the values
of a comma-separated list of expressions to the console.  The
generated C\# code uses the System.Object.ToString() method to convert
the values to printable strings. The expressions may of course include
strings that are used for captions. There is no implicit new line
added, so to get a new line you should include “n” as part of one of
the expressions. Dafny automatically creates overrides for the
ToString() method for Dafny data types.

\begin{sphinxVerbatim}[commandchars=\\\{\}]
print \PYGZdq{}Print: The set is \PYGZdq{}, \PYGZob{} 1, 2, 3\PYGZcb{}, \PYGZdq{}\PYGZbs{}n\PYGZdq{}; // print the set
\end{sphinxVerbatim}


\subsection{Return}
\label{\detokenize{06-dafny-language:return}}
From the reference manual: A return statement can only be used in a
method. It terminates the execution of the method. To return a value
from a method, the value is assigned to one of the named return values
before a return statement. The return values act very much like local
variables, and can be assigned to more than once. Return statements
are used when one wants to return before reaching the end of the body
block of the method.  Return statements can be just the return keyword
(where the current value of the out parameters are used), or they can
take a list of values to return. If a list is given the number of
values given must be the same as the number of named return values.

To return a value from a method, assign to the return parameter
and then either use an explicit return statement or just let the
method complete.

\begin{sphinxVerbatim}[commandchars=\\\{\}]
method ReturnExample() returns (retval: int)
\PYGZob{}
    retval := 10;
    // implicit return here
\PYGZcb{}
\end{sphinxVerbatim}

Methods can return multiple values.

\begin{sphinxVerbatim}[commandchars=\\\{\}]
method ReturnExample2() returns (x: int, y:int)
\PYGZob{}
    x := 10;
    y := 20;
\PYGZcb{}
\end{sphinxVerbatim}

The return keyword can be used to return immediatey

\begin{sphinxVerbatim}[commandchars=\\\{\}]
method ReturnExample3() returns (x: int)
\PYGZob{}
    x := 5;     // don\PYGZsq{}t \PYGZdq{}var\PYGZdq{} decare return variable
    return;     // return immediately
    x := 6;     // never gets executed
    assert 0 == 1; // can\PYGZsq{}t be reached to never gets checked!
\PYGZcb{}
\end{sphinxVerbatim}


\section{Expressions}
\label{\detokenize{06-dafny-language:expressions}}

\subsection{Literals Expressions}
\label{\detokenize{06-dafny-language:literals-expressions}}
A literal expression is a boolean literal (true or false), a null
object reference (null), an unsigned integer (e.g., 3) or real (e.g.,
3.0) literal, a character (e.g., ‘a’) or string literal (e.g., “abc”),
or “this” which denote the current object in the context of an
instance method or function. We have not yet seen objects or talked
about instance methods or functions.


\subsection{If (Conditional) Expressions}
\label{\detokenize{06-dafny-language:if-conditional-expressions}}
If expressions first evaluate a Boolean expression and then evaluate
one of the two following expressions, the first if the Boolean
expression was true, otherwise the second one.  Notice in this example
that an IF \sphinxstyleemphasis{expression} is used on the right side of an
update/assignment statement. There is also an if \sphinxstyleemphasis{statement}.

\begin{sphinxVerbatim}[commandchars=\\\{\}]
var x := 11;
var h := if x != 0 then (10 / x) else 1;    // if expression
assert h == 0;
if (h == 0) \PYGZob{}x := 3; \PYGZcb{} else \PYGZob{} x := 0; \PYGZcb{}     // if statement
assert x == 3;
\end{sphinxVerbatim}


\subsection{Conjunction and Disjunction Expressions}
\label{\detokenize{06-dafny-language:conjunction-and-disjunction-expressions}}
Conjunction and disjuction are associative. This means that no matter
what b1, b2, and b3 are, (b1 \&\& b2) \&\& b3 is equal to (b1 \&\& (b2 \&\&
b3)), The same property holds for \textbar{}\textbar{}.

These operators are also \sphinxstyleemphasis{short circuiting}. What this means is that
their second argument is evaluated only if evaluating the first does
not by itself determine the value of the expression.

Here’s an example where short circuit evaluation matters. It is what
prevents the evaluation of an undefined expressions after the \&\&
operator.

\begin{sphinxVerbatim}[commandchars=\\\{\}]
var a: array\PYGZlt{}int\PYGZgt{} := null;
var b1: bool := (a != null) \PYGZam{}\PYGZam{} (a[0]==1);
\end{sphinxVerbatim}

Here short circuit evaluation protects against evaluation of a{[}0{]} when
a is null. Rather than evaluating both expressions, reducing them both
to Boolean values, and then applying a Boolean \sphinxstyleemphasis{and} function, instead
the right hand expressions is evaluated “lazily”, i.e., only of the
one on the left doesn’t by itself determine what the result should
be. In this case, because the left hand expression is false, the whole
expression must be false, so the right side not only doesn’t have to
be evaluated; it also \sphinxstyleemphasis{won’t} be evaluated.


\subsection{Sequence, Set, Multiset, and Map Expressions}
\label{\detokenize{06-dafny-language:sequence-set-multiset-and-map-expressions}}
Values of these types can be written using so-called \sphinxstyleemphasis{display}
expressions. Sequences are written as lists of values within square
brackets; sets, within braces; and multisets using “multiset” followed
by a list of values within braces.

\begin{sphinxVerbatim}[commandchars=\\\{\}]
var aSeq: seq\PYGZlt{}int\PYGZgt{} := [1, 2, 3];
var aVal := aSeq[1];    // get the value at index 1
assert aVal == 2;       // don\PYGZsq{}t forget about zero base indexing

var aSet: set\PYGZlt{}int\PYGZgt{} := \PYGZob{} 1, 2, 3\PYGZcb{};   // sets are unordered
assert \PYGZob{} 1, 2, 3 \PYGZcb{} == \PYGZob{} 3, 1, 2\PYGZcb{};   // set equality ignores order
assert [ 1, 2, 3 ] != [ 3, 1, 2];   // sequence equality doesn\PYGZsq{}t

var mSet := multiset\PYGZob{}1, 2, 2, 3, 3, 3\PYGZcb{};
assert (3 in mSet) == true;         // in\PYGZhy{}membership is Boolean
assert mSet[3] == 3;                // [] counts occurrences
assert mSet[4] == 0;

var sqr := map [0 := 0, 1 := 1, 2 := 4, 3 := 9, 4 := 16];
assert \textbar{}sqr\textbar{} == 5;
assert sqr[2] == 4;
\end{sphinxVerbatim}


\subsection{Relational Expressions}
\label{\detokenize{06-dafny-language:relational-expressions}}
Relation expressions, such as less than, have a relational operator
that compares two or more terms and returns a Boolean result. The ==,
!=, \textless{}, \textgreater{}, \textless{}=, and \textgreater{}= operators are examples. These operators are also
“chaining”. That means one can write expressions such as 0 \textless{}= x \textless{} n,
and what this means is 0 \textless{}= x \&\& x \textless{} n.

The in and !in relational operators apply to collection types. They
compute membership or non-membership respectively.

The !! operator computes disjointness of sets and multisets. Two such
collections are said to be disjoint if they have no elements in
common. Here are a few examples of relational expressions involving
collections (all given within assert statements).

\begin{sphinxVerbatim}[commandchars=\\\{\}]
assert 3 in \PYGZob{} 1, 2, 3 \PYGZcb{};                            // set member
assert 4 !in \PYGZob{} 1, 2, 3 \PYGZcb{};                           // non\PYGZhy{}member
assert \PYGZdq{}foo\PYGZdq{} in [\PYGZdq{}foo\PYGZdq{}, \PYGZdq{}bar\PYGZdq{}, \PYGZdq{}bar\PYGZdq{}];              // seq member
assert \PYGZdq{}foo\PYGZdq{} in \PYGZob{} \PYGZdq{}foo\PYGZdq{}, \PYGZdq{}bar\PYGZdq{}\PYGZcb{};                    // set member
assert \PYGZob{} \PYGZdq{}foo\PYGZdq{}, \PYGZdq{}bar\PYGZdq{} \PYGZcb{} !! \PYGZob{} \PYGZdq{}baz\PYGZdq{}, \PYGZdq{}bif\PYGZdq{}\PYGZcb{};         // disjoint
assert \PYGZob{} \PYGZdq{}foo\PYGZdq{}, \PYGZdq{}bar\PYGZdq{} \PYGZcb{} \PYGZlt{} \PYGZob{} \PYGZdq{}foo\PYGZdq{}, \PYGZdq{}bar\PYGZdq{}, \PYGZdq{}baz\PYGZdq{} \PYGZcb{};  // subset
assert \PYGZob{} \PYGZdq{}foo\PYGZdq{}, \PYGZdq{}bar\PYGZdq{} \PYGZcb{} == \PYGZob{} \PYGZdq{}foo\PYGZdq{}, \PYGZdq{}bar\PYGZdq{} \PYGZcb{};        // set equals
\end{sphinxVerbatim}


\subsection{Array Allocation Expressions}
\label{\detokenize{06-dafny-language:array-allocation-expressions}}
Arrays in Dafny are \sphinxstyleemphasis{reference values}. That is, the value
of an array variable is a \sphinxstyleemphasis{reference} to an address in the
\sphinxstyleemphasis{heap} part of memory, or it is \sphinxstyleemphasis{null}. To get at the data
in an array, one \sphinxstyleemphasis{dereferences} the array variable, using
the \sphinxstyleemphasis{subscripting} operator. The array variable must not be
null in this case. It must reference a chunk of memory that
has been allocated for the array values, in the \sphinxstyleemphasis{heap} part
of memory.

To allocate memory for a new array for n elements of type T one
uses an expression like this: a: array\textless{}T\textgreater{} := new T{[}n{]}. The type
of \sphinxstyleemphasis{a} here is “an array of elements of type \sphinxstyleemphasis{T},” and the size
of the allocated memory chunk is big enough to hold \sphinxstyleemphasis{n} values
of this type.

Multi-dimensional arrays (matrices) are also supported. The types of
these arrays are “arrayn\textless{}T\textgreater{}, where “n” is the number of dimensions and
T is the type of the elements. All elements of an array or matrix must
be of the same type.

\begin{sphinxVerbatim}[commandchars=\\\{\}]
a := new int[10];       // type of a already declared above
var m: array2\PYGZlt{}int\PYGZgt{} := new int[10, 10];
a[0] := 1;              // indexing into 1\PYGZhy{}d array
m[0,0] := 1;            // indexing into multi\PYGZhy{}dimensional array
\end{sphinxVerbatim}


\subsection{Old Expressions}
\label{\detokenize{06-dafny-language:old-expressions}}
An old expression is used in postconditions. old(e) evaluates to the
value expression e had on entry to the current method.  Here’s an
example showing the use of the old expression.  This method increments
(adds one {\color{red}\bfseries{}to\_} the first element of an array.  The specification part
of the method \sphinxstyleemphasis{ensures} that the method body has this effect by
explaining that the new value of a{[}0{]} must be the original (the “old”)
value plus one. The \sphinxstyleemphasis{requires} (preconditions) statements are needed
to ensure that the array is not null and not zero length. The modifies
command explains that the method body is allowed to change the value
of a.

\begin{sphinxVerbatim}[commandchars=\\\{\}]
method incr(a: array\PYGZlt{}nat\PYGZgt{}) returns (r: array\PYGZlt{}nat\PYGZgt{})
requires a != null;
requires a.Length \PYGZgt{} 0;
modifies a;
ensures a[0] == old(a[0]) + 1;
\PYGZob{}
    a[0] := a[0] + 1;
    return a;
\PYGZcb{}
\end{sphinxVerbatim}


\subsection{Cardinality Expressions}
\label{\detokenize{06-dafny-language:cardinality-expressions}}
For a collection expression c, {\color{red}\bfseries{}\textbar{}c\textbar{}} is the cardinality of c. For a set
or sequence the cardinality is the number of elements. For a multiset
the cardinality is the sum of the multiplicities of the elements. For
a map the cardinality is the cardinality of the domain of the
map. Cardinality is not defined for infinite maps.

\begin{sphinxVerbatim}[commandchars=\\\{\}]
var c1 := \textbar{} [1, 2, 3] \textbar{};            // cardinality of sequence
assert c1 == 3;
var c2 := \textbar{} \PYGZob{} 1, 2, 3 \PYGZcb{} \textbar{};          // cardinality of a set
assert c2 == 3;
var c3 := \textbar{} map[ 0 := 0, 1 := 1, 2 := 4, 3 := 9] \textbar{}; // of a map
assert c3 == 4;
assert \textbar{} multiset\PYGZob{} 1, 2, 2, 3, 3, 3, 4, 4, 4, 4 \PYGZcb{} \textbar{} == 10; // multiset
\end{sphinxVerbatim}


\subsection{Let Expressions}
\label{\detokenize{06-dafny-language:let-expressions}}
A let expression allows binding of intermediate values to identifiers
for use in an expression. The start of the let expression is signaled
by the var keyword. They look like local variable declarations except
the scope of the variable only extends to following
expression. (Adapted from RefMan.)

Here’s an example (see the following code).

First x+x is computed and bound to sum, the result of the overall
expression on the right hand side of the update/assignment statement
is then the value of “sum * sum” given this binding. The binding does
not persist past the evaluation of the “let” expression.  The
expression is called a “let” expression because in many other
languages, you’d use a let keyword to write this: let sum = x + x in
sum * sum. Dafny just uses a slightly different syntax.

\begin{sphinxVerbatim}[commandchars=\\\{\}]
assert x == 3;               // from code above
var sumsquared := (var sum := x + x; sum * sum);  // let example
assert sumsquared == 36;     // because of the let expression
\end{sphinxVerbatim}


\chapter{7. Set Theory}
\label{\detokenize{07-set-theory::doc}}\label{\detokenize{07-set-theory:set-theory}}
Modern mathematics is largely founded on set theory: in particular, on
what is called \sphinxstyleemphasis{Zermelo-Fraenkel set theory with the axiom of Choice},
or \sphinxstyleemphasis{ZFC}. Every concept you have ever learned in mathematics can, at
least in principle, be reduced to expresions involving sets.  For
example, every natural number can be represented as a set: zero as the
\sphinxstyleemphasis{empty set, \{\}}; one as the set containing the empty set, \sphinxstyleemphasis{\{\{\}\}}; two
as the set that contains that set, \sphinxstyleemphasis{\{\{\{\}\}\}}; ad infinitum.

Set theory includes the treatment of sets, including the special cases
of relations (sets of tuples), functions (\sphinxstyleemphasis{single-valued} relations),
sequences (functions from natural numbers to elements), and other
related concepts.  ZFC is a widely accepted \sphinxstyleemphasis{formal foundation} for
modern mathematics: a set of axioms that describe properties of sets,
from which all the rest of mathematics can be deduced.


\section{Naive Set Theory}
\label{\detokenize{07-set-theory:naive-set-theory}}
So what is a set? A \sphinxstyleemphasis{naive} definition (which will actually be good
enough for our purposes and for most of practical computer science) is
that a set is just an unordered collection of elements. In principle,
these elements are themselves reducible to setsm but we don’t need to
think in such reductionist terms. We can think about a set of natural
numbers, for example, without having to think of each number as itself
being some weird kind of set.

In practice, we just think sets as unordered collections of elements
of some kind, where any given element is either \sphinxstyleemphasis{in} or \sphinxstyleemphasis{not in} any
given set. An object can be a member of many different sets, but can
only by in any give set zero or one times. Membership is binary.  So,
for example, when we combine (take the \sphinxstyleemphasis{union} of) two sets, each of
which contains some common element, the resulting combined set will
have that element as a member, but it won’t have it twice.

This chapter introduces \sphinxstyleemphasis{naive}, which is to say \sphinxstyleemphasis{intuitive and
practical}, set theory. It does not cover \sphinxstyleemphasis{axiomatic} set theory, in
which every concept is ultimately reduced to a set of logical axioms
that define what precisely it means to be a set and what operations
can be use to manipulate sets.


\section{Overly Naive Set Theory}
\label{\detokenize{07-set-theory:overly-naive-set-theory}}
Before we go on, however, we review a bit of history to understand
that an overly naive view of sets can lead to logical contradictions
that make such a theory useless as a foundation for mathematics.

One of the founders of modern logic, Gotlob Frege, had as his central
aim to establish logical foundations for all of mathematics: to show
that everything could be reduced to a set of axioms, or propositions
accepted without question, from which all other mathematical truths
could be deduced.  The concept of a set was central to his effort. His
logic therefore allowed one to define sets as collections of elements
that satisfy given propositions, and to talk about whether any given
element is in a particular set of not. Frege’s notion of sets, in
turn, traced back to the work of Georg Cantor.

But then, boom! In 1903, the British analytical philosopher, Bertrand
Russell, published a paper presenting a terrible paradox in Frege’s
conception. Russell showed that a logic involving naive set theory
would be \sphinxstyleemphasis{inconsistent} (self-contradicting) and there useless as a
foundation for mathematics.

To see the problem, one consider the set, \sphinxstyleemphasis{S}, of all sets that do not
contain themselves. In \sphinxstyleemphasis{set comprehension} notation, we would write
this set as \(S = \{ a: set | a \notin a \}.\) That is, \sphinxstyleemphasis{S} is the
set of elements, \sphinxstyleemphasis{a}, each a set, such that \sphinxstyleemphasis{a} is not a member of
itself.

Now ask the decisive question: Does \sphinxstyleemphasis{S} contain itself?

Let’s adopt a notation, \sphinxstyleemphasis{C(S)}, to represent the proposition that \sphinxstyleemphasis{S}
contains itself. Now suppose that \sphinxstyleemphasis{C(S)} is true, i.e., that \sphinxstyleemphasis{S} does
contain itself. In this case, \sphinxstyleemphasis{S}, being a set that contains itself,
cannot be a member of \sphinxstyleemphasis{S}, because we just defined \sphinxstyleemphasis{S} to be the set
of sets that do \sphinxstyleemphasis{not} contain themselves. So, the assumption that \sphinxstyleemphasis{S}
containss itself leads to the conclusion that \sphinxstyleemphasis{S} does not contain
itself. In logical terms, \(C(S) \rightarrow \neg C(S).\) This is
a contradiction and thus a logical impossibility.

Now suppose \sphinxstyleemphasis{S} does not contain itself: \(\neg C(S).\). Being
such a set, and given that \sphinxstyleemphasis{S} is the set of sets that do not contain
themselves, it must now be in \sphinxstyleemphasis{S}. So \(\neg C(S) \rightarrow
C(S).\) The assumption that it does \sphinxstyleemphasis{not} contain itself leads right
back to the conclusion that it \sphinxstyleemphasis{does} contain itself. Either the set
does or does not contain itself, but assuming either case leads to a
contradictory conclusion. All is lost!

That such an internal self-contradiction can arise in such a simple
way (or at all) is a complete disaster for any logic. The whole point
of a logic is that it gives one a way to reason that is sound, which
means that from true premises one can never reach a contradictory
conclusion. If something that is impossible can be proved to be true
in a given theory, then anything at all can be proved to be true, and
the whole notion of truth just collapses into meaninglessness. As soon
as Frege saw Russell’s Paradox, he knew that that was \sphinxstyleemphasis{game over} for
his profound attempt to base mathematics on a logic grounded in his
(Cantor’s) naive notion of sets.

Two solutions were eventually devised. Russell introduced a notion of
\sphinxstyleemphasis{types}, as opposed to sets, per se, as a foundation for mathematics.
The basic idea is that one can have elements of a certain \sphinxstyleemphasis{type}; then
sets of elements of that type, forming a new type; then sets of sets
elements of that type, forming yet another type; but one cannot even
talk about a set containing (or not containing) itself, because sets
can only contain elements of types lower in the type hierarchy.

The concept of types developed by Russell lead indirectly to modern
type theory, which remains an area of very active exploration in both
computer science and pure mathematics. Type theory is being explored
as an alternative foundation for mathematics, and is at the very heart
of a great deal of work going on in the areas of programming language
design and formal software specification and verification.

On the other hand, Zermelo repaired the paradox by adjusting some of
the axioms of set theory, to arrive at the starting point of what has
become ZFC. When we work in set theory today, whether with a \sphinxstyleemphasis{naive}
perspective or not, we are usually working in a set theory the logical
basic of which is ZFC.


\section{Sets}
\label{\detokenize{07-set-theory:sets}}
For our purposes, the \sphinxstyleemphasis{naive} notion of sets will be good enough. We
will take a \sphinxstyleemphasis{set} to be an unordered finite or infinite collection of
\sphinxstyleemphasis{elements}. An element is either \sphinxstyleemphasis{in} or \sphinxstyleemphasis{not in} a set, and can be in
a set at most once.  In this chapter, we will not encounter any of the
bizarre issues that Russell and others had to consider at the start of
the 20th century.

What we will find is that set-theoretical thinking is an incredibly
powerful intellectual tool. It’s at the heart of program specification
and verification, algorithm design and analysis, and theory of
computing, among many other areas in computer science. Moreover, Dafny
makes set theory not only fun but executable. The logic of Dafny, for
writing assertions, pre- and post-conditions, and invariants \sphinxstyleemphasis{is} set
theory, a first-order logic with sets and set-related operations as
built-in concepts.


\section{Set Theory Notations}
\label{\detokenize{07-set-theory:set-theory-notations}}

\subsection{Display notation}
\label{\detokenize{07-set-theory:display-notation}}
In everyday mathematical writing, andin Dafny, we denote small sets by
listing the elements of the set within curly brace. If \sphinxstyleemphasis{S} is the set
containing the numbers, one, two, and three, for example, we can write
\sphinxstyleemphasis{S} as \(\{ 1, 2, 3 \}.\)

In Dafny, we would write almost the same thing.

\begin{sphinxVerbatim}[commandchars=\\\{\}]
var S:set\PYGZlt{}int\PYGZgt{} := \PYGZob{} 1, 2, 3 \PYGZcb{}.
\end{sphinxVerbatim}

This code introduces the variable, \sphinxstyleemphasis{S}, declares that its type is
\sphinxstyleemphasis{finite set of integer} (\sphinxstyleemphasis{iset\textless{}T\textgreater{}} being the type of \sphinxstyleemphasis{infinite} sets
of elements of tyep \sphinxstyleemphasis{T}), and assigns to \sphinxstyleemphasis{S} the set value, \(\{
1, 2, 3 \}.\) Because the value on the right side of the assignment
operator, is evidently a set of integers, Dafny will infer the type of
\sphinxstyleemphasis{S}, and the explicit type declaration can therefore be omitted.

\begin{sphinxVerbatim}[commandchars=\\\{\}]
var S := \PYGZob{} 1, 2, 3 \PYGZcb{}.
\end{sphinxVerbatim}

When a set is finite but too large to write down easily as a list of
elements, but when it has a regular structure, mathematicians often
denote such a set using an elipsis. For example, a set, \sphinxstyleemphasis{S}, of even
natural numbers from zero to one hundred could be written like this:
\(S = \{ 0, 2, 4, \ldots, 100 \}.\) This expression is a kind of
quasi-formal mathematics. It’s mostly formal but leaves details that
an educated person should be able to infer to the human reader.

It is not (currently) possible to write such expressions in Dafny.
Dafny does not try to fill in missing details in specifications. A
system that does do such a thing might make a good research project.
On the other hand, ordinary mathematical writing as well as Dafny do
have ways to precisely specify sets, including even infinite sets, in
very concise ways, using what is called \sphinxstyleemphasis{set comprehension} or \sphinxstyleemphasis{set
builder} notation.


\subsection{Set comprehension notation}
\label{\detokenize{07-set-theory:set-comprehension-notation}}
Take the example of the set, \sphinxstyleemphasis{S}, of even numbers from zero to one
hundred, inclusive. We can denote this set precisely in mathematical
writing as \(S = \{ n: {\mathbb Z}~|~0 <= n <= 100 \land n~mod~2
= 0 \}.\) Let’s pull this expression apart.

The set expression (to the right of the first equals sign) can be read
in three parts. The vertical bar is read \sphinxstyleemphasis{such that}. To the left of
the bar is an expression identifying the set from which the elements
of \sphinxstyleemphasis{this} set are drawn, and a name is given to an arbitrary element
of this source set. So here we can say that \sphinxstyleemphasis{S} is a set each element
\sphinxstyleemphasis{n} of which is a natural number.  A name, here \sphinxstyleemphasis{n}, for an arbitrary
element is given for two purposes. First it desribes the form of
elements in the set being built: here just \sphinxstyleemphasis{integers}. Second, the
name can then be used in writing a condition that must be true of each
such element.  That expression is written to the right of the vertical
bar.

Here the condition is that each such element, \sphinxstyleemphasis{n} must be greater than
or equal to zero, less than or equal to one hundred, and even, in that
the remainder must be zero when \sphinxstyleemphasis{n} is divided by \sphinxstyleemphasis{2}. The overall set
comprehension expression is thus read literally as, \sphinxstyleemphasis{S} is the set of
integers, \sphinxstyleemphasis{n}, such that \sphinxstyleemphasis{n} is greater than or equal to zero, \sphinxstyleemphasis{n} is
less than or equal to 100, and \sphinxstyleemphasis{n} evenly divisible by \sphinxstyleemphasis{2}. A more
fluent reading would simply be \sphinxstyleemphasis{S} is the set of even integers between
zero and one hundred inclusive.

Dafny supports set comprehension notations. This same set would be
written as follows (we assume that the type of S has already been
declared to be \sphinxstyleemphasis{set\textless{}int\textgreater{})}:

\begin{sphinxVerbatim}[commandchars=\\\{\}]
S := set s: int \textbar{} 0 \PYGZlt{}= s \PYGZlt{}= 100;
\end{sphinxVerbatim}

Another way to define the same set in ordinary mathematical writing
would use a slightly richer form of set comprehension notation. In
particular, we can define the same set as the set of values of the
expression \sphinxstyleemphasis{2*n} for \sphinxstyleemphasis{n} is in the range zero to fifty, inclusive.
Where it’s readily inferred, mathematicians will usually also leave
out explicit type information. {\color{red}\bfseries{}{}`}S = \{ 2 * n \textbar{} 0 \textless{}= n \textless{}= 50 \}. In
this expression it’s inferred that \sphinxstyleemphasis{n} ranges over all the natural
numbers, these values are \sphinxstyleemphasis{filtered} by the expression on the right,
and these filtered values are then fed through the expression on the
left of the bar to produce the elements of the intended set.

Dafny also supports set comprehension notation in this style. To
define this very same set in Dafny we could also write this:

\begin{sphinxVerbatim}[commandchars=\\\{\}]
S := set s: int \textbar{} 0 \PYGZlt{}= s \PYGZlt{}= 50 :: 2 * s;
\end{sphinxVerbatim}

This command assigns to S a set of values, \sphinxstyleemphasis{2 * s},, where \sphinxstyleemphasis{s}
ranges over the integers and satisfies the predicate (or filter)
\sphinxstyleemphasis{0 \textless{}= s \textless{}= 50}.

The collection of values from which element are drawn to be
build into a new set need not just be a built-in type but can
be another programmer-defined set. Given that \sphinxstyleemphasis{S} is the set
of even numbers from zero to one hundred, we can define the
subset of \sphinxstyleemphasis{S} of elements that are less than \sphinxstyleemphasis{25} by writing
a richer set comprehension. In pure mathematical writing, we
could write \(T = \{ t | t \in S \land t < 25\}.\) That is,
\sphinxstyleemphasis{T} is the set of elements that are in \sphinxstyleemphasis{S} and less than \sphinxstyleemphasis{25}.
The Dafny notation is a little different, but not too much:

\begin{sphinxVerbatim}[commandchars=\\\{\}]
var T := set t \textbar{} t in S \PYGZam{}\PYGZam{} t \PYGZlt{} 25;
\end{sphinxVerbatim}

This Dafny code defines \sphinxstyleemphasis{T} to be the set (of integers, but note that
we let Dafny infer the type of \sphinxstyleemphasis{t} in this case), such that \sphinxstyleemphasis{t} is in
the set \sphinxstyleemphasis{S} (that we just defined) and \sphinxstyleemphasis{t} is also less than \sphinxstyleemphasis{25}.

As a final example, let’s suppose that we want to define the set of
all ordered pairs whose first elements are from \sphinxstyleemphasis{S} and whose second
elements are from \sphinxstyleemphasis{T}, as we’ve defined them here. For example, the
pair \sphinxstyleemphasis{(76,24)} would be in this set, but not \sphinxstyleemphasis{(24 76)}. In ordinary
mathematical writing, this would be \(\{ (s,t) | s \in S \land t
\in T\}.\) This set is, as we’ll learn more about shortly, called the
\sphinxstyleemphasis{product set} of the sets, \sphinxstyleemphasis{S} and \sphinxstyleemphasis{T}.

In Danfy, this would be written like this:

\begin{sphinxVerbatim}[commandchars=\\\{\}]
var Q := set s, t \textbar{} s in S \PYGZam{}\PYGZam{} t in T :: (s, t);
\end{sphinxVerbatim}

This code assigns to the new variable, \sphinxstyleemphasis{Q}, a set formed by taking
elements, \sphinxstyleemphasis{s} and \sphinxstyleemphasis{t},, such that \sphinxstyleemphasis{s} is in \sphinxstyleemphasis{S} and \sphinxstyleemphasis{t} is in \sphinxstyleemphasis{T}, and
forming the elements of the new set as tuples, \sphinxstyleemphasis{(s, t)}. This is a far
easier way to write code for a product set than by explicit iteration
over the sets \sphinxstyleemphasis{S} and \sphinxstyleemphasis{T}!

In Dafny, the way to extract an element of a tuple, \sphinxstyleemphasis{t}, of arity,
\sphinxstyleemphasis{n}, is by writing \sphinxstyleemphasis{t.n}, where \sphinxstyleemphasis{n} is a natural number in the range
\sphinxstyleemphasis{0} up to \sphinxstyleemphasis{n - 1}. So, for example, \sphinxstyleemphasis{(3, 4).1} evaluates to \sphinxstyleemphasis{4}. It’s
not a notation that is common to many programming languages. One can
think of it as a kind of subscripting, but using a different notation
than the usual square bracket subscripting used with sequences.


\section{Set Operations}
\label{\detokenize{07-set-theory:set-operations}}

\subsection{Cardinality}
\label{\detokenize{07-set-theory:cardinality}}
By the cardinality of a set, \sphinxstyleemphasis{S}, we mean the number of elements
in S. When \sphinxstyleemphasis{S} is finite, the cardinality of \sphinxstyleemphasis{S} is a natural number.
The cardinarily of the empty set is zero, for example, because it has
no (zero) elements. In ordinary mathematics, if \sphinxstyleemphasis{S} is a finite set,
then its cardinality is denoted \(|S|\). With \sphinxstyleemphasis{S} defined as in
the preceding section, the cardinality of \sphinxstyleemphasis{S} is \sphinxstyleemphasis{50}. (There are
\sphinxstyleemphasis{50} numbers between \sphinxstyleemphasis{0} and \sphinxstyleemphasis{49}, inclusive.)

The Dafny notation for set cardinality is just the same. The following
code will print the cardinality of \sphinxstyleemphasis{S}, namely \sphinxstyleemphasis{50}, for example.

\begin{sphinxVerbatim}[commandchars=\\\{\}]
print \textbar{}S\textbar{};
\end{sphinxVerbatim}

If a set is infinite in size, as for example is the set of natural
numbers, the cardinality of the set is obviously not any natural
number. One has entered the realm of \sphinxstyleemphasis{transfinite numbers}. We will
discuss transfinite numbers later in this course. In Dafny, as you
might expect, the cardinality operator is not defined for infinite
sets (of type \sphinxstyleemphasis{iset\textless{}T\textgreater{}}).


\subsection{Equality}
\label{\detokenize{07-set-theory:equality}}
Two sets are considered equal if and only if they contain exactly
the same elements. To assert that sets \sphinxstyleemphasis{S} and \sphinxstyleemphasis{T} are equal in
mathematical writing, we would write \sphinxstyleemphasis{S = T}. In Dafny, such an
assertion would be written, \sphinxstyleemphasis{S == T}.


\subsection{Subset}
\label{\detokenize{07-set-theory:subset}}
A set, \sphinxstyleemphasis{T}, can be said to be a subset of a set \sphinxstyleemphasis{S} if and only if
every element in \sphinxstyleemphasis{T} is also in \sphinxstyleemphasis{S}. In this case, mathematicians
write \(T \subseteq S\). In logical notation, we would write,
\(T \subseteq S \iff \forall t \in T, t \in S\). That is, \sphinxstyleemphasis{T} is a
subset of \sphinxstyleemphasis{S} if and only if every element in \sphinxstyleemphasis{T} is also in \sphinxstyleemphasis{S}.

A set \sphinxstyleemphasis{T}, is said to be a \sphinxstyleemphasis{proper} subset of \sphinxstyleemphasis{S}, if \sphinxstyleemphasis{T} is a subset
of \sphinxstyleemphasis{S} but \sphinxstyleemphasis{T} is not equal to \sphinxstyleemphasis{S}. This is written in mathematics as
\(T \subset S\). In other words, every element of \sphinxstyleemphasis{T} is in \sphinxstyleemphasis{S} but
there is at least one element of \sphinxstyleemphasis{S} that is not in \sphinxstyleemphasis{T}.

In our example, \sphinxstyleemphasis{T} (the set of even natural numbers less than \sphinxstyleemphasis{25})
is a proper subset of \sphinxstyleemphasis{S} (the set of even natural numbers less than
or equal to \sphinxstyleemphasis{100}).

In Dafny, one uses the usual arithmetic less and and less than or
equal operator symbols, \sphinxstyleemphasis{\textless{}} and \sphinxstyleemphasis{\textless{}=}, to assert \sphinxstyleemphasis{proper subset} and
\sphinxstyleemphasis{subset} relationships, respectively. The first two of the following
assertions are thus both true in Dafny, but the third is not. That
said, limitations in the Dafny verifier make it hard for Dafny to see
the truth of such assertions without help. We will not discuss how to
provide such help at this point.

\begin{sphinxVerbatim}[commandchars=\\\{\}]
assert T \PYGZlt{} S;
assert T \PYGZlt{}= S;
assert S \PYGZlt{}= T;
\end{sphinxVerbatim}

We note every set is a subset, but not a proper subset, of
itself. It’s also the case that the empty set is a subset of every
set, in that \sphinxstyleemphasis{all} elements in the empty set are in any other set,
because there are none. In logic-speak, we’d say \sphinxstyleemphasis{a universally
quantified proposition over an empty set is trivially true.}

If we reverse the operator, we get the notion of supersets and proper
supersets. If \sphinxstyleemphasis{T} is a subset of \sphinxstyleemphasis{S}, then \sphinxstyleemphasis{S} is a superset of \sphinxstyleemphasis{T},
written, \(S \supseteq T\). If \sphinxstyleemphasis{T} is a proper subset of \sphinxstyleemphasis{S} then
\sphinxstyleemphasis{S} is a proper superset of \sphinxstyleemphasis{T}, written \(S \supset T\). In
Dafny, the greater than and greater than or equals operator are used
to denote proper superset and superset relationships between sets.
So, for example, \sphinxstyleemphasis{S \textgreater{}= T} is the assertion that \sphinxstyleemphasis{S} is a superset of
\sphinxstyleemphasis{T}. Note that every set is a superset of itself, but never a proper
superset of itself, and every set is a superset of the empty set.


\subsection{Intersection}
\label{\detokenize{07-set-theory:intersection}}
The intersection, \(S \cap T\), of two sets, \sphinxstyleemphasis{S} and \sphinxstyleemphasis{T}, is the
set of elements that are in both \sphinxstyleemphasis{S} and \sphinxstyleemphasis{T}. Mathematically speaking,
\(S \cap T = \{ e~|~e \in S \land e \in T \}\).

In Dafny, the \sphinxstyleemphasis{*} operator is used for set intersection.  The
intersection of \sphinxstyleemphasis{S} and \sphinxstyleemphasis{T} is thus written \sphinxstyleemphasis{S * T}. For example, the
command \sphinxstyleemphasis{Q := S * T} assigns the intersection of \sphinxstyleemphasis{S} and \sphinxstyleemphasis{T} as the
new value of \sphinxstyleemphasis{Q}.


\subsection{Union}
\label{\detokenize{07-set-theory:union}}
The union, \(S \cup T\), of two sets, \sphinxstyleemphasis{S} and \sphinxstyleemphasis{T}, is the set of
elements that are in either (including both) \sphinxstyleemphasis{S} and \sphinxstyleemphasis{T}. That is,
\(S \cup T = \{ e~|~e \in S \lor e \in T \}\).

In Dafny, the \sphinxstyleemphasis{+} operator is used for set union.  The union of \sphinxstyleemphasis{S}
and \sphinxstyleemphasis{T} is thus written \sphinxstyleemphasis{S + T}. For example, the command \sphinxstyleemphasis{V := S +
T} assigns the union of \sphinxstyleemphasis{S} and \sphinxstyleemphasis{T} as the new value of \sphinxstyleemphasis{V}.


\subsection{Difference}
\label{\detokenize{07-set-theory:difference}}
The difference, \sphinxstyleemphasis{S\textbackslash{}T} (\sphinxstyleemphasis{S} minus \sphinxstyleemphasis{T}), of sets \sphinxstyleemphasis{S} and \sphinxstyleemphasis{T} is the set
of elements in \sphinxstyleemphasis{S} that are not also in \sphinxstyleemphasis{T}. Thus, :math:{\color{red}\bfseries{}{}`}S setminus
T = \{e\textasciitilde{}\textbar{}\textasciitilde{}e in S land e notin T). In Dafny, the minus sign is used
to denote set difference, as in the expression, \sphinxstyleemphasis{S - T}. Operators in
Dafny can be applied to sets to make up more complex expressions. So,
for example, \sphinxstyleemphasis{\textbar{}S-T\textbar{}} denotes the cardinality of \sphinxstyleemphasis{S-T}.


\subsection{Product Set}
\label{\detokenize{07-set-theory:product-set}}
The product set, \(S \times T\), is the set of all the ordered
pairs, \sphinxstyleemphasis{(s,t)}, that can be formed by taking one element, \sphinxstyleemphasis{s}, from
\sphinxstyleemphasis{S}, and one element, \sphinxstyleemphasis{t}, from \sphinxstyleemphasis{T}. That is, \(S \times T = \{
(s, t) | s \in S \land t \in T \}\). The cardinality of a product set
is the product of the cardinalities of the individual sets.

There is no product set operator, per se, in Dafny, but given sets,
\sphinxstyleemphasis{S} and \sphinxstyleemphasis{T} a product set can easily be expressed using Dafny’s set
comprehension notation: \sphinxstyleemphasis{set s, t: s in S \&\& t in T :: (s,t)}. The
keyword, \sphinxstyleemphasis{set}, is followed by the names of the variables that will be
used to form the set comprehension expression, followed by a colon,
followed by an assertion that selects the values of \sphinxstyleemphasis{s} and \sphinxstyleemphasis{t} that
will be included in the result, followed by a double colon, and then,
finally an expression using the local variables that states how each
value of the resulting set will be formed.


\subsection{Power Set}
\label{\detokenize{07-set-theory:power-set}}
The power set of a set, \sphinxstyleemphasis{S}, denoted \({\mathbb P}(S),\) is the
set of all subsets of \sphinxstyleemphasis{S}. If \sphinxstyleemphasis{S = \{1, 2 \}}, for example, the powerset
of \sphinxstyleemphasis{S} is the set containing the proper and improper subsets of \sphinxstyleemphasis{S},
namely \sphinxstyleemphasis{\{\}, \{ 1 \}, \{ 2 \},} and \sphinxstyleemphasis{\{ 1, 2\}}.

The powerset of a set with \sphinxstyleemphasis{n} element will have \(2^n\) elements.
Consider the powerset of the empty set. The only subset of the empty
set is the empty set itself, so the powerset of the empty set is the
set containing only the empty set. This set has just \sphinxstyleemphasis{1} element. It’s
cardinality thus satisfies the rule, as \sphinxstyleemphasis{2} to the power, zero (the
number of elements in the empty set), is \sphinxstyleemphasis{1}.

Now suppose that for every set, \sphinxstyleemphasis{S}, with cardinality \sphinxstyleemphasis{n}, the
cardinality of its powerset is \sphinxstyleemphasis{2} to the \sphinxstyleemphasis{n}. Consider a set, \sphinxstyleemphasis{S’},
of cardinality one bigger than that of \sphinxstyleemphasis{S}. Its powerset contains
every set in the powerset of \sphinxstyleemphasis{S}, plus every set in that set with the
new element included, and that’s all the element it includes.

The number of sets in the powerset of \sphinxstyleemphasis{S’} is thus double the number
of sets in the powerset of \sphinxstyleemphasis{S}. Given that the cardinality of the
powerset of \sphinxstyleemphasis{S} is \sphinxstyleemphasis{2} to the \sphinxstyleemphasis{n}, the cardinality of \sphinxstyleemphasis{S’}, being
twice that number, is \sphinxstyleemphasis{2} to the \sphinxstyleemphasis{n + 1}.

Now because the rule holds for sets of size zero, and whenver it holds
for sets of size \sphinxstyleemphasis{n} it also holds for sets of size \sphinxstyleemphasis{n + 1}, it must
hold for sets of every (finite) size. So what we have is an informal
\sphinxstyleemphasis{proof by induction} of a theorem: \(\forall S, |{\mathbb P}(S)|
= 2^{|S|}\).

In Dafny, there is no explicit powerset operator, one that would take
a set and returning its powerset, but the concept can be expressed in
an elegant form using a set comprehension. The solution is simply to
say \sphinxstyleemphasis{the set of all sets that are subsets of a given set, *}. In pure
mathematical notation this would be \({ R | R \subseteq S }.\) In
Dafny it’s basically the same expression.  The follwing three-line
program computes and prints out the powerset of \sphinxstyleemphasis{S = \{ 1, 2, 3 \}}.
The key expression is to the right of the assignment operator on the
second line.

\begin{sphinxVerbatim}[commandchars=\\\{\}]
var S := \PYGZob{} 1, 2, 3 \PYGZcb{};
var P := set R \textbar{} R \PYGZlt{}= S;
print P;
\end{sphinxVerbatim}

Exercise: Write a pure function that when given a value of type set\textless{}T\textgreater{}
returns its powerset. The function will have to be polymorphic.  Call
it powerset\textless{}T\textgreater{}.


\section{Tuples}
\label{\detokenize{07-set-theory:tuples}}
A tuple is an ordered collection of elements. The type of elements in
a tuple need not all be be the same. The number of elements in a tuple
is called its \sphinxstyleemphasis{arity}. Ordered pairs are tuples of arity, \sphinxstyleemphasis{2}, for
example. A tuple of arity \sphinxstyleemphasis{3} can be called a (an ordered) \sphinxstyleemphasis{triple}.
A tuple of a larger arity, \sphinxstyleemphasis{n}, is called an \sphinxstyleemphasis{n-tuple}.  The tuple,
\sphinxstyleemphasis{(7, X, “house”, square\_func)}, for example, is a \sphinxstyleemphasis{4-tuple}.

As is evident in this example, the elements of a tuple are in general
not of the same type, or drawn from the same sets. Here, the first
element is an integer; the second, a variable;, the third, a string;
and last, a function.

An \sphinxstyleemphasis{n}-tuples should be understood as values taken from a product of
\sphinxstyleemphasis{n} sets.  If \sphinxstyleemphasis{S} and \sphinxstyleemphasis{T} are our sets of even numbers between zero
and one hundred, and zero and twenty four, for example, then the
ordered pair, \sphinxstyleemphasis{(60,24)} is an element of the product set \(S
\times T\).  The preceding \sphinxstyleemphasis{4}-tuple would have come from a product of
four sets: one of integers, one of variables, one of strings, and one
of functions.

The \sphinxstyleemphasis{type} of a tuple is the tuple of the types of its elements. In
mathematical writing, we’d say that the tuple, \sphinxstyleemphasis{(-3,4)} is al element
of the set \({\mathbb Z} \times {\mathbb Z},\) and if asked about
its type, most mathematicians would say \sphinxstyleemphasis{pair of integers}. In Dafny,
where types are more explicit than they usually are in quasi-formal
mathematical discourse, the type of this tuple is \sphinxstyleemphasis{(int, int)}. In
general, in both math and in Dafny, in particular, the type of a tuple
in a set product, ::\sphinxtitleref{S\_1 times S\_2 times ldots time S\_n}, where
the types of these sets are \(T_1, \ldots, T_n\) is \((T_1,
\ldots, T_n)\).

The elements of a tuple are sometimes called \sphinxstyleemphasis{fields of that tuple.
Given an *n}-tuple, \sphinxstyleemphasis{t}, we are often interested in working with the
value of one of its fields. We thus need a function for \sphinxstyleemphasis{projecting}
the value of a field out of a tuple. We actually think of an \sphinxstyleemphasis{n}-tuple
as coming with \sphinxstyleemphasis{n} projection functions, one for each field.

Projection functions are usually written using the Greek letter,
::\sphinxtitleref{pi}, with a natural number subscript indicating which field a
given projection function ” projects”. Given a \sphinxstyleemphasis{4}-tuple, \sphinxstyleemphasis{t = (7, X,
“house”, square\_func)}, we would have math::\sphinxtitleref{pi\_0(t) = 7} and
\(\pi_3(t) = square_func.\)

The type of a projection funcion is \sphinxstyleemphasis{function from tuple type to field
type}. In general, because tuples have fields of different types, they
will also have projection functions of different types. For example,
\(pi_0\) here is of type (in Dafny) \((int, variable, string,
int \rightarrow int) \rightarrow {\mathbb Z}\) while \(pi_3\) is of
type \((int, variable, string, int \rightarrow int) \rightarrow
(int \rightarrow int).\)

In Dafny, tuples are written as they are in mathematics, as lists of
field values separated by commas and enclosed in parentheses.  For
example \sphinxstyleemphasis{t := (1, “hello”, {[}1,2,3{]})” assigns to *t} a \sphinxstyleemphasis{3-tuple} whose
first field has the value, \sphinxstyleemphasis{1} (of type \sphinxstyleemphasis{int}); whose second field has
the value, “hello”, a string; and whose third element is the list of
integers, \sphinxstyleemphasis{{[}2, 4, 6{]}}.

Projection in Dafny is accomplished using the \sphinxstyleemphasis{tuple} subscripting (as
opposed to array or list subscripting) operation. Tuple subscripting is
done by putting a dot (period) followed by an index after the tuple
expression. Here’s a little Dafny code to illustrate. It defines \sphinxstyleemphasis{t}
to be the triple, \sphinxstyleemphasis{(7, ‘X’, “hello”)} (of type \sphinxstyleemphasis{(int, char, string)}),
and then usses the \sphinxstyleemphasis{.0} and \sphinxstyleemphasis{.2} projection functions to project the
first and third elements of the tuple, which it prints. To make the
type of the tuple explicit, the final line of code declare \sphinxstyleemphasis{t’} to be
the same tuple value, but this time explicitly declares its type.

\begin{sphinxVerbatim}[commandchars=\\\{\}]
var t := (7, \PYGZsq{}X\PYGZsq{}, \PYGZdq{}hello\PYGZdq{});
print t.0;
print t.2;
var t\PYGZsq{}: (int, char, string) := (7, \PYGZsq{}X\PYGZsq{}, \PYGZdq{}hello\PYGZdq{});
\end{sphinxVerbatim}

While all of this might seem a little abstract, it’s actually simple
and very useful. Any table of data, such as a table with columns that
hold names, birthdays, and social security numbers, represents data in
a product set. Each row is a tuple. The columns correspond to the sets
from which the field values are drawn. One set is a set of names; the
second, a set birthdays; the third, a set of social security numbers.
Each row is just a particular tuple in product of these three sets,
and the table as a whole is what we call a \sphinxstyleemphasis{relation}. If you have
heard of a \sphinxstyleemphasis{relational database}, you now know what kind of data such
a system handles: tables, i.e., \sphinxstyleemphasis{relations}.


\section{Relations}
\label{\detokenize{07-set-theory:relations}}
A relation in nothing but a subset of (the tuples in) a product set. A
table such as the one just described, will, in practice, usually not
have a row with every possible combination of names, birthdays, and
SSNs. In other words, it won’t be the entire product of the sets from
which the field values drawn. Rather, it will usually contain a small
subset of the product set.

In mathematical writing, we will thus often see a sentence of the
form, Let \(R \subseteq S \times T\) be a (binary) relation on \sphinxstyleemphasis{S}
and \sphinxstyleemphasis{T}. All this says is that \sphinxstyleemphasis{R} is some subset of the set of all
tuples in the product set of \sphinxstyleemphasis{S} and \sphinxstyleemphasis{T}. If \sphinxstyleemphasis{S = \{ hot, cold \}} and
\sphinxstyleemphasis{T = \{ cat, dog \}}, then the product set is \sphinxstyleemphasis{\{ (hot, cat), (hot, dog),
(cold, cat), (cold, dog) \}}, and a relation on \sphinxstyleemphasis{S} and \sphinxstyleemphasis{T} is any
subset of this product set.  The set, \sphinxstyleemphasis{\{ (hot, cat), (cold, dog) \}} is
thus one such relation on \sphinxstyleemphasis{S} and \sphinxstyleemphasis{T}.

Here’s an exercise. If \sphinxstyleemphasis{S} and \sphinxstyleemphasis{T} are finite sets, with cardinalities
\sphinxstyleemphasis{\textbar{}S\textbar{} = n} and \sphinxstyleemphasis{\textbar{}T\textbar{} = m}, how many relations are there over \sphinxstyleemphasis{S} and
\sphinxstyleemphasis{T}? Hint: First, how many tuples are in the product set? Second, how
many subsets are there of that set? For fun, write a little Dafny
program that takes two sets of integers as arguments as return the
number of relations over them.  Write another function that takes two
sets and returns the set of all possible relations over the sets. Use
a set comprehension expression rather than writing a while loop. Be
careful: the number of possible relations will be very large even in
cases where the given sets contain only a few elements each.


\section{Binary Relations}
\label{\detokenize{07-set-theory:binary-relations}}
Binary relations, which play an especially important role in
mathematics and computer science, are relations over just \sphinxstyleemphasis{2}
sets. Suppose \(R \subseteq S \times T\) is a binary relation on
\sphinxstyleemphasis{S} and \sphinxstyleemphasis{T}. Then \sphinxstyleemphasis{S} is called the \sphinxstyleemphasis{domain} of the relation, and \sphinxstyleemphasis{T}
is called its \sphinxstyleemphasis{co-domain}. That is, a binary relation is a subset of
the ordered pairs in a product of the given domain and codomain sets.

If a particular tuple, \sphinxstyleemphasis{(s, t)} is an element of such a relation, \sphinxstyleemphasis{R},
we will say \sphinxstyleemphasis{R} is \sphinxstyleemphasis{defined for} the value, s, and that \sphinxstyleemphasis{R achieves}
the value, \sphinxstyleemphasis{t}. The \sphinxstyleemphasis{support} of a relation is the subset of values in
the domain on which it is defined. The \sphinxstyleemphasis{range} of a relation is the
subset of co-domain values that it achieves.

For example, if \sphinxstyleemphasis{S = \{ hot, cold \}} and \sphinxstyleemphasis{T = \{ cat, dog \}}, and \sphinxstyleemphasis{R =
*\{ (hot, cat), (hot, dog) \}}, then the domain of \sphinxstyleemphasis{R} is \sphinxstyleemphasis{S}; the
co-domain of \sphinxstyleemphasis{R} is \sphinxstyleemphasis{T}; the support of \sphinxstyleemphasis{R} is just \sphinxstyleemphasis{\{ hot \}} (and \sphinxstyleemphasis{R}
is thus \sphinxstyleemphasis{not defined} for the value \sphinxstyleemphasis{cold}); and the range of \sphinxstyleemphasis{R} is
the whole co-domain, \sphinxstyleemphasis{T}.

The everyday functions you have studies in mathematics are binary
relations, albeit usually infinite ones. For example, the \sphinxstyleemphasis{square}
function, that associates every real number with its square, can be
understood as the infinite set of ordered pairs of real numbers in
which the second is the square of the first. Mathematically this is
:\{ (x, y) \textbar{} y = x\textasciicircum{}2 \}:{\color{red}\bfseries{}{}`}, where we take as implicit that \sphinxstyleemphasis{x} and \sphinxstyleemphasis{y}
range over the real numbers. Elements of this set include the pairs,
\sphinxstyleemphasis{(-2, 4)} and \sphinxstyleemphasis{(2, 4)}.

The concept of \sphinxstyleemphasis{square roots} of real numbers is also best understood
as a relation. The tuples are again pairs of real numbers, but now the
elements include tuples, \sphinxstyleemphasis{(4, 2)} and \sphinxstyleemphasis{(4, -2)}.


\section{Functions: \sphinxstyleemphasis{Single-Valued} Relations}
\label{\detokenize{07-set-theory:functions-single-valued-relations}}
A binary-relation is said to be \sphinxstyleemphasis{single-valued} if it does not have
tuples with the same first element and different second elements.  A
single-valued binary relation is also called a \sphinxstyleemphasis{function}.  Another
way to say that \sphinxstyleemphasis{R} is single valued is to say that if \sphinxstyleemphasis{(x, y)} and
\sphinxstyleemphasis{(x, z)} are both in \sphinxstyleemphasis{R} then it must be that \sphinxstyleemphasis{y} and \sphinxstyleemphasis{z} are the same
value. Otherwise the relation would not be single-valued! To be more
precise, then, if \(R \subseteq S \times T\), is single valued
relation, then \((x, y) \in R \land (x, z) \in R \rightarrow y =
z\).

As an example of a single-valued relation, i.e., a function, consider
the \sphinxstyleemphasis{square}. For any given natural number (in the domain) under this
function there is just a \sphinxstyleemphasis{single} associated value in the range (the
square of the first number). The relation is single-valued in exactly
this sense. By contrast, the square root relation is not a function,
because it is not single-valued. For any given non-negative number in
its domain, there are \sphinxstyleemphasis{two} associated square roots in its range. The
relation is not single-valued and so it is not a function.

There are several ways to represent functions in Dafny, or any other
programming language. One can represent a given function \sphinxstyleemphasis{implicity}:
as a \sphinxstyleemphasis{program} that computes that function. But one can also represent
a function \sphinxstyleemphasis{explicitly}, as a relation: that is, as a set of pairs.
The (polymorphic) \sphinxstyleemphasis{map} type in Dafny provides such a representation.

A “map”, i.e., a value of type \sphinxstyleemphasis{map\textless{}S,T\textgreater{}} (where \sphinxstyleemphasis{S} and \sphinxstyleemphasis{T} are type
parameters), is to be understood as an explicit representation of a
single-valued relation: a set of pairs: a function. In addition to a
mere set of pairs, this data type also provides helpful functions and
a clever representation underlying representation that both enforce
the single-valuedness of maps, and that make it very efficient to look
up range values given domain values where the map is defined, i.e., to
\sphinxstyleemphasis{apply} such a function to a domain value (a “key”) to obtained the
related range \sphinxstyleemphasis{value}.

Given a Dafny map object, \sphinxstyleemphasis{m}, of type \sphinxstyleemphasis{map\textless{}S,T\textgreater{}}, one can obtain the
set of values of type \sphinxstyleemphasis{S} for which the map is defined as \sphinxstyleemphasis{m.Keys().}
One can obtain the range, i.e., the set of values of type \sphinxstyleemphasis{T} that the
map maps \sphinxstyleemphasis{to}, as \sphinxstyleemphasis{m.Values().} One can determine whether a given key,
\sphinxstyleemphasis{s} of type \sphinxstyleemphasis{S} is defined in a map with the expression, \sphinxstyleemphasis{s in m}.

Exercise: Write a method (or a function) that when given a map\textless{}S,T\textgreater{} as
an argument returns a set\textless{}(T,S)\textgreater{} as a result where the return result
represents the \sphinxstyleemphasis{inverse} of the map. The inverse of a function is not
necessarily a function so the inverse of a map cannot be represented
as a map, in general. Rather, we represent the inverse just as a \sphinxstyleemphasis{set}
of \sphinxstyleemphasis{(S,T)} tuples.

Exercise: Write a pure function that when given a set of ordered pairs
returns true if, viewed as a relation, the set is also a function, and
that returns false, otherwise.

Exercise: Write a function or method that takes a set of ordered pairs
with a pre-condition requiring that the set satisfy the predicate from
the preceding exercise and that then returns a \sphinxstyleemphasis{map} that contains the
same set of pairs as the given set.

Exercise: Write a function that takes a map as an argument and that
returns true if the function that it represents is invertible and that
otherwise returns false. Then write a function that takes a map
satisfying the precondition that it be invertible and that in this
case returns its inverse, also as a map.


\section{Properties of Relations and Functions}
\label{\detokenize{07-set-theory:properties-of-relations-and-functions}}
We now introduce essential concepts and terminology regarding for
distinguishing essential properties and special cases of relations and
functions.


\subsection{Total vs Partial}
\label{\detokenize{07-set-theory:total-vs-partial}}
A binary relation (including a function) is said to be \sphinxstyleemphasis{total} if
every element of its domain appears as the first element in at least
one tuple, i.e., its \sphinxstyleemphasis{support} is its entire \sphinxstyleemphasis{domain}.  A relation
that is not total is said to be \sphinxstyleemphasis{partial}. For example, the square
function on the real numbers is total, in that it is defined on its
entire real number domain. By contrast, the square root function is
not total (if it domain is taken to be the real numbers) because it is
not defined for real numbers that are less than zero.

Note that if one considers a slightly different function, the square
root function on the \sphinxstyleemphasis{non-negative} real numbers the only difference
being in the domainm then this function \sphinxstyleemphasis{is} total. Totality is thus
relative to the specified domain. Here we have two functions with the
very same set of ordered pairs, but one is total and the other is not.

Exercises: Is the function \sphinxstyleemphasis{y = x} on the real numbers total?  Is the
\sphinxstyleemphasis{log} function defined on the non-negative real numbers total? Answer:
no, because it’t not defined at \sphinxstyleemphasis{x = 0}.  Is the \sphinxstyleemphasis{SSN} function, that
assigns a U.S. Social Security Number to every person, total? No, not
every person has a U.S. Social Security number.

Implementing partial functions as methods or pure function in software
presents certain problems. Either a pre-conditions has to be enforced
to prevent the function or method being called with a value for which
it’s not defined, or the function or method needs to be made total by
returning some kind of \sphinxstyleemphasis{error} value if it’s called with such a value.
In this case, callers of such a function are obligated always to check
whether \sphinxstyleemphasis{some} validfunction value was returned or whether instead a
value was returned that indicates that there is \sphinxstyleemphasis{no such value}. Such
a value indicates an \sphinxstyleemphasis{error} in the use of the function, but one that
the program caught. The failure of programmers systematically to check
for \sphinxstyleemphasis{error returns} is a common source of bugs in real software.

Finally we note that by enforcing a requirement that every loop and
recursion terminates, Dafny demands that every function and method be
total in the sense that it returns and that it returns some value,
even it it’s a value that could flag an error.

When a Dafny total function is used to implement a mathematical
function that is itself partial (e.g., \sphinxstyleemphasis{log(x)} for any real number,
\sphinxstyleemphasis{x}), the problem thus arises what to return for inputs for which the
underlying mathematical function is not defined.  A little later in
the course we will see a nice way to handle this issue using what are
called \sphinxstyleemphasis{option} types. An option type is like a box that contains
either a good value or an error flag; and to get a good value out of
such a box, one must explicitly check to see whether the box has a
good value in it or, alternatively, and error flag.


\subsection{Inverse}
\label{\detokenize{07-set-theory:inverse}}
The inverse of a given binary relation is simply the set of tuples
formed by reversing the order of all of the given tuples. To put this
in mathematical notation, if \sphinxstyleemphasis{R} is a relation, its inverse, denoted
::\sphinxtitleref{R\textasciicircum{}\{-1\}}, is \(\{ (y, x) | (x, y) \in R \}\). You can see this
immediately in our example of squares and square roots. Each of these
relations is the inverse of the other. One contains the tuples, \sphinxstyleemphasis{(-2,
4), (2, 4)}, while the other contains \sphinxstyleemphasis{(4, 2), (4, -2)}.

It should immediately be clear that the inverse of a function is not
always also a function. The inverse of the \sphinxstyleemphasis{square} function is the
\sphinxstyleemphasis{square root} relation, but that relation is not itself a function,
because it is not single valued.

Here’s a visual way to think about these concept. Consider the graph
of the \sphinxstyleemphasis{square} function. Its a parabola that opens either upward in
the \sphinxstyleemphasis{y} direction, or downward. Now select any value for \sphinxstyleemphasis{x} and draw
a vertical line. It will intersect the parabola at only one point.
The function is single-valued.

The graph of a square root function, on the other hand, is a parabola
that opens to the left or right. So if one draws a vertial line at
some value of \sphinxstyleemphasis{x}, either the line fails to hit the graph at all (the
square root function is not defined for all values of \sphinxstyleemphasis{x}), or it
intersects the line at two points. The square root “function” is not
single-valued, and isn’t really even a \sphinxstyleemphasis{function} at all. (If the
vertical line hits the parabola right at its bottom, the set of points
at which it intersects contains just one element, but if one takes the
solution set to be a \sphinxstyleemphasis{multi-set}, then the value, zero, occurs in that
set twice.)

A function whose inverse is a function is said to be \sphinxstyleemphasis{invertible}.
The function, \sphinxstyleemphasis{f(x) = x} (or \sphinxstyleemphasis{y = x} if you prefer) is invertible in
this sense. In fact, its inverse is itself.

Exercise: Is the cube root function invertible? Prove it informally.

Exercise: Write a definition in mathematical logic of what precisely
it means for a function to be invertible. Model your definition on our
definition of what it means for a relation to be single valued.


\subsection{Injective}
\label{\detokenize{07-set-theory:injective}}
A relation (or function, in particular) is said to be \sphinxstyleemphasis{injective} if
no two elements of the domain are associated with the same element in
the co-domain. Such a relation is also said to be \sphinxstyleemphasis{one-one-one},
rather than \sphinxstyleemphasis{many-to-one}).

Take a moment to think about the difference between being injective
and single valued. Single-valued means no \sphinxstyleemphasis{one} element of the domain
“goes to” {\color{red}\bfseries{}*}more than one” value in the range. Injective means that “no
more than one” value in the domain “goes to” and one value in the
range.

Exercise: Draw a picture. Draw the domain and range sets as clouds
with points inside, representing objects (values) in the domain and
co-domain. Represent a relation as a set of \sphinxstyleemphasis{arrows} that connect
domain objects to co-domain objects. The arrows visually depict the
ordered pairs in the relation. What does it look like visually for a
relation to be single-valued? What does it look like for a relation to
be injective?

The square function is a function because it is single-valued, but it
is not injective. To see this, observe that two different values in
the domain, \sphinxstyleemphasis{-2} and \sphinxstyleemphasis{2}, have the same value in the co-domain: \sphinxstyleemphasis{4}.
Think about the graph: if you can draw a \sphinxstyleemphasis{horizontal} line for any
value of \sphinxstyleemphasis{y} that intersects the graph at multiple points, then the
points at which it intersects correspond to different values of \sphinxstyleemphasis{x}
that have the same value \sphinxstyleemphasis{under the relation}. Such a relation is not
injective.

Exercises: Write a precise mathematical definition of what it means
for a binary relation to be injective.  Is the cube root function
injective? Is \sphinxstyleemphasis{f(x) = sin(x)} injective?


\subsubsection{An Aside: Injectivity in Type Theory}
\label{\detokenize{07-set-theory:an-aside-injectivity-in-type-theory}}
As an aside, we note that the concept of injectivity is essential in
\sphinxstyleemphasis{type theory}.  Whereas \sphinxstyleemphasis{set theory} provides a universally accepted
axiomatic foundation for mathematics, \sphinxstyleemphasis{type} theory is of increasing
interest as alternative foundation. It is also at the very heart of a
great deal of work in programming languages and software verification.

Type theory takes types rather than sets to be elementary. A type in
type theory comprises a collection of objects, just as a set does in
set theory. But whereas in set theory, an object can be in many sets,
in type theory, and object can have only one type.

The set of values of a given type is defined by a set of constants and
functions called constructors. Constant constructors define what one
can think of as the \sphinxstyleemphasis{smallest} values of a type, while constructors
that are functions provide means to build larger values of a type by
{\color{red}\bfseries{}*}packaging up” smaller values of the same and/or other types.

As a simple example, one might say that the set of values of the type,
\sphinxstyleemphasis{Russian Doll,} is given by one constant constructor, \sphinxstyleemphasis{SolidDoll} and
by one constructor function, \sphinxstyleemphasis{NestDoll} that takes a nested doll as an
argument (the solid one or any other one built by \sphinxstyleemphasis{NestDoll} itself).
Speaking intuitively, this constructor function does nothing other
than \sphinxstyleemphasis{package up} the smaller nest doll it was given inside a “box”
labelled \sphinxstyleemphasis{NestDoll}.  One can thus obtain a nested doll either as the
constant \sphinxstyleemphasis{SolidDoll} or by applying the \sphinxstyleemphasis{NestDoll} constructor some
finite number of times to smaller nested dolls. Such a nesting will
always be finitely deep, with the solid doll at the core.

A key idea in type theory is that \sphinxstyleemphasis{constructors are injective}. Two
values of a given type built by different constructors, or by the same
constructor with different arguments, are \sphinxstyleemphasis{always} different. So, for
example, the solid doll is by definition unequal to any doll built by
the \sphinxstyleemphasis{NestDoll} constructor; and a russian doll nested two levels deep
(built by applying \sphinxstyleemphasis{NestDoll} to an argument representing a doll that
is nested one level deep)is necessarily unequal to a russian doll one
level deep (built by applying \sphinxstyleemphasis{NestDoll} to the solid doll).

Running this inequality idea in reverse, we can conclude that if two
values of a given type are known to be equal, then for sure they were
constructed by the same constructor taking the same arguments (if
any).  It turns out that knowing such a fact, rooted in the
\sphinxstyleemphasis{injectivity of constructors} is often essential to completing proofs
about programs using type theory. But more on this later.


\subsection{Surjective}
\label{\detokenize{07-set-theory:surjective}}
A binary relation, and in particular a function, is \sphinxstyleemphasis{surjective} if
every element in the co-domain appears in some tuple in the relation.
A surjective relation is also said to be \sphinxstyleemphasis{onto}. The \sphinxstyleemphasis{range} of values
it achieves covers its whole co-domain. Mathematically, a relation
\(R \subseteq S \times T\) is surjective if \(\forall t \in
T, \exists s \in S~|~(s,t) \in R\).

In the intuitive terms of high school algebra, a relation involving
\sphinxstyleemphasis{x} and \sphinxstyleemphasis{y} is surjective if for any given \sphinxstyleemphasis{y} value there is always
some \sphinxstyleemphasis{x} that “leads to” that \sphinxstyleemphasis{y}. The \sphinxstyleemphasis{square} function on the real
numbers is not surjective, because there is no \sphinxstyleemphasis{x} that when squared
gets one to \sphinxstyleemphasis{y = -1}.

Exercise: Is the function, \sphinxstyleemphasis{f(x) = sin(x)}, from the real numbers (on
the x-axis) to real numbers (on the y-axis) surjective? How might you
phrase an informal but rigorous proof of your answer?

Exercise: Is the inverse of a surjective function always total? How
would you “prove” this with a rigorous, step-by-step argument based on
the definitions we’ve given here? Hint: It is almost always useful to
start with definitions. What does it mean for a relation to be total?
What does it mean for one relation to be the inverse of another? How
can you connect these definitons to show for sure that your answer is
right?


\subsection{Bijective}
\label{\detokenize{07-set-theory:bijective}}
A total function is said to be \sphinxstyleemphasis{bjective} if it is also both injective
and surjective. Such a function is also often called a \sphinxstyleemphasis{bijection}.

Take a moment to think about the implications of being a bijection.
Consider a bijective relation, \(R \subseteq S \times T.\) \sphinxstyleemphasis{R} is
total, so there is an \sphinxstyleemphasis{arrow} from every \sphinxstyleemphasis{s} in \sphinxstyleemphasis{S} to some \sphinxstyleemphasis{t} in
\sphinxstyleemphasis{T}.  \sphinxstyleemphasis{R} is injective, so no two arrows from any \sphinxstyleemphasis{s} in \sphinxstyleemphasis{s} ever hit
the same \sphinxstyleemphasis{t} in \sphinxstyleemphasis{T}. An injection is one-to-one. So there is exactly
one \sphinxstyleemphasis{t} in \sphinxstyleemphasis{T} hit by each \sphinxstyleemphasis{s} in \sphinxstyleemphasis{S}. But \sphinxstyleemphasis{R} is also surjective, so
every \sphinxstyleemphasis{t} in \sphinxstyleemphasis{T} is hit by some arrow from \sphinxstyleemphasis{S}. Therefore, there has
to be exactly one element in \sphinxstyleemphasis{t} for each element in \sphinxstyleemphasis{s}. So the sets
are of the same size, and there is a one-to-one correspondence between
their elements.

Now consider some \sphinxstyleemphasis{t} in \sphinxstyleemphasis{T}. It must be hit by exactly one arrow from
\sphinxstyleemphasis{S}, so the \sphinxstyleemphasis{inverse} relation, {\color{red}\bfseries{}:matg:{}`R\textasciicircum{}\{-1\}{}`}, from \sphinxstyleemphasis{T} to \sphinxstyleemphasis{S}, must
also single-valued (a function). Moreover, because \sphinxstyleemphasis{R} is surjective,
every \sphinxstyleemphasis{t} in \sphinxstyleemphasis{T} is hit by some \sphinxstyleemphasis{s} in \sphinxstyleemphasis{S}, so the inverse relation is
defined for every \sphinxstyleemphasis{t} in \sphinxstyleemphasis{T}. It, too, is total. Now every arrow from
any \sphinxstyleemphasis{s} to some \sphinxstyleemphasis{t} leads back from that \sphinxstyleemphasis{t} to that \sphinxstyleemphasis{s}, so the
inverse And it’s also (and because \sphinxstyleemphasis{R} is total, there is such an
arror for \sphinxstyleemphasis{every} \sphinxstyleemphasis{s} in \sphinxstyleemphasis{S}), the inverse relation is surjective (it
covers all of \sphinxstyleemphasis{S}).

Exercise: Must the inverse of a bijection be one-to-one? Why or why
not?  Make a rigorous argument based on assumptions derived from our
definitions.

Exercise: Must a bijective function be invertible? Make a rigorous
argument.

Exercise: What is the inverse of the inverse of a bijective function,
\sphinxstyleemphasis{R}. Prove it with a rigorous argument.

A bijection estabishes an invertible, one-to-one correspondence
between elements of two sets. Bijections can only be established
between sets of the same size. So if you want to prove that two sets
are of the same size, it sufficies to show that one can define a
bijection between the two sets. That is, one simply shows that there
is some function that covers each element in each set with arrows
connecting them, one-to-one in both directions.

Exercise: Prove that the number of non-negative integers (the
cardinality of \({\mathbb N}\)), is the same as the number of
non-negative fractions (the cardinality of \({\mathbb Q^{+}}\)).

Exercise: How many bijective relations are there between two sets of
cardinality \sphinxstyleemphasis{k}? Hint: Pick a first element in the first set. There
are \sphinxstyleemphasis{n} ways to map it to some element in the second set. Now for the
second element in the first set, there are only \sphinxstyleemphasis{(n-1)} ways to pair
it up with an element in the second set, as one cannot map it to the
element chosen in the first step (the result would not be injective).
Continue this line of reasoning until you get down to all elements
having been mapped.

Exercise: How many bijections are there from a set, \sphinxstyleemphasis{S}, to itself?
You can think of such a bijection as a simple kind of encryption. For
example, if you map each of the \sphinxstyleemphasis{26} letters of the alphabet to some
other letter, but in a way that is unambiguous (injective!), then you
have a simple encryption mechanisms. How many ways can you encrypt a
text that uses \sphinxstyleemphasis{26} letters in this way? Given a cyphertext, how would
you recover the original plaintext?

Exercise: If you encrypt a text in this manner, using a bijection,
\sphinxstyleemphasis{R} and then encrpty the resulting cyphertext using another one \sphinxstyleemphasis{T},
can you necessarily recover the plaintext? How? Is there a \sphinxstyleemphasis{single}
bijection that would have accomplished the same encryption result?
Would the inverse of that bijection effectively decrypt messages?

Exercise: Is the composition of any two bijections also a bijection?
If so, can you express its inverse in terms of the inverses of the two
component bijections?

Exercise: What is the \sphinxstyleemphasis{identity} bijection on the set of \sphinxstyleemphasis{26} letters?

Question: Are such bijections commutative? That is, you have two of
them, say \sphinxstyleemphasis{R} and \sphinxstyleemphasis{T}, is the bijection that you get by applying \sphinxstyleemphasis{R}
and then \sphinxstyleemphasis{T} the same as the bijection you get by applying \sphinxstyleemphasis{T} and
then \sphinxstyleemphasis{R}? If your answer is \sphinxstyleemphasis{no}, prove it by giving a counterexample
(e.g., involving bijections on a small set). If your answer is yes,
make rigorous argument.

Programming exercise: Implement encryption and decryption schemes in
Dafny using bijections over the \sphinxstyleemphasis{26} capital letters of the English
alphabet.

Programming exercise: Implement a \sphinxstyleemphasis{compose} function in Dafny that
takes two pure functions, \sphinxstyleemphasis{R} and \sphinxstyleemphasis{T}, each implementing a bijection
between the set of capital letters and that returns a pure function
that when applied has the effect of first applying \sphinxstyleemphasis{T} then applying
\sphinxstyleemphasis{R}.


\subsection{Reflexive}
\label{\detokenize{07-set-theory:reflexive}}
We now consider some additional fundamental properties of binary
relations, starting with relations that maps sets to themselves.  A
relation that maps real numbers to real numbers is a subset of
\({\mathbb R} \times {\mathbb R}\), for example. The \sphinxstyleemphasis{friends}
relation on a social network site associates people with people.

Such a relation is said to be \sphinxstyleemphasis{reflexive} if (perhaps among other
properties) it (at least) associates every element with itself.  The
equality relation (e.g., on real numbers) is the “canonical” example
of a reflexive relation. It associates every number with itself and
with no other number. The tuples of the equality relation on real
numbers thus includes \sphinxstyleemphasis{(2.5, 2.5)} and (-3.0, -3.0)* but definitely
not \sphinxstyleemphasis{(2.5, -3.0)}.

In more mathematical terms, consider a set \sphinxstyleemphasis{S} and a bindary relation,
\sphinxstyleemphasis{R}, on S*S, :math:R subseteq S times S.{}` \sphinxstyleemphasis{R} is reflexive, which we
can write as \sphinxstyleemphasis{Reflexive(R)}, if and only if for every \sphinxstyleemphasis{e} in \sphinxstyleemphasis{S}, the
tuple \sphinxstyleemphasis{(e,e)} is in R. Or to be rigorous about it, \(Reflexive(R)
\iff \forall e \in S, (e,e) \in R.\)

Exercise: Is the function, \sphinxstyleemphasis{y = x}, reflexive? If every person loves
themself, is the \sphinxstyleemphasis{loves} relation reflexive? Is the \sphinxstyleemphasis{less than or
equals} relation reflexive? Hint: the tuples \sphinxstyleemphasis{(2,3)} and \sphinxstyleemphasis{(3,3)} are
in this relation becaue \sphinxstyleemphasis{2} is less than or equal to \sphinxstyleemphasis{3}, and so is
\sphinxstyleemphasis{3}, but \sphinxstyleemphasis{(4,3)} is not in this relation, because \sphinxstyleemphasis{4} isn’t less than
or equal to \sphinxstyleemphasis{3}. Is the less than relation reflexive?


\subsection{Symmetric}
\label{\detokenize{07-set-theory:symmetric}}
A binary relation, \sphinxstyleemphasis{R}, on a set \sphinxstyleemphasis{S} is said to be \sphinxstyleemphasis{symmetric} if
whenever the tuple \sphinxstyleemphasis{(x,y)} is in \sphinxstyleemphasis{R}, the tuple, \sphinxstyleemphasis{(y,x)} is in \sphinxstyleemphasis{R} as
well. On Facebook, for example, if Joe is “friends” with “Tom” then
“Tom” is necessarily also friends with “Joe.” The Facebook friends
relation is thus symmetric in this sense.

More formally, if \sphinxstyleemphasis{R} is a binary relation on a set \sphinxstyleemphasis{S}, i.e., given
\(R \subseteq S \times S\), then \(Symmetric(R) \iff \forall
(x,y) \in R, (y,x) \in R\).

Question: is the function \sphinxstyleemphasis{y = x} symmetric? How about the \sphinxstyleemphasis{square}
function? In an electric circuit, if a conducting wire connects
terminal \sphinxstyleemphasis{T} to terminal \sphinxstyleemphasis{Q}, it also connects terminal \sphinxstyleemphasis{Q} to
terminal \sphinxstyleemphasis{T} in the sense that electricity doesn’t care which way it
flows over the wire. Is the \sphinxstyleemphasis{connects} relation in electronic circuits
symmetric? If \sphinxstyleemphasis{A} is \sphinxstyleemphasis{near} \sphinxstyleemphasis{B} then \sphinxstyleemphasis{B} is \sphinxstyleemphasis{near} \sphinxstyleemphasis{A}. Is \sphinxstyleemphasis{nearness}
symmetric? In the real work is the \sphinxstyleemphasis{has-crush-on} relation symmetric?


\subsection{Transitive}
\label{\detokenize{07-set-theory:transitive}}
Given a binary relation \(R \subseteq S \times S\), \sphinxstyleemphasis{R} is said to
be transitive if whenever \sphinxstyleemphasis{(x,y)} is in \sphinxstyleemphasis{R} and \sphinxstyleemphasis{(y,z)} is in \sphinxstyleemphasis{R},
then \sphinxstyleemphasis{(x,z)} is also in \sphinxstyleemphasis{R}. Formally, \(Transitive(R) \iff
forall (x,y) in R, \forall (y,z) \in R, (x,z) \in R\).

Exercise: Is equality transitive? That is, if \sphinxstyleemphasis{a = b} and \sphinxstyleemphasis{b = c} it
is also necessarily the case that \sphinxstyleemphasis{a = c}? Answer: Sure, any sensible
notion of an equality relation has this transitivity property.

Exercise: What about the property of being less than? If \sphinxstyleemphasis{a \textless{} b} and
\sphinxstyleemphasis{b \textless{} c} is it necessarily the case that \sphinxstyleemphasis{a \textless{} c}? Answer: again,
yes. The less than, as well as the less than or equal, and greater
then, and the greater than or equal relations, are all transitive.

How about the \sphinxstyleemphasis{likes} relation amongst real people. If Harry likes
Sally and Sally likes Bob does Harry necesarily like Bob, too? No, the
human “likes” relation is definitely not transitive. (And this is the
cause of many a tragedy.)


\subsection{Equivalence}
\label{\detokenize{07-set-theory:equivalence}}
Finally (for now), a relation is said to be an \sphinxstyleemphasis{equivalence relation}
if it is reflexive, transitive, and symmetric. Formally, we can write
this property as a conjunction of the three individual properties:
\(Equivalence(R) \iff Symmetric(R) \land Reflexive(R) \land
Transitive(R)\). Equality is the canonical example of an equivalence
relation: it is reflexive (\sphinxstyleemphasis{x = x}), symmetric (if \sphinxstyleemphasis{x = y} then \sphinxstyleemphasis{y =
x}) and transitive (if \sphinxstyleemphasis{x = y} and \sphinxstyleemphasis{y = z} then \sphinxstyleemphasis{x = z}.).

An important property of equivalence relations is that they divide up
a set into subsets of \sphinxstyleemphasis{equivalent} values. As an example, take the
equivalence relation on people, \sphinxstyleemphasis{has same birthday as}. Clearly every
person has the same birthday as him or herself; if Joe has the same
birthday as Mary, then Mary has the same birthday as Joe; and if Tom
has the same birthday as mary then Joe necessarily also has the same
birthday as Tom. This relation thus divides the human population into
366 equivalence classes. Mathematicians usually use the notation \sphinxstyleemphasis{a \textasciitilde{}
b} to denote the concept that \sphinxstyleemphasis{a} is equivalent to \sphinxstyleemphasis{b} (under whatever
equivalence relation is being considered).


\section{Sequences}
\label{\detokenize{07-set-theory:sequences}}
A sequence of elements is an ordered collection in which elements can
appear zero or more times. In both mathematical writing and in Dafny,
sequences are often denoted as lists of elements enclosed in square
brackets.  The same kinds of elisions (using elipses) can be used as
shorthands in quasi-formal mathematical writing as with set notation.
For example, in Dafny, a sequence \sphinxstyleemphasis{s := {[}1, 2, 3, 1{]}} is a sequence of
integers, of length four, the elements of which can be referred to by
subscripting. So \sphinxstyleemphasis{s{[}0{]}} is \sphinxstyleemphasis{1}, for example, as is \sphinxstyleemphasis{s{[}3{]}}.

While at first a sequence might seem like an entirely different kind
of thing than a set, in reality a sequence of length, \sphinxstyleemphasis{n}, is best
understood, and is formalized, as a binary relation. The domain of the
relation is the sequence of natural numbers from \sphinxstyleemphasis{0} to \sphinxstyleemphasis{n-1}.  These
are the index values. The relation then associates each such index
value with the value in that position in the sequence. So in reality,
a sequence is a special case of a binary relation, and a binary
relation is, as we’ve seen, just a special case of a set.  So here we
are, at the end of this chapter, closing the loop with where we
started. We have seen that the concept of sets really is a fundamental
concept, and a great deal of other machinery is then built as using
special cases, including relations, maps, and sequences.

Tuples, too, are basically maps from indices to values. Whereas all
the values in a sequence are necessarily of the same type, elements in
a tuple can be of different types. Tuples also use the \sphinxstyleemphasis{.n} notation
to apply projection functions to tuples. So, again, the value of, say,
\sphinxstyleemphasis{(“hello”, 7).1} is \sphinxstyleemphasis{7} (of type \sphinxstyleemphasis{int}), while the value of
\sphinxstyleemphasis{(“hello”, 7).0} is the string, “hello.”


\chapter{8. Boolean Algebra}
\label{\detokenize{08-boolean-algebra::doc}}\label{\detokenize{08-boolean-algebra:boolean-algebra}}
As a first stepping stone toward a deeper exploration of deductive
logic, we explore the related notion of Boolean \sphinxstyleemphasis{algebra}. Boolean
algebra is a mathematical framework for representing and reasoning
about truth.

This algebra is akin to ordinary high school algebra, and as such,
deals with values, operators, and the syntax and the evaluation of
expressions involving values and operators.  However, the values in
Boolean algebra are limited to the two values in the set, \(bool
= \{ 0, 1\}\). They are often written instead as \sphinxstyleemphasis{false} and \sphinxstyleemphasis{true},
respectively. And rather than arithmetic operators such as numeric
negation, addition, and subtraction, Boolean algebra defines a set of
\sphinxstyleemphasis{Boolean operators}. They are typically given names such as \sphinxstyleemphasis{and},
\sphinxstyleemphasis{or}, and \sphinxstyleemphasis{not}, and they both operate on and yield Boolean values.

In this chapter, we first discuss Boolean algebra in programming, a
setting with which the reader is already familar, baesd on a first
course in programming. We then take a deeper look at the syntax and
semantics of \sphinxstyleemphasis{expressions} in Boolean algebra. We do this by seeing
how to use \sphinxstyleemphasis{inductive definitions} and \sphinxstyleemphasis{recursive functions} in the
Dafny language to implement an \sphinxstyleemphasis{inductive data type} for representing
Boolean expressions and a recursive \sphinxstyleemphasis{evaluation} function that when
given any Boolean expression tells whether it is \sphinxstyleemphasis{true} or \sphinxstyleemphasis{false}.


\section{Boolean Algebra in Dafny}
\label{\detokenize{08-boolean-algebra:boolean-algebra-in-dafny}}
All general-purpose programming languages support Boolean
algebra. Dafny does so through its \sphinxstyleemphasis{bool} data type and the
\sphinxstyleemphasis{operators} associated with it. Having taking a programming course,
you will already have been exposed to all of the important ideas.
In Dafny, as in many languages, the Boolean values are called
\sphinxstyleemphasis{true} and \sphinxstyleemphasis{false} (rather than \sphinxstyleemphasis{1} and \sphinxstyleemphasis{0}).

The Boolean operators are also denoted not by words, such as \sphinxstyleemphasis{or} and
\sphinxstyleemphasis{not} but by math-like operators. For example, \sphinxstyleemphasis{!} is the not operator
and \sphinxstyleemphasis{\textbar{}\textbar{}} is the \sphinxstyleemphasis{or} operator.

Here’s a (useless) Dafny method that illustrates how Boolean values
and operators can be used in Dafny. It presents a method, \$BoolOps\$,
that takes a Boolean value and returns one. The commands within the
method body illustrate the use of Boolean constant (literal) values
and the unary and binary operators provided by the Dafny language.

\begin{sphinxVerbatim}[commandchars=\\\{\}]
method BoolOps(a: bool) returns (r: bool)
\PYGZob{}
    var t: bool := true;    // explicit type declaration
    var f := false;         // type inferred automatically
    var not := !t;          // negation
    var conj := t \PYGZam{}\PYGZam{} f;     // conjunction, short\PYGZhy{}circuit evaluation
    var disj := t \textbar{}\textbar{} f;     // disjunction, short\PYGZhy{}circuit (sc) evaluation
    var impl := t ==\PYGZgt{} f;    // implication, right associative, sc from left
    var foll := t \PYGZlt{}== f;    // follows, left associative, sc from right
    var equv := t \PYGZlt{}==\PYGZgt{} t;   // iff, bi\PYGZhy{}implication
    return true;            // returning a Boolean value
 \PYGZcb{}
\end{sphinxVerbatim}

The first line assigns the Boolean constant, \sphinxstyleemphasis{true}, to a Boolean
variable, \sphinxstyleemphasis{t}, that is explicitly declared to be of type,, \sphinxstyleemphasis{bool}.
The second line assigns the Boolean constant, \sphinxstyleemphasis{false}, to \sphinxstyleemphasis{f}, and
allows Dafny to infer that the type of \sphinxstyleemphasis{f} must be \sphinxstyleemphasis{bool}, based on
the type of value being assigned to it. The third line illustrates the
use of the \sphinxstyleemphasis{negation} operator, denoted as \sphinxstyleemphasis{!} in Dafny. Here the
negation of \sphinxstyleemphasis{t} is assigned to the new Boolean variable, \sphinxstyleemphasis{not}. The
next line illustrates the use of the Boolean \sphinxstyleemphasis{and}, or \sphinxstyleemphasis{conjunction}
operator (\sphinxstyleemphasis{\&\&}). Next is the Boolean \sphinxstyleemphasis{or}, or \sphinxstyleemphasis{disjunction}, operator,
(\sphinxstyleemphasis{\textbar{}\textbar{}}). These should all be familiar.

Implication (\sphinxstyleemphasis{==\textgreater{}}) is a binary operator (taking two Boolean values)
that is read as \sphinxstyleemphasis{implies} and that evaluates to false only when the
first argument is true and the second one is false, and that evaluates
to true otherwise. The \sphinxstyleemphasis{follows} operator (\sphinxstyleemphasis{\textless{}==}) swaps the order of
the arguments, and evaluates to false if the first argument is false
and the second is true, and evaluates to true otherwise. Finally, the
\sphinxstyleemphasis{equivalence} operator evaluates to true if both arguments have the
same Boolean value, and evaluates to false otherwise. These operators
are especially useful in writing assertions in Dafny.

The last line returns the Boolean value true as the result of running
this method. Other operations built into Dafny also return Boolean
values.  Arithmetic comparison operators, such as \sphinxstyleemphasis{\textless{}}, are examples.
The less than operator, for example, takes two numerical arguments and
returns true if the first is strictly less than the second, otherwise
it returns false.


\section{Boolean Values}
\label{\detokenize{08-boolean-algebra:boolean-values}}
Boolean algebra is an algebra, which is a set of values and of
operations that take and return these values. The set of values in
Boolean algebra, is just the set containing \sphinxstyleemphasis{0} and \sphinxstyleemphasis{1}.
\begin{equation*}
\begin{split}bool = \{ 0, 1 \}.\end{split}
\end{equation*}
In English that expression just gave a name that we can use, \sphinxstyleemphasis{bool},
to the set containing the values, \sphinxstyleemphasis{0} and \sphinxstyleemphasis{1}. Although these values
are written as if they were small natural numbers, you must think of
them as elements of a different type. They aren’t natural numbers but
simply the two values in this other, Boolean, algebra. We could use
different symbols to represent these values. In fact, they are often
written instead as \sphinxstyleemphasis{false} (for \sphinxstyleemphasis{0}) and \sphinxstyleemphasis{true} (for \sphinxstyleemphasis{1}).The exact
symbols we use to represent these values don’t really matter. What
really makes Boolean algebra what it is are the \sphinxstyleemphasis{operators} defined
by Boolean algebra and how they behave.


\section{Boolean Operators}
\label{\detokenize{08-boolean-algebra:boolean-operators}}
An algebra, again, is a set of values of a particular kind and a set
of operators involving that kind of value. Having introduced the set
of two values of the Boolean type, let’s turn to the \sphinxstyleemphasis{operations} of
Boolean algebra.


\subsection{Nullary, Unary, Binary, and n-Ary Operators}
\label{\detokenize{08-boolean-algebra:nullary-unary-binary-and-n-ary-operators}}
The operations of an algebra take zero or more values and return (or
reduce to) values of the same kind. Boolean operators, for example,
take zero or more Boolean values and reduce to Boolean values. An
operator that takes no values (and nevertheless returns a value, as
all operators do) is called a \sphinxstyleemphasis{constant}. Each value in the value set
of an algebra can be though of as an operator that takes no values.

Such an operator is also called \sphinxstyleemphasis{nullary}. An operator that takes one
value is called \sphinxstyleemphasis{unary}; one that takes two, \sphinxstyleemphasis{binary}, and in general,
one that takes \sphinxstyleemphasis{n} arguments is called \sphinxstyleemphasis{n-ary} (pronounced “EN-airy”).

Having already introduced the constant (\sphinxstyleemphasis{nullary}) values of Boolean
algebra, each of the type we have called \sphinxstyleemphasis{bool}, we now introduce the
types and behaviors the unary and binary Boolean operators, including
each of those supported in Dafny.


\subsection{The Unary Operators of Boolean Algebra}
\label{\detokenize{08-boolean-algebra:the-unary-operators-of-boolean-algebra}}
While there are two constants in Boolean algebra, each of type \sphinxstyleemphasis{bool},
there are four unary operators, each of type \(bool \rightarrow
bool\). This type, which contains an arrow, is a \sphinxstyleemphasis{function} type. It is
the type of any function that first takes an argument of type \sphinxstyleemphasis{bool}
then reduces to a value of type \sphinxstyleemphasis{bool}. It’s easier to read, write,
and say in math than in English. In math, the type would be prounced
as “bool to bool.”

There is more than one value of this function type. For example one
such function takes any \sphinxstyleemphasis{bool} argument and always returns the other
one. This function is of type “bool to bool”, but it is not the same
as the function that takes any bool argument and always returns the
same value that it got. The type of each function is \(bool
\rightarrow bool\), but the function \sphinxstyleemphasis{values} are different.

In the programming field, the type of a function is given when it
name, its arguments, and return values are declared. This part of a
function definition is sometimes called the function \sphinxstyleemphasis{signature}, but
it’s just as well to think of it as decaring the function \sphinxstyleemphasis{type}.  The
\sphinxstyleemphasis{body} of the function, usually a sequence of commands enclosed in
curly braces, describes its actual behavior, the particular function
value associated with the given function name and type.

We know that there is more than one unary Boolean function. So how
many are there? To specify the behavior of an operator completely, we
have to define what result it returns for each possible combination of
its argument values. A unary operator takes only one argument (of the
given type). In Boolean algebra, a unary function can thus take one of
only two possible values; and it can return only one of two possible
result values. The answer to the question is just the number of ways
that a function can \sphinxstyleemphasis{map} two argument values to two result values.

And the answer to this question is \sphinxstyleemphasis{four}. A function can map both \sphinxstyleemphasis{0}
and \sphinxstyleemphasis{1} to \sphinxstyleemphasis{0}; both \sphinxstyleemphasis{0} and \sphinxstyleemphasis{1} to \sphinxstyleemphasis{1}; \sphinxstyleemphasis{0} to \sphinxstyleemphasis{0} and \sphinxstyleemphasis{1} to \sphinxstyleemphasis{1};
and \sphinxstyleemphasis{0} to \sphinxstyleemphasis{1} and \sphinxstyleemphasis{1} to \sphinxstyleemphasis{0}. There are no other possibilities. An
easy-to-understand way to graphically represent the behavior of each
of these operations is with a \sphinxstyleemphasis{truth table}.

The rows of a truth table depict all possible combinations of argument
values in the columns to the left, and in the last column on the right
a truth tables presents the corresponding resulting value.  The column
headers give names to the argument values and results column headers
present expressions using mathematical logic notations that represent
how the resulting values are computed.


\subsubsection{Constant False}
\label{\detokenize{08-boolean-algebra:constant-false}}
Here then is a truth table for what we will call the \sphinxstyleemphasis{constant\_false}
operator, which takes a Boolean argument, either \sphinxstyleemphasis{true} or \sphinxstyleemphasis{false},
and always returns \sphinxstyleemphasis{false.} In our truth tables, we use the symbols,
\sphinxstyleemphasis{true} and \sphinxstyleemphasis{false}, instead of \sphinxstyleemphasis{1} and \sphinxstyleemphasis{0}, for consistency with the
symbols that most programming languages, including Dafny, use for the
Boolean constants.


\begin{savenotes}\sphinxattablestart
\centering
\begin{tabular}[t]{|\X{6}{12}|\X{6}{12}|}
\hline

\(P\)
&
\(false\)
\\
\hline
true
&
false
\\
\hline
false
&
false
\\
\hline
\end{tabular}
\par
\sphinxattableend\end{savenotes}


\subsubsection{Constant True}
\label{\detokenize{08-boolean-algebra:constant-true}}
The \sphinxstyleemphasis{constant\_true} operator always returns \sphinxstyleemphasis{true}.


\begin{savenotes}\sphinxattablestart
\centering
\begin{tabular}[t]{|\X{6}{12}|\X{6}{12}|}
\hline

\(P\)
&
\(true\)
\\
\hline
true
&
true
\\
\hline
false
&
true
\\
\hline
\end{tabular}
\par
\sphinxattableend\end{savenotes}


\subsubsection{Identity Function(s)}
\label{\detokenize{08-boolean-algebra:identity-function-s}}
The Boolean \sphinxstyleemphasis{identity} function takes one Boolean value as an argument
and returns that value, whichever it was.


\begin{savenotes}\sphinxattablestart
\centering
\begin{tabular}[t]{|\X{6}{12}|\X{6}{12}|}
\hline

\(P\)
&
\(P\)
\\
\hline
true
&
true
\\
\hline
false
&
false
\\
\hline
\end{tabular}
\par
\sphinxattableend\end{savenotes}

As an aside we will note that \sphinxstyleemphasis{identity functions} taking any type of
value are functions that always return exactly the value they took as
an argument. What we want to say is that “for any type, \sphinxstyleemphasis{T}, and any
value, \sphinxstyleemphasis{t} of that type, the identity function for type \sphinxstyleemphasis{T} applied to
\sphinxstyleemphasis{t} always returns \sphinxstyleemphasis{t} itself. In mathematical logical notation,
\(\forall T: Type, \forall t: T, id_T(t) = t.\) It’s clearer in
mathematical language than in English! Make sure that both make sense
to you now. That is the end of our aside. Now back to Boolean algebra.


\subsubsection{Negation}
\label{\detokenize{08-boolean-algebra:negation}}
The Boolean negation, or \sphinxstyleemphasis{not}, operator, is the last of the four
unary operators on Boolean values. It returns the value that it was
\sphinxstyleemphasis{not} given as an argument. If given \sphinxstyleemphasis{true}, it evaluates to \sphinxstyleemphasis{false},
and if given \sphinxstyleemphasis{false}, to \sphinxstyleemphasis{true.}

The truth table makes this behavior clear.  It also introduces the
standard notation in mathematical logic for the negation operator,
\(\neg P\). This expression is pronounced, \sphinxstyleemphasis{not P}. It evaluates
to \sphinxstyleemphasis{true} if \sphinxstyleemphasis{P} is false, and to \sphinxstyleemphasis{false} if \sphinxstyleemphasis{P} is \sphinxstyleemphasis{true}.


\begin{savenotes}\sphinxattablestart
\centering
\begin{tabular}[t]{|\X{6}{12}|\X{6}{12}|}
\hline

\(P\)
&
\(\neg P\)
\\
\hline
true
&
false
\\
\hline
false
&
true
\\
\hline&\\
\hline
\end{tabular}
\par
\sphinxattableend\end{savenotes}


\subsection{Binary Boolean Operators}
\label{\detokenize{08-boolean-algebra:binary-boolean-operators}}
Now let’s consider the binary operators of Boolean algebra. Each takes
two Boolean arguments and returns a Boolean value as a result. The
type of each such function is written \(bool \rightarrow bool
\rightarrow bool\), pronounced “bool to bool to bool.” A truth table
for a binary Boolean operator will have two columns for arguments, and
one on the right for the result of applying the operator being defined
to the argument values in the left two columns.

Because binary Boolean operators take two arguments, each with two
possible values, there is a total of four possible combinations of
argument values: \sphinxstyleemphasis{true} and \sphinxstyleemphasis{true}, \sphinxstyleemphasis{true} and \sphinxstyleemphasis{false}, \sphinxstyleemphasis{false} and
\sphinxstyleemphasis{true}, and \sphinxstyleemphasis{false} and \sphinxstyleemphasis{false}. A truth table for a binary operator
will thus have four rows.

The rightmost column of a truth table for an operator is really where
the action is. It defines what result is returned for each combination
of argument values. In a table with four rows, there will be four
cells to fill in the final column. In a Boolean algebra there are two
ways to fill each cell. And there are exactly \sphinxstyleemphasis{12\textasciicircum{}4 = 6} ways to do
that. We can write them as \sphinxstyleemphasis{0000, 0001, 0010, 0011, 0100, 0101, 0110,
0111, 1000, 1001, 1010, 1011, 1100, 1101, 1110, 1111}. There are thus
exactly \sphinxstyleemphasis{16} total binary operators in Boolean algebra.

Mathematicians have given names to all \sphinxstyleemphasis{16}, but in practice we tend
to use just a few of them. They are called \sphinxstyleemphasis{and}, \sphinxstyleemphasis{or}, and \sphinxstyleemphasis{not}. The
rest can be expressed as combinations these operators.  It is common
in computer science also to use binary operations called \sphinxstyleemphasis{nand} (for
\sphinxstyleemphasis{not and}), \sphinxstyleemphasis{xor} (for \sphinxstyleemphasis{exclusive or}) and \sphinxstyleemphasis{implies}.  Here we present
truth tables for each of the binary Boolean operators in Dafny.


\subsubsection{And (conjunction)}
\label{\detokenize{08-boolean-algebra:and-conjunction}}
The \sphinxstyleemphasis{and} operator in Boolean algebra takes two Boolean arguments and
returns \sphinxstyleemphasis{true} when both arguments are \sphinxstyleemphasis{true}, and otherwise, \sphinxstyleemphasis{false}.


\begin{savenotes}\sphinxattablestart
\centering
\begin{tabular}[t]{|\X{6}{18}|\X{6}{18}|\X{6}{18}|}
\hline

\(P\)
&
\(Q\)
&
\(P \land Q\)
\\
\hline
true
&
true
&
true
\\
\hline
true
&
false
&
false
\\
\hline
false
&
true
&
false
\\
\hline
false
&
false
&
false
\\
\hline
\end{tabular}
\par
\sphinxattableend\end{savenotes}


\subsubsection{Nand (not and)}
\label{\detokenize{08-boolean-algebra:nand-not-and}}
The \sphinxstyleemphasis{nand} operator, short for \sphinxstyleemphasis{not and}, returns the opposite value
from the \sphinxstyleemphasis{and} operator: \sphinxstyleemphasis{false} if both arguments are \sphinxstyleemphasis{true} and
\sphinxstyleemphasis{true} otherwise.


\begin{savenotes}\sphinxattablestart
\centering
\begin{tabular}[t]{|\X{6}{18}|\X{6}{18}|\X{6}{18}|}
\hline

\(P\)
&
\(Q\)
&
\(P \uparrow Q\)
\\
\hline
true
&
true
&
false
\\
\hline
true
&
false
&
true
\\
\hline
false
&
true
&
true
\\
\hline
false
&
false
&
true
\\
\hline
\end{tabular}
\par
\sphinxattableend\end{savenotes}

As an aside, the \sphinxstyleemphasis{nand} operator is especially important for designers
of digital logic circuits. The reason is that \sphinxstyleemphasis{every} binary Boolean
operator can be simulated by composing \sphinxstyleemphasis{nand} operations in certain
patterns. So if we have a billion tiny \sphinxstyleemphasis{nand} circuits (each with two
electrical inputs and an output that is off only when both inputs are
on), then all we have to do is connect all these little ciruits up in
the right patterns to implement very complex Boolean functions. The
capability to etch billions of tiny \sphinxstyleemphasis{nand} circuits in silicon and to
connect them in complex ways is the heart of the computer revolution.
Now back to Boolean algebra.


\subsubsection{Or (disjunction)}
\label{\detokenize{08-boolean-algebra:or-disjunction}}
The \sphinxstyleemphasis{or}, or \sphinxstyleemphasis{disjunction}, operator evaluates to \sphinxstyleemphasis{false} only if both
arguments are \sphinxstyleemphasis{false}, and otherwise to \sphinxstyleemphasis{true}.

It’s important to note that it evaluates to \sphinxstyleemphasis{true} if either one or
both of its arguments are true. When a dad says to his child, “You can
have a candy bar \sphinxstyleemphasis{or} a donut, \sphinxstyleemphasis{he likely doesn’t mean *or} in the
sense of \sphinxstyleemphasis{disjunction}.  Otherwise the child well educated in logic
would surely say, “Thank you, Dad, I’ll greatly enjoy having both.”


\begin{savenotes}\sphinxattablestart
\centering
\begin{tabular}[t]{|\X{6}{18}|\X{6}{18}|\X{6}{18}|}
\hline

\(P\)
&
\(Q\)
&
\(P \lor Q\)
\\
\hline
true
&
true
&
true
\\
\hline
true
&
false
&
true
\\
\hline
false
&
true
&
true
\\
\hline
false
&
false
&
false
\\
\hline
\end{tabular}
\par
\sphinxattableend\end{savenotes}


\subsubsection{Xor (exclusive or)}
\label{\detokenize{08-boolean-algebra:xor-exclusive-or}}
What the dad most likely meant by \sphinxstyleemphasis{or} is what in Boolean algebra we
call \sphinxstyleemphasis{exclusive or}, written as \sphinxstyleemphasis{xor}.  It evalutes to true if either
one, but \sphinxstyleemphasis{not both}, of its arguments is true, and to false otherwise.


\begin{savenotes}\sphinxattablestart
\centering
\begin{tabular}[t]{|\X{6}{18}|\X{6}{18}|\X{6}{18}|}
\hline

\(P\)
&
\(Q\)
&
\(P \oplus Q\)
\\
\hline
true
&
true
&
false
\\
\hline
true
&
false
&
true
\\
\hline
false
&
true
&
true
\\
\hline
false
&
false
&
false
\\
\hline
\end{tabular}
\par
\sphinxattableend\end{savenotes}


\subsubsection{Nor (not or)}
\label{\detokenize{08-boolean-algebra:nor-not-or}}
The \sphinxstyleemphasis{nor} operator returns the negation of what the \sphinxstyleemphasis{or} operator
applied to the same arguments returns: \sphinxstyleemphasis{xor(b1,b2) = not(or(b1, b2))}.
As an aside, like \sphinxstyleemphasis{nand}, the \sphinxstyleemphasis{nor} operator is \sphinxstyleemphasis{universal}, in the
sense that it can be composed to with itself in different patterns to
simulate the effects of any other binary Boolean operator.


\begin{savenotes}\sphinxattablestart
\centering
\begin{tabular}[t]{|\X{6}{18}|\X{6}{18}|\X{6}{18}|}
\hline

\(P\)
&
\(Q\)
&
\(P \downarrow Q\)
\\
\hline
true
&
true
&
false
\\
\hline
true
&
false
&
false
\\
\hline
false
&
true
&
false
\\
\hline
false
&
false
&
true
\\
\hline
\end{tabular}
\par
\sphinxattableend\end{savenotes}


\subsubsection{Implies}
\label{\detokenize{08-boolean-algebra:implies}}
The \sphinxstyleemphasis{implies} operator is used to express the idea that if one
condition, a premise, is true, another one, the conclusion, must be.
So this operator returns true when both arguments are true. If the
first argument is false, this operator returns true. It returns false
only in the case where the first argument is true and the second is
not, because that violates the idea that if the first is true then the
second must be.


\begin{savenotes}\sphinxattablestart
\centering
\begin{tabular}[t]{|\X{6}{18}|\X{6}{18}|\X{6}{18}|}
\hline

\(P\)
&
\(Q\)
&
\(P \rightarrow Q\)
\\
\hline
true
&
true
&
true
\\
\hline
true
&
false
&
false
\\
\hline
false
&
true
&
true
\\
\hline
false
&
false
&
true
\\
\hline
\end{tabular}
\par
\sphinxattableend\end{savenotes}


\subsubsection{Follows}
\label{\detokenize{08-boolean-algebra:follows}}
The \sphinxstyleemphasis{follows} operator reverses the sense of an implication. Rather
than being understood to say that truth of the first argument should
\sphinxstyleemphasis{lead to} the truth of the second, it says that the truth of the first
should \sphinxstyleemphasis{follow from} the truth of the second.


\begin{savenotes}\sphinxattablestart
\centering
\begin{tabular}[t]{|\X{6}{18}|\X{6}{18}|\X{6}{18}|}
\hline

\(P\)
&
\(Q\)
&
\(P \leftarrow Q\)
\\
\hline
true
&
true
&
true
\\
\hline
true
&
false
&
true
\\
\hline
false
&
true
&
false
\\
\hline
false
&
false
&
true
\\
\hline
\end{tabular}
\par
\sphinxattableend\end{savenotes}

There are other binary Boolean operators. They even have names, though
one rarely sees these names used in practice.


\subsection{A Ternary Binary Operator}
\label{\detokenize{08-boolean-algebra:a-ternary-binary-operator}}
We can of course define Boolean operators of any arity. As just one
example, we introduce a \sphinxstyleemphasis{ternary} (3-ary) Boolean operator. It takes
three Boolean values as arguments and returns a Boolean result. It’s
type is thus ::\sphinxtitleref{bool rightarrow bool rightarrow bool rightarrow
bool}. We will call it \sphinxstyleemphasis{ifThenElse\_\{bool\}}.

The way this operator works is that the value of the first argument
determines which of the next two arguments values the function will
return. If the first argument is \sphinxstyleemphasis{true} then the value of the whole
expression is the value of the second argument, otherwise it is the
value of the third. So, for example, \sphinxstyleemphasis{ifThenElse\_\{bool\}(true, true,
false)} evaluates to true, while \sphinxstyleemphasis{ifThenElse\_\{bool\}(false, true,
false)} is false.

It is sometimes helpful to write Boolean expressions involving \sphinxstyleemphasis{n-ary}
operators for \sphinxstyleemphasis{n\textgreater{}1} using something other than function application
(prefix) notation. So, rather than \sphinxstyleemphasis{and(true,false)}, with the
operator in front of the arguments (\sphinxstyleemphasis{prefix} notation), we would
typically write \sphinxstyleemphasis{true \&\& false} to mean the same thing. We have first
sed a symbol, \sphinxstyleemphasis{\&\&}, instead of the English word, \sphinxstyleemphasis{and} to name the
operator of interest. We have also put the function name (now \sphinxstyleemphasis{\&\&})
\sphinxstyleemphasis{between} the arguments rather than in front of them. This is called
\sphinxstyleemphasis{infix} notation.

With ternary and other operators, it can even make sense to break up
the name of the operator and spread its parts across the whole
expression. For example, instead of writing, \sphinxstyleemphasis{ifThenElse\_\{bool\}(true,
true, false)}, we could write it as \sphinxstyleemphasis{IF true THEN true ELSE false.}
Here, the capitalized words all together represent the name of the
function applied to the three Boolean arguments in the expression.

As an aside, when we use infix notation, we have to do some extra
work, namely to specify the \sphinxstyleemphasis{order of operations}, so that when we
write expressions, the meaning is unambiguous. We have to say which
operators have higher and lower \sphinxstyleemphasis{precedence}, and whether operators
are \sphinxstyleemphasis{left}, \sphinxstyleemphasis{right}, or not associative. In everyday arithmetic, for
example, multiplication has higher precedence than addition, so the
expression \sphinxstyleemphasis{3 + 4 * 5} is read as \sphinxstyleemphasis{3 + (4 * 5)} even though the \sphinxstyleemphasis{+}
operator comes first in the expression.

Exercise: How many ternary Boolean operations are there? Hint: for an
operator with \sphinxstyleemphasis{n} Boolean arguments there are \(2^n\) combinations
of input values. This means that there will be \(2^n\) rows in its
truth table, and so \(2^n\) blanks to fill in with Boolean values
in the right column. How many ways are there to fill in \(2^n\)
Boolean values? Express your answer in terms of \sphinxstyleemphasis{n}.

Exercise: Write down the truth table for our Boolean if-then-else
operator.


\section{Formal Languages: Syntax and Semantics}
\label{\detokenize{08-boolean-algebra:formal-languages-syntax-and-semantics}}
Any introduction to programming will have made it clear that there is
an infinite set of Boolean expressions. For example, in Dafny, \sphinxstyleemphasis{true}
is a Boolean expression; so are \sphinxstyleemphasis{false}, \sphinxstyleemphasis{true \textbar{}\textbar{} false}, \sphinxstyleemphasis{(true \textbar{}\textbar{}
false) \&\& (!false)}, and one could keep going on forever.

Boolean \sphinxstyleemphasis{expressions}, as we see here, are a different kind of thing
than Boolean \sphinxstyleemphasis{values}. There are only two Boolean values, but there is
an infinity of Boolean expressions. The connection is that each such
expression has a corresponding Boolean truth value. For example, the
expression, \sphinxstyleemphasis{(true \textbar{}\textbar{} false) \&\& (!false)} has the value, \sphinxstyleemphasis{true}.

The set of valid Boolean expressions is defined by the \sphinxstyleemphasis{syntax} of the
Boolean expression language. The sequence of symbols, \sphinxstyleemphasis{(true \textbar{}\textbar{} false)
\&\& (!false)}, is a valid expression in the language, for example, but
\sphinxstyleemphasis{)( true false()\textbar{}\textbar{}) false !\&\&} is not, just as the sequence of words,
“Mary works long hours” is a valid sentence in the English language,
but “long works hours Mary” isn’t.

The syntax of a language defines the set of valid sentences in the
language. The semantics of a language gives a meaning to each valid
sentence in the language. In the case of Boolean expressions, the
meaning given to each valid “sentence” (expression) is simply the
Boolean value that that expression \sphinxstyleemphasis{reduces to}.

In the rest of this chapter, we use the case of Boolean expressions to
introduce the concepts of the \sphinxstyleemphasis{syntax} and the \sphinxstyleemphasis{semantics} of \sphinxstyleemphasis{formal
languages}. The syntax of a formal language precisely defines a set of
\sphinxstyleemphasis{expressions} (sometimes called sentences or formulae). A \sphinxstyleemphasis{semantics}
associates a \sphinxstyleemphasis{meaning}, in the form of a \sphinxstyleemphasis{value}, with each expression
in the language.


\section{The Syntax of Boolean Expressions: Inductive Definitions}
\label{\detokenize{08-boolean-algebra:the-syntax-of-boolean-expressions-inductive-definitions}}
As an example of syntax, the \sphinxstyleemphasis{true}, in the statement, \sphinxstyleemphasis{var b :=
true;} is a valid expression in the language of Boolean expressions,
as defined by the \sphinxstyleemphasis{syntaxt} of this language. The semantics of the
language associates the Boolean \sphinxstyleemphasis{value}, \sphinxstyleemphasis{true}, with this expression.

You probably just noticed that we used the same symbol, \sphinxstyleemphasis{true}, for
both an expression and a value, blurring the distinction between
expressions and values. Expressions that directly represent values are
called \sphinxstyleemphasis{literal expressions}. Many languages use the usual name for a
value as a literal expression, and the semantics of the language then
associate each such expression with its corresponding value.

In the semantics of practical formal languages, literal expressions
are assigned the values that they name. So the \sphinxstyleemphasis{expression}, \sphinxstyleemphasis{true},
means the \sphinxstyleemphasis{value}, \sphinxstyleemphasis{true}, for example. Similarly, when \sphinxstyleemphasis{3} appears on
the right side of an assignment/update statement, such as in \sphinxstyleemphasis{x := 3},
it is an \sphinxstyleemphasis{expression}, a literal expression, that when \sphinxstyleemphasis{evaluated} is
taken to \sphinxstyleemphasis{mean} the natural number (that we usually represent as) \sphinxstyleemphasis{3}.

As computer scientists interested in languages and meaning, we can
make these concepts of syntax and semantics not only precisely clear
but also \sphinxstyleemphasis{runnable}. So let’s get started.


\subsection{The Syntax and Semantics of \sphinxstyleemphasis{Simplified} Boolean Expression Language}
\label{\detokenize{08-boolean-algebra:the-syntax-and-semantics-of-simplified-boolean-expression-language}}
We start by considering a simplified language of Boolean expressions:
one with only two literal expressions.  To make it clear that they are
not Boolean values but expressions, we will call them not \sphinxstyleemphasis{true} and
\sphinxstyleemphasis{false} but \sphinxstyleemphasis{bTrue} and \sphinxstyleemphasis{bFalse}.


\subsubsection{Syntax}
\label{\detokenize{08-boolean-algebra:syntax}}
We can represent the syntax of this language in Dafny using what we
call an \sphinxstyleemphasis{inductive data type definition.} A data type defines a set of
values. So what we need to define is a data type whose values are all
and only the valid expressions in the language. The data type defines
the \sphinxstyleemphasis{syntax} of the language.

In the current case, we need a type with only two values, each one of
them representing a valid expression in our language. Here’s how we’d
write it in Dafny.

\begin{sphinxVerbatim}[commandchars=\\\{\}]
datatype Bexp =
     bTrue \textbar{}
     bFalse
\end{sphinxVerbatim}

The definition starts with the \sphinxstyleemphasis{datatype} keyword, followed by the
name of the type being defined (\sphinxstyleemphasis{Bexp}, short for Boolean expression)
then an equals sign, and finally a list of \sphinxstyleemphasis{constructors} separated by
vertical bar characters. The constructors define the ways in which the
values of the type (or \sphinxstyleemphasis{terms}) can be created. Each constructor has a
and can take optional parameters. Here the names are \sphinxstyleemphasis{bTrue} and
\sphinxstyleemphasis{bFalse} and neither takes any parameters.

The only values of an inductive type are those that can be built using
the provided constructors. So the language that we have specified thus
far has only two values, which we take to be the valid expressions in
the language we are specifying, namely \sphinxstyleemphasis{bTrue} and \sphinxstyleemphasis{bFalse}.  That is
all there is to specifying the \sphinxstyleemphasis{syntax} of our simplified language of
Boolean expressions.


\subsubsection{Semantics}
\label{\detokenize{08-boolean-algebra:semantics}}
To give a preview of what is coming, we now specify a semantics for
this language. Speaking informally, we want to associate, to each of
the expressions, a correponding meaning in the form of a Boolean
value.  We do this by defining a \sphinxstyleemphasis{function} that takes an expression
(a value of type \sphinxstyleemphasis{bExp}) as an argument and that returns the Boolean
\sphinxstyleemphasis{value} that the semantics defines as the meaning of that expression.
Here, we want a function that returns Dafny’s Boolean value \sphinxstyleemphasis{true} for
the expression, \sphinxstyleemphasis{bTrue}, and the Boolean value \sphinxstyleemphasis{false} for \sphinxstyleemphasis{bFalse}.

Here’s how we can write this function in Dafny.

\begin{sphinxVerbatim}[commandchars=\\\{\}]
function method bEval(e: bExp): bool
\PYGZob{}
  match e
  \PYGZob{}
      case bTrue =\PYGZgt{} true
      case bFalse =\PYGZgt{} false
  \PYGZcb{}
\PYGZcb{}
\end{sphinxVerbatim}

As a shorhand for \sphinxstyleemphasis{Boolean semantic evaluator} we call it \sphinxstyleemphasis{bEval}. It
takes an expression (a value of type, \sphinxstyleemphasis{bExp}) and returns a Boolean
value.  The function implementation uses an important construct that
is probably new to most readers: a \sphinxstyleemphasis{match} expression. What a match
expression does is to: first determine how a value of an inductive
type was buit, namely what constructor was used and what arguments
were provided (if any) and then to compute a result for the case at
hand.

The match expression starts with the match keyword followed by the
variable naming the value being matched. Then within curly braces
there is a \sphinxstyleemphasis{case} for each constructor for the type of that value.
There are two constructors for the type, \sphinxstyleemphasis{bExp}, so there are two
cases. Each case starts with the \sphinxstyleemphasis{case} keyword, then the name of a
constructor followed by an argument list if the construtor took
parameters. Neither constructor took any parameters, so there is no
need to deal with parameters here. The left side thus determines how
the value was constructed. Each case has an arrow, \sphinxstyleemphasis{=\textgreater{}}, that is
followed by an expression that when evaluated yields the result \sphinxstyleemphasis{for
that case}.

The code here can thus be read as saying, first look at the given
expression, then determine if it was \sphinxstyleemphasis{bTrue} or \sphinxstyleemphasis{bFalse}. In the first
case, return \sphinxstyleemphasis{true}. In the second case, return \sphinxstyleemphasis{false}. That is all
there is to defining a semantics for this simple language.


\subsection{The Syntax of a Complete Boolean Expression Language}
\label{\detokenize{08-boolean-algebra:the-syntax-of-a-complete-boolean-expression-language}}
The real language of Boolean expressions has many more than two valid
expressions, of course. In Dafny’s Boolean expression sub-language,
for example, one can write not only the literal expressions, \sphinxstyleemphasis{true}
and \sphinxstyleemphasis{false}, but also expressions such as \sphinxstyleemphasis{(true \textbar{}\textbar{} false) \&\& (not
false)}.

There is an infinity of such expressions, because given any one or two
valid expressions (starting with \sphinxstyleemphasis{true} and \sphinxstyleemphasis{false}) we can always
build a bigger expression by combing the two given ones with a Boolean
operator. No matter how complex expressions \sphinxstyleemphasis{P} and \sphinxstyleemphasis{Q} are, we can,
for example, always form the even more complex expressions, \sphinxstyleemphasis{!P}, \sphinxstyleemphasis{P
\&\& Q}, and \sphinxstyleemphasis{P \textbar{}\textbar{} Q}, among others.

How can we extend the syntax of our simplified language so that it
specifies the infinity set of well formed expressions in the language
of Boolean expressions? The answer is that we need to add some more
cosntructors. In particular, for each Boolean operator (including
\sphinxstyleemphasis{and, or}, and \sphinxstyleemphasis{not}), we need a a constructor that takes the right
number of smaller expressions as arguments and the builds the right
larger expression.

For example, if \sphinxstyleemphasis{P} and \sphinxstyleemphasis{Q} are arbitrary “smaller” expressions, we
need a consructor to build the expression \sphinxstyleemphasis{P and Q}, a constructor to
build the expression, \sphinxstyleemphasis{P or Q}, and one that can build the expressions
\sphinxstyleemphasis{not P} and \sphinxstyleemphasis{not Q}. Here then is the induction: some constructors of
the \sphinxstyleemphasis{bExp} type will take values of the very type we’re defining as
parameters. And because we’ve defined some values as constants, we
have some expressions to get started with. Here’s how we’d write the
code in Dafny.

\begin{sphinxVerbatim}[commandchars=\\\{\}]
datatype bExp =
     bTrue \textbar{}
     bFalse \textbar{}
     bNot (e: bExp) \textbar{}
     bAnd (e1: bExp, e2: bExp) \textbar{}
     bOr (e1: bExp, e2: bExp)
\end{sphinxVerbatim}

We’ve added three new constructors: one corresponding to each of the
\sphinxstyleemphasis{operator} in Boolean algebra (to keep things simple, we’re dealing
with only three of them here). We have named each constructor in a way
that makes the connection to the corresponding operator clear.

We also see that these new constructors take parameters. The \sphinxstyleemphasis{bNot}
constructor takes a “smaller” expression, \sphinxstyleemphasis{e}, and builds/returns the
expression, \sphinxstyleemphasis{bNot e}, which we will interpret as \sphinxstyleemphasis{not e}, or, as we’d
write it in Dafny, \sphinxstyleemphasis{!e}. Similarly, given expressions, \sphinxstyleemphasis{e1} and \sphinxstyleemphasis{e2},
the \sphinxstyleemphasis{bAnd} and \sphinxstyleemphasis{bOr} operators construct the expressions \sphinxstyleemphasis{bAnd(e1,e2)}
and \sphinxstyleemphasis{bOr(e1,e2)}, respectively, representing \sphinxstyleemphasis{e1 and e2} and \sphinxstyleemphasis{e1 or
e2}, respectively, or, in Dafny syntax, \sphinxstyleemphasis{e1 \&\& e2} and \sphinxstyleemphasis{e1 \textbar{}\textbar{} e2}.

An expression in our \sphinxstyleemphasis{bExp} language for the Dafny expression \sphinxstyleemphasis{(true
\textbar{}\textbar{} false) and (not false))} would be written as \sphinxstyleemphasis{bAnd( (bOr (bTrue,
bFalse)), (bNot bFalse))}. Writing complex expressions like this in
a single line of code can get awkward, to we could also structure the
code like this:

\begin{sphinxVerbatim}[commandchars=\\\{\}]
var T: bExp := bTrue;
var F:      := bFalse;
var P:      := bOr ( T,  F );
var Q       := bNot ( F );
var R       := bAnd ( P, Q );
\end{sphinxVerbatim}


\section{The Semantics of Boolean Expressions: Recursive Evaluation}
\label{\detokenize{08-boolean-algebra:the-semantics-of-boolean-expressions-recursive-evaluation}}
The remaining question, then, is how to give meanings to each of the
expressions in the infinite set of expressions that can be built by
finite numbers of applications of the constructor of our extended
\sphinxstyleemphasis{bExp} type? When we had only two values in the type, it was easy to
write a function that returned the right meaning-value for each of the
two cases. We can’t possibly write a separate case, though, for each
of an infinite number of expressions. The solution lies again in the
realm of recursive functions.

Such a function will simply do mechanically what you the reader would
do if presented with a complex Boolean expression to evaluate.  You
first figure out what operator was applied to what smaller expression
or expressions. You then evaluate those expressions to get values for
them. And finally you apply the Boolean operator to those values to
get a result.

Take the expression, \sphinxstyleemphasis{(true \textbar{}\textbar{} false) and (not false))}, which in our
language is expressed by the term, \sphinxstyleemphasis{bAnd( (bOr (bTrue, bFalse)), (bNot
bFalse))}. First we identify the \sphinxstyleemphasis{constructor} that was used to build
the expression In this case it’s the constructor corresponding to the
\sphinxstyleemphasis{and} operator: \sphinxstyleemphasis{\&\&} in the Dafny expression and the \sphinxstyleemphasis{bAnd} in our own
expression language. What we then do depends on what case has occured.

In the case at hand, we are looking at the constructor for the \sphinxstyleemphasis{and}
operator. It took two smaller expressions as arguments. To enable the
precise expression of the return result, ew given temporary names to
the argument values that were passed to the constructor. We can call
them \sphinxstyleemphasis{e1} and \sphinxstyleemphasis{e2}, for example.
sub-expressions that the operator was applied to.

In this case, \sphinxstyleemphasis{e1} would be \sphinxstyleemphasis{(true \textbar{}\textbar{} false)} and \sphinxstyleemphasis{e2} would be \sphinxstyleemphasis{(not
false)}. To compute the value of the whole expression, we then obtain
Boolean values for each of \sphinxstyleemphasis{e1} and \sphinxstyleemphasis{e2} and then combine them using
the Boolean \sphinxstyleemphasis{and} operator.

The secret is that we get the values for \sphinxstyleemphasis{e1} and \sphinxstyleemphasis{e2} by the very
same means: recursively! Within the evaluation of the overall Boolean
expression, we thus recursively evaluate the subexpressions. Let’s
work through the recursive evaluation of \sphinxstyleemphasis{e1}. It was built using the
\sphinxstyleemphasis{bOr} constructor. That constructor took two arguments, and they were,
in this instance, the literal expressions, \sphinxstyleemphasis{bTrue} and \sphinxstyleemphasis{bFalse}. To
obtain an overall result, we recursively evaluate each of these
expressions and then combine the result using the Boolean \sphinxstyleemphasis{or}
operator. Let’s look at the recursive evaluation of the \sphinxstyleemphasis{bTrue}
expression. It just evaluates to the Boolean value, \sphinxstyleemphasis{true} with no
further recursion, so we’re done with that. The \sphinxstyleemphasis{bFalse} evaluates to
\sphinxstyleemphasis{false}. These two values are then combined using \sphinxstyleemphasis{or} resulting in
a value of \sphinxstyleemphasis{true} for \sphinxstyleemphasis{eq}. A similarly recursive process produces
the value, \sphinxstyleemphasis{true}, for \sphinxstyleemphasis{e2}. (Reason through the details yourself!)
And finally the two Boolean values, \sphinxstyleemphasis{true} and \sphinxstyleemphasis{true} are combined
using Boolean \sphinxstyleemphasis{and}, and a value for the overall expression is thus
computed and returned.

Here’s the Dafny code.

\begin{sphinxVerbatim}[commandchars=\\\{\}]
function method bEval(e: bExp): (r: bool)
\PYGZob{}
    match e
    \PYGZob{}
        case bTrue =\PYGZgt{} true
        case bFase =\PYGZgt{} false
        case bNot(e: bExp) =\PYGZgt{} !bEval(e)
        case bAnd(e1, e2) =\PYGZgt{} bEval(e1) \PYGZam{}\PYGZam{} bEval(e2)
        case bOrEe1, e2) =\PYGZgt{}  bEval(e1) \textbar{}\textbar{} bEval(e2)
    \PYGZcb{}
\PYGZcb{}
\end{sphinxVerbatim}

This code extends our simpler example by adding three cases, one for
each of the new constructor. These constructors took smaller
expression values as arguments, so the corresponding cases have used
parameter lists to temporarily give names (\sphinxstyleemphasis{e1}, \sphinxstyleemphasis{e2}, etc.) to the
arguments that were given when the constructor was orginally used.
These names are then used to write the expressions on the right sides
of the arrows, to compute the final results.

These result-computing expressions use recursive evalation of the
constitute subexpressions to obtain their meanings (actual Boolean
values in Dafny) which they then combine using actual Dafny Boolean
operators to produce final results.

The meaning (Boolean value) of any of the infinite number of Boolean
expressions in the Boolean expression language defined by our syntax
(or \sphinxstyleemphasis{grammar}) can be found by a simple application of our \sphinxstyleemphasis{bEval}
function. To compute the value of \$R\$, above, for example, we just run
\sphinxstyleemphasis{bEval(R)}. For this \sphinxstyleemphasis{R}, this function will without any doubt return
the intended result, \sphinxstyleemphasis{true}.


\section{The Syntax and Semantics of Programming Languages}
\label{\detokenize{08-boolean-algebra:the-syntax-and-semantics-of-programming-languages}}
Syntax defines legal expressions. Semantics give each legal expression
an associated meaning. The meanings of Boolean expressions are Boolean
values. Using exactly the same ideas used here for Boolean expressions
we could not only specify but compute with the syntax semantics of a
language of arithmetic expressions.

Indeed, the same ideas apply to programming language. A programming
language has a syntax. It defines the set of valid “programs” in that
language. A programming language also has a semantics, It specifies
what each such program means. However, th meaning of a program is not
captured in a single value. Rather, it is expressed ina relation that
explains how running the programs transforms any pre-execution state
that satisfies the program preconditions into a post-execution state.


\chapter{9. Propositional Logic}
\label{\detokenize{09-propositional-logic::doc}}\label{\detokenize{09-propositional-logic:propositional-logic}}
A logic is a system for writing, evaluating, and reasoning about truth
statements, or \sphinxstyleemphasis{propositions}. A proposition asserts that some state
of affairs holds in some domain. Truth statements can be particular:
e.g., Tom’s mom is Mary; or general: e.g., Every person has a mother.

In Dafny, a logical language is used to express states of affairs in a
program that are either required or asserted to hold. An example is
aproposition that some variable has a value that is greater than \sphinxstyleemphasis{0}.
When used for program specification and verification, propositions are
taken as descriptions of states of affairs that are required to hold.

It is then the responsibility of the programmer or programming system
to ensure that such states of affairs do hold. The failure to prove
the truth of given propositions in this context leaves possibilities
open that programs fail to satisfy their specifications.

Dafny is a programming system with a reasonably expressive logic based
on what is called \sphinxstyleemphasis{first-order set theory}, with automated and often
reasonably efficient tools for enforcing the truth of propositions
aboutprograms. Ideally programmers change their programs until they
satisfy propositions used to specify pre-conditions, post-conditions,
assertions, and invariants.

In practice, programmers sometimes compromise their work in favor of
pragmatism by weaking specifications until satisfaction can be proved.
But this approach casts the validity of the specifications themselves
into question. A program might satisfy a weakened specification, but
that might no longer guarantee that it will satisfy the requirements
for the system.

Of course logic has far broader applications in computer science than
just in program specification and verification. It is also central to
many artificial intelligence (AI) systems, optimization systems (e.g.,
for finding good travel routes), in the development and analysis of
algorithms, in verification of the functionality of hardware circuits
in digital logic design, and in many other fields.

Up until now in this course, you have seen one compelling application
of formal, mechanized logic in programming, namely for specification
and verification. Having shown one compelling \sphinxstyleemphasis{use case} for logic in
practical computing, we now start a \sphinxstyleemphasis{deeper dive} to understand logic
and formal reasoning more generally.

In this chapter and the next, we explore a simple, useful logic called
\sphinxstyleemphasis{propositional logic}. This is a logic in which the basic elements are
\sphinxstyleemphasis{atomic propositions}\textendash{}that can be broken down no further\textendash{} that can
be taken to be either true or false, and in which propositions can be
composed into larger propositions using logical \sphinxstyleemphasis{connectives}, such as
\sphinxstyleemphasis{and, or, not,} and \sphinxstyleemphasis{implies}.

For example, atomic propositions might be (1) \sphinxstyleemphasis{it is raining}, (2)
\sphinxstyleemphasis{the streets are wet}. If one considers all possible worlds, then in
some of them, each proposition is sometimes true and sometimes false.
A larger proposition, a \sphinxstyleemphasis{formula}, can be formed by combining these
(basic) propositions into a larger one: \sphinxstyleemphasis{it is raining} \sphinxstylestrong{implies}
\sphinxstyleemphasis{the streets are wet}. Another way to say this in English would be,
\sphinxstyleemphasis{whever it is raining, the streets are wet.} This larger proposition
is true in some but not all possible worlds. If it’s not raining, the
proposition correctly describes the world whether the streets are wet
or not, and so is judged to be true. If it is raining and the streets
are wet, it also correctly describes the world, and so is judged to be
world is consistent with the proposition whether the streets are wet
or not. Only in a world in which it is raining and the streets are dry
does the proposition fail to correctly describe the state of affairs,
and so in this world, it’s judged to be false.


\section{The Elements of a Logic}
\label{\detokenize{09-propositional-logic:the-elements-of-a-logic}}
There are many forms of logic, but they all share three basic
elements.  First, a logic provides a \sphinxstyleemphasis{formal language} in which
propositions (truth statements) about states of affairs can be
expressed with mathematical precision. The set of valid expressions in
such a language is defined by a \sphinxstyleemphasis{grammar} that expresses the \sphinxstyleemphasis{syntax}
of the language.

Second, a logic defines a of what is required for a proposition to be
judged true. This definition constitutes what we call the \sphinxstyleemphasis{semantics}
of the language. The semantics of a logic given \sphinxstyleemphasis{meaning} to what are
otherwise abstract mathematical expressions; and do so in particular
by explaining when a given proposition is true or not true.

Finally, a logic provides a set of \sphinxstyleemphasis{inference rules} for deriving new
propositions (conclusions) from given propositions (premises) in ways
that guarantee that if the premises are true, the conclusions will be,
too. The crucial characteristic of inference rules is that although
they are guarantee to \sphinxstyleemphasis{preserve meaning} (in the form of truthfulness
of propositions), they work entirely at the level of syntax.

Each such rule basically says, “if you have a set of premises with
certain syntactic structures, then you can combine them in ways to
derive new propositions with absolute certainty that, if the premises
are true, the conclusion will be, too.  Inference rules are thus rules
for transforming \sphinxstyleemphasis{syntax} in ways that are \sphinxstyleemphasis{semantically sound}. They
allow one to derive \sphinxstyleemphasis{meaningful} new conclusions without ever having
to think about meaning at all.

These ideas bring us to the concept of \sphinxstyleemphasis{proofs} in deductive logic. If
one is given a proposition that is not yet known to be true or not,
and a set of premises known or assumed to be true, a proof is simply a
set of applications of availabile inference rules in a way that, step
by step, connects the premises \sphinxstyleemphasis{syntactically} to the conclusion.

A key property of such a proof is that it can be checked mechanically,
without any consideration of \sphinxstyleemphasis{semantics} (meaning) to determine if it
is a valid proof or not. It is a simple matter at each step to check
whether a given inference rule was applied correctly to convert one
collection of propositions into another, and thus to check whether
\sphinxstyleemphasis{chains} of inference rules properly connect premises to conclusions.

For example, a simple inference rule called \sphinxstyleemphasis{modus ponens} states that
if \sphinxstyleemphasis{P} and \sphinxstyleemphasis{Q} are propositions and if one has as premises that (1)
\sphinxstyleemphasis{P} is true*, and (2) \sphinxstyleemphasis{if P is true then Q is true}, then one can
deduce that \sphinxstyleemphasis{Q is true}. This rule is applicable \sphinxstyleemphasis{no matter what} the
propositions \sphinxstyleemphasis{P} and \sphinxstyleemphasis{Q} are. It thus encodes a general rule of sound
reasoning.

A logic enables \sphinxstyleemphasis{semantically sound} “reasoning” by way of syntactic
transformations alone. And a wonderful thing about syntax is that it
is relatively easy to mechanize with software. What this means is that
we can implement systems that can reasoning \sphinxstyleemphasis{meaningfully} based on
syntactic transformation rules alone.

Note: Modern logic initially developed by Frege as a ” formula
language for pure though,t modeled on that of arithmetic,” and later
elaborated by Russel, Peano, and others as a language in which, in
turn, to establish completely formal foundations for mathematics.


\section{Using Logic in Practice}
\label{\detokenize{09-propositional-logic:using-logic-in-practice}}
To use a logic for practical purposes, one must (1) understand how to
represent states of affairs in the domain of discourse of interest as
expressions in the logical language of the logic, and (2) havee some
means of evaluating the truth values of the resulting expressions. In
Dafny, one must understand the logical language in which assertions
and related constructs (such as pre- and post-conditions) are written.

In many cases\textendash{}the magic of an automated verifier such as Dafny\textendash{}a
programmer can rely on Dafny to evaluate truth values of assertions
automatically. When Dafny is unable to verify the truth of a claim,
however, the programmer will also have to understand something about
the way that truth is ascertained in the logic, so as to be able to
provide Dafny with the help it might need to be able to complete its
verification task.

In this chapter, we take a major step toward understanding logic and
proofs by introducing the language \sphinxstyleemphasis{propositional logic} and a means
of evaluating the truth of any sentence in the language. The language
is closely related to the language of Boolean expressions introduced
in the last chapter. The main syntactic difference is that we add a
notion of \sphinxstyleemphasis{propositional variables}. We will defined the semantics of
this language by introducing the concept of an \sphinxstyleemphasis{interpration}, which
specifies a Boolean truth value for each such variable. We will then
evaluate the truth value of an expression \sphinxstyleemphasis{given an interpration for
the proposition variables in that expression} by replacing each of the
variables with its corresponding Boolean value and then using our
Boolean expression evaluator to determing the truth value of the
expression.

We will also note that this formulation gives rise to an important new
set of logical problems. Given an expression, does there exist an
interpretation that makes that expression evaluate to true? Do all
interpretations make it value to true? Can it be there there are no
interpretations that make a given expression evaluate to true?  And,
finally, are there \sphinxstyleemphasis{efficient} algorithms for \sphinxstyleemphasis{deciding} whether or
not the answer to any such question is yes or no.


\section{Propositional Logic}
\label{\detokenize{09-propositional-logic:id1}}
The rest of this chapter illustrates and further develops these ideas
using Boolean algebra, and a language of Boolean expressions, as a
case study in precise definition of the syntax (expression structure)
and semantics (expression evaluation) of a simple formal language: of
Boolean expressions containing Boolean variables.

To illustrate the potential utility of this language and its semantics
we will define three related \sphinxstyleemphasis{decision problems}. A decision problem
is a \sphinxstyleemphasis{kind} of problem for which there is an algorithm that can solve
any instance of the problem. The three decision problems we will study
start with a Boolean expression, one that can contain variables, and
ask where there is an assignment of \sphinxstyleemphasis{true} and \sphinxstyleemphasis{false} values to the
variables in the expression to make the overall expression evaluate to
\sphinxstyleemphasis{true}.

Here’s an example. Suppose you’re given the Boolean expression,
\((P \lor Q) \land (\lnot R)\). The top-level operator is
\sphinxstyleemphasis{and}. The whole expression thus evaluates to \sphinxstyleemphasis{true} if and only if
both subexpressions do: \((P \lor Q)\) and \(\land (\lnot
R)\), respectively. The first, \((P \lor Q)\), evaluates to \sphinxstyleemphasis{true}
if either of the variables, \sphinxstyleemphasis{P} and \sphinxstyleemphasis{Q}, are set to true. The second
evaluates to true if and only if the variable \sphinxstyleemphasis{R} is false. There are
thus settings of the variables that make the formula true. In each of
them, \sphinxstyleemphasis{R} is \sphinxstyleemphasis{false}, and either or both of \sphinxstyleemphasis{P} and \sphinxstyleemphasis{Q} are set to
true.

Given a Boolean expression with variables, an \sphinxstyleemphasis{interpretation} for
that expression is a binding of the variables in that expression to
corresponding Boolean values. A Boolean expression with no variables
is like a proposition: it is true or false on its own. An expression
with one or more variables will be true or false depending on how the
variables are used in the expression.

An interpretation that makes such a formula true is called a \sphinxstyleemphasis{model}.
The problem of finding a model is called, naturally enough, the model
finding problem, and the problem of finding \sphinxstyleemphasis{all} models that make a
Boolean expression true, the \sphinxstyleemphasis{model enumeration} or \sphinxstyleemphasis{model counting}
problem.

The first major \sphinxstyleemphasis{decision problem} that we identify is, for any given
Boolean expression, to determine whether it is \sphinxstyleemphasis{satisfiable}. That is,
is there at least one interpretation (assignment of truth values to
the variables in the expression that makes the expression evaluate to
\sphinxstyleemphasis{true}?  We saw, for example, that the expression, \((P \lor Q)
\land (\lnot R)\) is satifiable, and, moreover, that \(\{ (P,
true), (Q, false), (R, false) \}\) is a (one of three) interpretations
that makes the expression true.

Such an interpretation is called a \sphinxstyleemphasis{model}. The problem of finding a
model (if there is one), and thereby showing that an expression is
satisfiable, is naturally enough called the* model finding* problem.

A second problem is to determine whether a Boolean expression is
\sphinxstyleemphasis{valid}. An expression is valid if \sphinxstyleemphasis{every} interpretation makes the
expression true. For example, the Boolean expression \(P \lor
\neg P\) is always true. If \sphinxstyleemphasis{P} is set to true, the formula becomes
\(true \lor false\). If \sphinxstyleemphasis{P} is set to false, the formula is then
\(true \lor false\). Those are the only two interpretations and
under either of them, the resulting expression evaluates to true.

A third related problem is to determine whether a Boolean expression
is it \sphinxstyleemphasis{unsatisfiable}? This case occurs when there is \sphinxstyleemphasis{no} combination
of variable values makes the expression true. The expression \(P
\land \neg P\) is unsatisfiable, for example. There is no value of \$P\$
(either \sphinxstyleemphasis{true} or \sphinxstyleemphasis{false}) that makes the resulting formula true.

These decision problems are all solvable. There are algorithms that in
a finite number of steps can determine answers to all of them. In the
worst case, one need only look at all possible combinations of true
and false values for each of the (finite number of) variables in an
expression. If there are \sphinxstyleemphasis{n} variables, that is at most \(2^n\)
combinations of such values. Checking the value of an expression for
each of these interpretations will determine whether it’s satisfiable,
unsatisfiable, or valid. In this chapter, we will see how these ideas
can be translated into runnable code.

The much more interesting question is whether there is a fundamentally
more efficient approach than checking all possible interpretations: an
approach with a cost that increases \sphinxstyleemphasis{exponentially} in the number of
variables in an expression. This is the greatest open question in all
of computer science, and one of the greatest open questions in all of
mathematics.

So let’s see how it all works. The rest of this chapter first defines
a \sphinxstyleemphasis{syntax} for Boolean expressions. Then it defines a \sphinxstyleemphasis{semantics} in
the form of a procedure for \sphinxstyleemphasis{evaluating} any given Boolean expression
given a corresponding \sphinxstyleemphasis{interpretation}, i.e., a mapping from variables
in the expression to corresponding Boolean values. Next we define a
procedure that, for any given set of Boolean variables, computes and
returns a list of \sphinxstyleemphasis{all} interpretations. We also define a procedure
that, given any Boolean expression returns the set of variables in the
expression. For ths set we calculate the set of all interpretations.
Finally, by evaluating the expression on each such interpretation, we
decide whether the expression is satisfiable, unsatisfiable, or valid.

Along the way, we will meet \sphinxstyleemphasis{inductive definitions} as a fundamental
approach to concisely specifying languages with a potentially infinite
number of expressions, and the \sphinxstyleemphasis{match} expression for dealing with
values of inductively defined types. We will also see uses of several
of Dafny’s built-in abstract data types, including sets, sequences,
and maps. So let’s get going.


\section{Syntax}
\label{\detokenize{09-propositional-logic:syntax}}
Any basic introduction to programming will have made it clear that
there is an infinite set of Boolean expressions. First, we can take
the Boolean values, \sphinxstyleemphasis{true} and \sphinxstyleemphasis{false}, as \sphinxstyleemphasis{literal} expressions.
Second, we can take \sphinxstyleemphasis{Boolean variables}, such as \sphinxstyleemphasis{P} or \sphinxstyleemphasis{Q}, as a
Boolean \sphinxstyleemphasis{variable} expressions. Finally, we take take each Boolean
operator as having an associated expression constructor that takes one
or more smaller \sphinxstyleemphasis{Boolean expressions} as arguments.

Notice that in this last step, we introduced the idea of constructing
larger Boolean expressions out of smaller ones. We are thus defining
the set of all Boolean expressions \sphinxstyleemphasis{inductively}. For example, if \sphinxstyleemphasis{P}
is a Boolean variable expression, then we can construct a valid larger
expression, \(P \land true\) to express the conjunction of the
value of \sphinxstyleemphasis{P} (whatever it might be( with the value, \sphinxstyleemphasis{true}. From here
we could build the larger expression, \sphinxstyleemphasis{P lor (P land true)}, and so
on, ad infinitum.

We define an infinite set of “variables” as terms of the form
mkVar(s), where s, astring, represents the name of the variable. The
term mkVar(“P”), for example, is our way of writing “the var named P.”

\begin{sphinxVerbatim}[commandchars=\\\{\}]
datatype Bvar = mkVar(name: string)
\end{sphinxVerbatim}

Here’s the definition of the \sphinxstyleemphasis{syntax}:

\begin{sphinxVerbatim}[commandchars=\\\{\}]
datatype Bexp =
    litExp (b: bool) \textbar{}
    varExp (v: Bvar) \textbar{}
    notExp (e: Bexp) \textbar{}
    andExp (e1: Bexp, e2: Bexp) \textbar{}
    orExp (e1: Bexp, e2: Bexp)
\end{sphinxVerbatim}

Boolean expresions, as we’ve defined them here, are like propositions
with paramaters. The parameters are the variables. Depending on how we
assign them \sphinxstyleemphasis{true} and \sphinxstyleemphasis{false} values, the overall proposition might be
rendered true or false.


\section{Interpretations}
\label{\detokenize{09-propositional-logic:interpretations}}
Evaluate a Boolean expression in a given environment.  The recursive
structure of this algorithm reflects the inductive structure of the
expressions we’ve defined.

\begin{sphinxVerbatim}[commandchars=\\\{\}]
type interp = map\PYGZlt{}Bvar, bool\PYGZgt{}
\end{sphinxVerbatim}


\section{Evaluation}
\label{\detokenize{09-propositional-logic:evaluation}}
\begin{sphinxVerbatim}[commandchars=\\\{\}]
function method Beval(e: Bexp, i: interp): (r: bool)
\PYGZob{}
    match e
    \PYGZob{}
        case litExp(b: bool) =\PYGZgt{} b
        case varExp(v: Bvar) =\PYGZgt{} lookup(v,i)
        case notExp(e1: Bexp) =\PYGZgt{} !Beval(e1,i)
        case andExp(e1, e2) =\PYGZgt{} Beval(e1,i) \PYGZam{}\PYGZam{} Beval(e2, i)
        case orExp(e1, e2) =\PYGZgt{}  Beval(e1, i) \textbar{}\textbar{} Beval(e2, i)
    \PYGZcb{}
\PYGZcb{}
\end{sphinxVerbatim}

\}

Lookup value of given variable, v, in a given interpretation, i. If
there is not value for v in i, then just return false. This is not a
great design, in that a return of false could mean one of two things,
and it’s ambiguous: either the value of the variable really is false,
or it’s undefined.  For now, though, it’s good enough to illustate our
main points.

\begin{sphinxVerbatim}[commandchars=\\\{\}]
function method lookup(v: Bvar, i: interp): bool
\PYGZob{}
    if (v in i) then i[v]
    else false
\PYGZcb{}
\end{sphinxVerbatim}

Now that we know the basic values and operations of Boolean algebra,
we can be precise about the forms of and valid ways of transforming
\sphinxstyleemphasis{Boolean expressions.} For example, we’ve seen that we can transform
the expression \sphinxstyleemphasis{true and true} into \sphinxstyleemphasis{true}. But what about \sphinxstyleemphasis{true and
((false xor true) or (not (false implies true)))}?

To make sense of such expressions, we need to define what it means for
one to be well formed, and how to evaluate any such well formed
expressions by transforming it repeatedly into simpler forms but in
ways that preserve its meaning until we reach a single Boolean value.


\section{Models}
\label{\detokenize{09-propositional-logic:models}}

\section{Satisfiability, Validity}
\label{\detokenize{09-propositional-logic:satisfiability-validity}}
We can now characterize the most important \sphinxstyleemphasis{open question} (unsolved
mathematical problem) in computer science.  Is there an \sphinxstyleemphasis{efficient}
algorithm for determining whether any given Boolean formula is
satisfiable?

whether there is a combination of Boolean
variable values that makes any given Boolean expression true is the
most important unsolved problem in computer science. We currently do
not know of a solution that with runtime complexity that is better
than exponential the number of variables in an expression.  It’s easy
to determine whether an assignment of values to variables does the
trick: just evaluate the expression with those values for the
variables. But \sphinxstyleemphasis{finding} such a combination today requires, for the
hardest of these problems, trying all :math:\sphinxcode{2\textasciicircum{}n} combinations of
Boolean values for \sphinxstyleemphasis{n} variables.

At the same time, we do not know that there is \sphinxstyleemphasis{not} a more efficient
algorithm. Many experts would bet that there isn’t one, but until we
know for sure, there is a tantalizing possibility that someone someday
will find an \sphinxstyleemphasis{efficient decision procedure} for Boolean satisfiability.

To close this exploration of computational complexity theory, we’ll
just note that we solved an instances of another related problem: not
only to determine whether there is at least one (whether \sphinxstyleemphasis{there
exists}) at least one combination of variable values that makes the
expression true, but further determining how many different ways there
are to do it.

Researchers and advanced practitioners of logic and computation
sometimes use the word \sphinxstyleemphasis{model} to refer to a combination of variable
values that makes an expression true. The problem of finding a Boolean
expression that \sphinxstyleemphasis{satisfies} a Boolean formula is thus somtetimes
called the \sphinxstyleemphasis{model finding} problem. By contrast, the problem of
determining how many ways there are to satisfy a Boolean expression is
called the \sphinxstyleemphasis{model counting} problem.

Solutions to these problems have a vast array of practical uses.  As
one still example, many logic puzzles can be represented as Boolean
expressions, and a model finder can be used to determine whether there
are any “solutions”, if so, what one solution is.


\section{Logical Consequence}
\label{\detokenize{09-propositional-logic:logical-consequence}}
Finally, logic consequence. A set of logical propositions, premises,
is said to entail another, a conclusion, if in every interpretation
where all of the premises are true the conclusion is also true. See
the file, consequence.dfy, for a consequence checker that works by
exhaustive checking of all interpretations. \textless{}More to come\textgreater{}.


\chapter{10. Natural Deduction}
\label{\detokenize{10-natural-deduction::doc}}\label{\detokenize{10-natural-deduction:natural-deduction}}
One way
to defineand (3) a set of \sphinxstyleemphasis{inference} rules that define ways that one
can transform one set of expressions (premises) into another (a
conclusion) in such a manner that whenver all the premises are true,
the conclusion will be, too.

Why would anyone care about rules for transforming expressions in
abstract languages? Well, it turns out that \sphinxstyleemphasis{syntactic} reasoning is
pretty useful. The idea is that we represent a real-world phenomenon
symbolically, in such a language, so the abstract sentence means
something in the real world.

Now comes the key idea: if we imbue mathematical expressions with
real-world meanings and then transform these expression in accordance
with valid rules for acceptable transformations of such expressions,
then the resulting expressions will also be meaningful.

A logic, then, is basically a formal language, one that defines a set
of well formed expressions, and that provides a set of \sphinxstyleemphasis{inference}
rules for taking a set of expressions as premises and deriving another
one as a consequence. Mathematical logic allows us to replace human
mental reasoning with the mechanical \sphinxstyleemphasis{transformation of symbolic
expressions}.


\chapter{Indices and tables}
\label{\detokenize{index::doc}}\label{\detokenize{index:indices-and-tables}}\begin{itemize}
\item {} 
\DUrole{xref,std,std-ref}{genindex}

\item {} 
\DUrole{xref,std,std-ref}{modindex}

\item {} 
\DUrole{xref,std,std-ref}{search}

\end{itemize}



\renewcommand{\indexname}{Index}
\printindex
\end{document}